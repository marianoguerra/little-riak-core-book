% Generated by Sphinx.
\def\sphinxdocclass{report}
\documentclass[letterpaper,10pt,english]{sphinxmanual}
\usepackage[utf8]{inputenc}
\DeclareUnicodeCharacter{00A0}{\nobreakspace}
\usepackage{cmap}
\usepackage[T1]{fontenc}
\usepackage{babel}
\usepackage{times}
\usepackage[Bjarne]{fncychap}
\usepackage{longtable}
\usepackage{sphinx}
\usepackage{multirow}


\title{Little Riak Core Book Documentation}
\date{October 30, 2015}
\release{1.0}
\author{Mariano Guerra}
\newcommand{\sphinxlogo}{}
\renewcommand{\releasename}{Release}
\makeindex

\makeatletter
\def\PYG@reset{\let\PYG@it=\relax \let\PYG@bf=\relax%
    \let\PYG@ul=\relax \let\PYG@tc=\relax%
    \let\PYG@bc=\relax \let\PYG@ff=\relax}
\def\PYG@tok#1{\csname PYG@tok@#1\endcsname}
\def\PYG@toks#1+{\ifx\relax#1\empty\else%
    \PYG@tok{#1}\expandafter\PYG@toks\fi}
\def\PYG@do#1{\PYG@bc{\PYG@tc{\PYG@ul{%
    \PYG@it{\PYG@bf{\PYG@ff{#1}}}}}}}
\def\PYG#1#2{\PYG@reset\PYG@toks#1+\relax+\PYG@do{#2}}

\expandafter\def\csname PYG@tok@gd\endcsname{\def\PYG@tc##1{\textcolor[rgb]{0.63,0.00,0.00}{##1}}}
\expandafter\def\csname PYG@tok@gu\endcsname{\let\PYG@bf=\textbf\def\PYG@tc##1{\textcolor[rgb]{0.50,0.00,0.50}{##1}}}
\expandafter\def\csname PYG@tok@gt\endcsname{\def\PYG@tc##1{\textcolor[rgb]{0.00,0.27,0.87}{##1}}}
\expandafter\def\csname PYG@tok@gs\endcsname{\let\PYG@bf=\textbf}
\expandafter\def\csname PYG@tok@gr\endcsname{\def\PYG@tc##1{\textcolor[rgb]{1.00,0.00,0.00}{##1}}}
\expandafter\def\csname PYG@tok@cm\endcsname{\let\PYG@it=\textit\def\PYG@tc##1{\textcolor[rgb]{0.25,0.50,0.56}{##1}}}
\expandafter\def\csname PYG@tok@vg\endcsname{\def\PYG@tc##1{\textcolor[rgb]{0.73,0.38,0.84}{##1}}}
\expandafter\def\csname PYG@tok@m\endcsname{\def\PYG@tc##1{\textcolor[rgb]{0.13,0.50,0.31}{##1}}}
\expandafter\def\csname PYG@tok@mh\endcsname{\def\PYG@tc##1{\textcolor[rgb]{0.13,0.50,0.31}{##1}}}
\expandafter\def\csname PYG@tok@cs\endcsname{\def\PYG@tc##1{\textcolor[rgb]{0.25,0.50,0.56}{##1}}\def\PYG@bc##1{\setlength{\fboxsep}{0pt}\colorbox[rgb]{1.00,0.94,0.94}{\strut ##1}}}
\expandafter\def\csname PYG@tok@ge\endcsname{\let\PYG@it=\textit}
\expandafter\def\csname PYG@tok@vc\endcsname{\def\PYG@tc##1{\textcolor[rgb]{0.73,0.38,0.84}{##1}}}
\expandafter\def\csname PYG@tok@il\endcsname{\def\PYG@tc##1{\textcolor[rgb]{0.13,0.50,0.31}{##1}}}
\expandafter\def\csname PYG@tok@go\endcsname{\def\PYG@tc##1{\textcolor[rgb]{0.20,0.20,0.20}{##1}}}
\expandafter\def\csname PYG@tok@cp\endcsname{\def\PYG@tc##1{\textcolor[rgb]{0.00,0.44,0.13}{##1}}}
\expandafter\def\csname PYG@tok@gi\endcsname{\def\PYG@tc##1{\textcolor[rgb]{0.00,0.63,0.00}{##1}}}
\expandafter\def\csname PYG@tok@gh\endcsname{\let\PYG@bf=\textbf\def\PYG@tc##1{\textcolor[rgb]{0.00,0.00,0.50}{##1}}}
\expandafter\def\csname PYG@tok@ni\endcsname{\let\PYG@bf=\textbf\def\PYG@tc##1{\textcolor[rgb]{0.84,0.33,0.22}{##1}}}
\expandafter\def\csname PYG@tok@nl\endcsname{\let\PYG@bf=\textbf\def\PYG@tc##1{\textcolor[rgb]{0.00,0.13,0.44}{##1}}}
\expandafter\def\csname PYG@tok@nn\endcsname{\let\PYG@bf=\textbf\def\PYG@tc##1{\textcolor[rgb]{0.05,0.52,0.71}{##1}}}
\expandafter\def\csname PYG@tok@no\endcsname{\def\PYG@tc##1{\textcolor[rgb]{0.38,0.68,0.84}{##1}}}
\expandafter\def\csname PYG@tok@na\endcsname{\def\PYG@tc##1{\textcolor[rgb]{0.25,0.44,0.63}{##1}}}
\expandafter\def\csname PYG@tok@nb\endcsname{\def\PYG@tc##1{\textcolor[rgb]{0.00,0.44,0.13}{##1}}}
\expandafter\def\csname PYG@tok@nc\endcsname{\let\PYG@bf=\textbf\def\PYG@tc##1{\textcolor[rgb]{0.05,0.52,0.71}{##1}}}
\expandafter\def\csname PYG@tok@nd\endcsname{\let\PYG@bf=\textbf\def\PYG@tc##1{\textcolor[rgb]{0.33,0.33,0.33}{##1}}}
\expandafter\def\csname PYG@tok@ne\endcsname{\def\PYG@tc##1{\textcolor[rgb]{0.00,0.44,0.13}{##1}}}
\expandafter\def\csname PYG@tok@nf\endcsname{\def\PYG@tc##1{\textcolor[rgb]{0.02,0.16,0.49}{##1}}}
\expandafter\def\csname PYG@tok@si\endcsname{\let\PYG@it=\textit\def\PYG@tc##1{\textcolor[rgb]{0.44,0.63,0.82}{##1}}}
\expandafter\def\csname PYG@tok@s2\endcsname{\def\PYG@tc##1{\textcolor[rgb]{0.25,0.44,0.63}{##1}}}
\expandafter\def\csname PYG@tok@vi\endcsname{\def\PYG@tc##1{\textcolor[rgb]{0.73,0.38,0.84}{##1}}}
\expandafter\def\csname PYG@tok@nt\endcsname{\let\PYG@bf=\textbf\def\PYG@tc##1{\textcolor[rgb]{0.02,0.16,0.45}{##1}}}
\expandafter\def\csname PYG@tok@nv\endcsname{\def\PYG@tc##1{\textcolor[rgb]{0.73,0.38,0.84}{##1}}}
\expandafter\def\csname PYG@tok@s1\endcsname{\def\PYG@tc##1{\textcolor[rgb]{0.25,0.44,0.63}{##1}}}
\expandafter\def\csname PYG@tok@gp\endcsname{\let\PYG@bf=\textbf\def\PYG@tc##1{\textcolor[rgb]{0.78,0.36,0.04}{##1}}}
\expandafter\def\csname PYG@tok@sh\endcsname{\def\PYG@tc##1{\textcolor[rgb]{0.25,0.44,0.63}{##1}}}
\expandafter\def\csname PYG@tok@ow\endcsname{\let\PYG@bf=\textbf\def\PYG@tc##1{\textcolor[rgb]{0.00,0.44,0.13}{##1}}}
\expandafter\def\csname PYG@tok@sx\endcsname{\def\PYG@tc##1{\textcolor[rgb]{0.78,0.36,0.04}{##1}}}
\expandafter\def\csname PYG@tok@bp\endcsname{\def\PYG@tc##1{\textcolor[rgb]{0.00,0.44,0.13}{##1}}}
\expandafter\def\csname PYG@tok@c1\endcsname{\let\PYG@it=\textit\def\PYG@tc##1{\textcolor[rgb]{0.25,0.50,0.56}{##1}}}
\expandafter\def\csname PYG@tok@kc\endcsname{\let\PYG@bf=\textbf\def\PYG@tc##1{\textcolor[rgb]{0.00,0.44,0.13}{##1}}}
\expandafter\def\csname PYG@tok@c\endcsname{\let\PYG@it=\textit\def\PYG@tc##1{\textcolor[rgb]{0.25,0.50,0.56}{##1}}}
\expandafter\def\csname PYG@tok@mf\endcsname{\def\PYG@tc##1{\textcolor[rgb]{0.13,0.50,0.31}{##1}}}
\expandafter\def\csname PYG@tok@err\endcsname{\def\PYG@bc##1{\setlength{\fboxsep}{0pt}\fcolorbox[rgb]{1.00,0.00,0.00}{1,1,1}{\strut ##1}}}
\expandafter\def\csname PYG@tok@mb\endcsname{\def\PYG@tc##1{\textcolor[rgb]{0.13,0.50,0.31}{##1}}}
\expandafter\def\csname PYG@tok@ss\endcsname{\def\PYG@tc##1{\textcolor[rgb]{0.32,0.47,0.09}{##1}}}
\expandafter\def\csname PYG@tok@sr\endcsname{\def\PYG@tc##1{\textcolor[rgb]{0.14,0.33,0.53}{##1}}}
\expandafter\def\csname PYG@tok@mo\endcsname{\def\PYG@tc##1{\textcolor[rgb]{0.13,0.50,0.31}{##1}}}
\expandafter\def\csname PYG@tok@kd\endcsname{\let\PYG@bf=\textbf\def\PYG@tc##1{\textcolor[rgb]{0.00,0.44,0.13}{##1}}}
\expandafter\def\csname PYG@tok@mi\endcsname{\def\PYG@tc##1{\textcolor[rgb]{0.13,0.50,0.31}{##1}}}
\expandafter\def\csname PYG@tok@kn\endcsname{\let\PYG@bf=\textbf\def\PYG@tc##1{\textcolor[rgb]{0.00,0.44,0.13}{##1}}}
\expandafter\def\csname PYG@tok@o\endcsname{\def\PYG@tc##1{\textcolor[rgb]{0.40,0.40,0.40}{##1}}}
\expandafter\def\csname PYG@tok@kr\endcsname{\let\PYG@bf=\textbf\def\PYG@tc##1{\textcolor[rgb]{0.00,0.44,0.13}{##1}}}
\expandafter\def\csname PYG@tok@s\endcsname{\def\PYG@tc##1{\textcolor[rgb]{0.25,0.44,0.63}{##1}}}
\expandafter\def\csname PYG@tok@kp\endcsname{\def\PYG@tc##1{\textcolor[rgb]{0.00,0.44,0.13}{##1}}}
\expandafter\def\csname PYG@tok@w\endcsname{\def\PYG@tc##1{\textcolor[rgb]{0.73,0.73,0.73}{##1}}}
\expandafter\def\csname PYG@tok@kt\endcsname{\def\PYG@tc##1{\textcolor[rgb]{0.56,0.13,0.00}{##1}}}
\expandafter\def\csname PYG@tok@sc\endcsname{\def\PYG@tc##1{\textcolor[rgb]{0.25,0.44,0.63}{##1}}}
\expandafter\def\csname PYG@tok@sb\endcsname{\def\PYG@tc##1{\textcolor[rgb]{0.25,0.44,0.63}{##1}}}
\expandafter\def\csname PYG@tok@k\endcsname{\let\PYG@bf=\textbf\def\PYG@tc##1{\textcolor[rgb]{0.00,0.44,0.13}{##1}}}
\expandafter\def\csname PYG@tok@se\endcsname{\let\PYG@bf=\textbf\def\PYG@tc##1{\textcolor[rgb]{0.25,0.44,0.63}{##1}}}
\expandafter\def\csname PYG@tok@sd\endcsname{\let\PYG@it=\textit\def\PYG@tc##1{\textcolor[rgb]{0.25,0.44,0.63}{##1}}}

\def\PYGZbs{\char`\\}
\def\PYGZus{\char`\_}
\def\PYGZob{\char`\{}
\def\PYGZcb{\char`\}}
\def\PYGZca{\char`\^}
\def\PYGZam{\char`\&}
\def\PYGZlt{\char`\<}
\def\PYGZgt{\char`\>}
\def\PYGZsh{\char`\#}
\def\PYGZpc{\char`\%}
\def\PYGZdl{\char`\$}
\def\PYGZhy{\char`\-}
\def\PYGZsq{\char`\'}
\def\PYGZdq{\char`\"}
\def\PYGZti{\char`\~}
% for compatibility with earlier versions
\def\PYGZat{@}
\def\PYGZlb{[}
\def\PYGZrb{]}
\makeatother

\renewcommand\PYGZsq{\textquotesingle}

\begin{document}

\maketitle
\tableofcontents
\phantomsection\label{index::doc}



\chapter{Starting}
\label{starting:little-riak-core-book}\label{starting::doc}\label{starting:starting}
We are going to build a system on top of riak\_core, for that we will use some
tools and to avoid copy paste and boilerplate we will use a template to get
started.


\section{Tools}
\label{starting:tools}\begin{itemize}
\item {} 
\href{http://www.rebar3.org/docs/getting-started}{rebar3}: The build tool, click on the link to see how to install it.

\item {} 
\href{http://www.erlang.org/}{erlang}: Our programming language, we assume erlang version to be at least 17.0

\end{itemize}

I also assume a unix-like environment with a shell similar to bash or zsh.


\section{Installing the Template}
\label{starting:installing-the-template}
At this point you should have \href{http://www.erlang.org/}{erlang} and \href{http://www.rebar3.org/docs/getting-started}{rebar3} installed, now let's install the
template we are going to use.

\begin{Verbatim}[commandchars=\\\{\}]
mkdir \PYGZhy{}p \PYGZti{}/.config/rebar3/templates
git clone https://github.com/marianoguerra/rebar3\PYGZus{}template\PYGZus{}riak\PYGZus{}core/ \PYGZti{}/.config/rebar3/templates/rebar3\PYGZus{}template\PYGZus{}riak\PYGZus{}core
\end{Verbatim}

We just created the folder \emph{\textasciitilde{}/.config/rebar3/templates} for the templates in
case it wasn't there and cloned our template inside of it.

You can read more about \href{http://www.rebar3.org/docs/using-templates}{rebar3 templates here}.


\section{Creating our Project}
\label{starting:creating-our-project}
Now that we have our tools and our template installed we can start by asking
rebar3 to create a new project we will call tanodb using the \href{https://github.com/basho/riak\_core}{riak\_core} template
we just installed:

\begin{Verbatim}[commandchars=\\\{\}]
rebar3 new rebar3\PYGZus{}riak\PYGZus{}core \PYG{n+nv}{name}\PYG{o}{=}tanodb
\end{Verbatim}

If it fails saying it can't find rebar3 check that it's in your \emph{\$PATH}
environment variable.

The output should be something like this:

\begin{Verbatim}[commandchars=\\\{\}]
===\PYGZgt{} Writing tanodb/apps/tanodb/src/tanodb.app.src
===\PYGZgt{} Writing tanodb/apps/tanodb/src/tanodb.erl
===\PYGZgt{} Writing tanodb/apps/tanodb/src/tanodb\PYGZus{}app.erl
===\PYGZgt{} Writing tanodb/apps/tanodb/src/tanodb\PYGZus{}sup.erl
===\PYGZgt{} Writing tanodb/apps/tanodb/src/tanodb\PYGZus{}console.erl
===\PYGZgt{} Writing tanodb/apps/tanodb/src/tanodb\PYGZus{}vnode.erl
===\PYGZgt{} Writing tanodb/rebar.config
===\PYGZgt{} Writing tanodb/.editorconfig
===\PYGZgt{} Writing tanodb/.gitignore
===\PYGZgt{} Writing tanodb/README.rst
===\PYGZgt{} Writing tanodb/Makefile
===\PYGZgt{} Writing tanodb/config/nodetool
===\PYGZgt{} Writing tanodb/config/extended\PYGZus{}bin
===\PYGZgt{} Writing tanodb/config/admin\PYGZus{}bin
===\PYGZgt{} Writing tanodb/config/config.schema
===\PYGZgt{} Writing tanodb/config/advanced.config
===\PYGZgt{} Writing tanodb/config/sys.config
===\PYGZgt{} Writing tanodb/config/vars.config
===\PYGZgt{} Writing tanodb/config/vars\PYGZus{}dev1.config
===\PYGZgt{} Writing tanodb/config/vars\PYGZus{}dev2.config
===\PYGZgt{} Writing tanodb/config/vars\PYGZus{}dev3.config
===\PYGZgt{} Writing tanodb/config/vm.args
===\PYGZgt{} Writing tanodb/config/dev1\PYGZus{}vm.args
===\PYGZgt{} Writing tanodb/config/dev2\PYGZus{}vm.args
===\PYGZgt{} Writing tanodb/config/dev3\PYGZus{}vm.args
\end{Verbatim}


\section{Building and Running}
\label{starting:building-and-running}
Before explaining what the files mean so you get an idea what just happened
let's run it!

\begin{Verbatim}[commandchars=\\\{\}]
\PYG{n+nb}{cd }tanodb
rebar3 release
rebar3 run
\end{Verbatim}

\emph{rebar3 release} asks rebar3 to build a release of our project, for that it uses a tool called \href{https://github.com/erlware/relx}{relx}.

The initial build may take a while since it has to fetch all the dependencies
and build them.

After the release is built (you can check the result by inspecting the folder
\emph{\_build/default/rel/tanodb/}) we can run it, for this we use a rebar3 plugin
called \href{https://github.com/tsloughter/rebar3\_run}{rebar3\_run}

When we run \emph{rebar3 run} we get some noisy output that should end with something like this:

\begin{Verbatim}[commandchars=\\\{\}]
\PYG{g+go}{Eshell V7.0  (abort with \PYGZca{}G)}
\PYG{g+go}{(tanodb@127.0.0.1)1\PYGZgt{}}
\end{Verbatim}

This is the erlang shell, something like a REPL connected to our system,
we now can test our system by calling \emph{tanodb:ping()} on it.

\begin{Verbatim}[commandchars=\\\{\}]
\PYG{g+go}{(tanodb@127.0.0.1)1\PYGZgt{} tanodb:ping().}
\PYG{g+go}{\PYGZob{}pong,1347321821914426127719021955160323408745312813056\PYGZcb{}}
\end{Verbatim}

The response is the atom \emph{pong} and a huge number that we will explain later,
but to make it short, it's the id of the process that replied to us.


\section{Exploring the Template Files}
\label{starting:exploring-the-template-files}
The template created a lot of files and you are like me, you don't like things
that make magic and don't explain what's going on, that's why we will get a
brief overview of the files created here.

First this files are created:

\begin{Verbatim}[commandchars=\\\{\}]
\PYG{n}{apps}\PYG{o}{/}\PYG{n}{tanodb}\PYG{o}{/}\PYG{n}{src}\PYG{o}{/}\PYG{n}{tanodb}\PYG{o}{.}\PYG{n}{app}\PYG{o}{.}\PYG{n}{src}
\PYG{n}{apps}\PYG{o}{/}\PYG{n}{tanodb}\PYG{o}{/}\PYG{n}{src}\PYG{o}{/}\PYG{n}{tanodb}\PYG{o}{.}\PYG{n}{erl}
\PYG{n}{apps}\PYG{o}{/}\PYG{n}{tanodb}\PYG{o}{/}\PYG{n}{src}\PYG{o}{/}\PYG{n}{tanodb\PYGZus{}app}\PYG{o}{.}\PYG{n}{erl}
\PYG{n}{apps}\PYG{o}{/}\PYG{n}{tanodb}\PYG{o}{/}\PYG{n}{src}\PYG{o}{/}\PYG{n}{tanodb\PYGZus{}sup}\PYG{o}{.}\PYG{n}{erl}
\PYG{n}{apps}\PYG{o}{/}\PYG{n}{tanodb}\PYG{o}{/}\PYG{n}{src}\PYG{o}{/}\PYG{n}{tanodb\PYGZus{}console}\PYG{o}{.}\PYG{n}{erl}
\PYG{n}{apps}\PYG{o}{/}\PYG{n}{tanodb}\PYG{o}{/}\PYG{n}{src}\PYG{o}{/}\PYG{n}{tanodb\PYGZus{}vnode}\PYG{o}{.}\PYG{n}{erl}
\end{Verbatim}

Those are the meat of this project, the source code we start with, if you
know a little of erlang you will recognice them many of them, let's explain them briefly,
if you think you need more information I recommend you this awesome book which
you can read online: \href{http://learnyousomeerlang.com/}{Learn You Some Erlang for great good!}
\begin{description}
\item[{tanodb.app.src}] \leavevmode
This file is ``The Application Resource File'', you can read it, it's quite self descriptive.
You can read more about it in the
\href{http://learnyousomeerlang.com/building-otp-applications}{Building OTP Applications Section of Learn You Some Erlang}
or in the \href{http://www.erlang.org/doc/man/app.html}{man page for app in the erlang documentation}.

\item[{tanodb.erl}] \leavevmode
This file is the main API of our application, here we expose all the things
you can ask our application to do, for now it can only handle the \emph{ping()}
command but we will add some more in the future.

\item[{tanodb\_app.erl}] \leavevmode
This file implements the \href{http://www.erlang.org/doc/design\_principles/applications.html}{application behaviour} it's a set of callbacks
that the erlang runtime calls to start and stop our application.

\item[{tanodb\_sup.erl}] \leavevmode
This file implements the \href{http://www.erlang.org/doc/design\_principles/sup\_princ.html}{supervisor behaviour} it's a set of callbacks
that the erlang runtime calls to build the supervisor hierarchy.

\item[{tanodb\_console.erl}] \leavevmode
This file is specific to riak\_core, it's a set of callbacks that will be
called by the \emph{tanodb-admin} command.

\item[{tanodb\_vnode.erl}] \leavevmode
This file is specific to riak\_core, it implements the riak\_code\_vnode
behaviour, which is a set of callbacks that riak\_core will call to
acomplish different tasks, it's the main file we will edit to add new
features.

\end{description}

Those were the source code files, but the template also created other files,
let's review them
\begin{description}
\item[{rebar.config}] \leavevmode
This is the file that rebar3 reads to get information about our project
like dependencies and build configuration, you can read more about it
on the \href{http://www.rebar3.org/docs/basic-usage}{rebar3 documentation}

\item[{.editorconfig}] \leavevmode
This file describes the coding style for this project, if your text editor
understands editorconfig files then it will change it's setting for this
project to the ones described in this file, read more about editor config
on the \href{http://editorconfig.org/}{editorconfig website}

\item[{.gitignore}] \leavevmode
A file to tell git which files to ignore from the repository.

\item[{README.rst}] \leavevmode
The README of the project

\item[{Makefile}] \leavevmode
A make file with some targets that will make it easier to achieve some
complex tasks without copying and pasting to much.

\item[{config/nodetool}] \leavevmode
An \href{http://www.erlang.org/doc/man/escript.html}{escript} that makes it
easier to interact with an erlang node from the command line, it will be
used by the \emph{tanodb} and \emph{tanodb-admin} commands.

\item[{config/extended\_bin}] \leavevmode
A template for the \emph{tanodb} command with some changes to support \href{https://github.com/basho/cuttlefish}{cuttlefish}
which is the library we use to load and validate our configuration

\item[{config/admin\_bin}] \leavevmode
A template for the \emph{tanodb-admin} command.

\item[{config/config.schema}] \leavevmode
The \href{https://github.com/basho/cuttlefish/wiki}{cuttlefish schema} file
that describes what configuration our application supports, it starts with
some example configuration fields that we will
use as the application grows.

\item[{config/advanced.config}] \leavevmode
This file is where we configure some advanced things of our application
that don't go on our \emph{tanodb.config} file, here we configure riak\_core and
our \href{https://github.com/basho/lager/}{logging library}

\item[{config/sys.config}] \leavevmode
This is a standard erlang application file, you can read more about it
in the \href{http://www.erlang.org/doc/man/config.html}{erlang documentation for sys.config}

\item[{config/vars.config}] \leavevmode
This file contains variables used by relx to build a release, you can
read more about it in the \href{http://www.rebar3.org/docs/releases}{rebar3 release documentation}

\end{description}

The following files are like vars.config but with slight differences to allow
running more than one node on the same machine:

\begin{Verbatim}[commandchars=\\\{\}]
\PYG{n}{config}\PYG{o}{/}\PYG{n}{vars\PYGZus{}dev1}\PYG{o}{.}\PYG{n}{config}
\PYG{n}{config}\PYG{o}{/}\PYG{n}{vars\PYGZus{}dev2}\PYG{o}{.}\PYG{n}{config}
\PYG{n}{config}\PYG{o}{/}\PYG{n}{vars\PYGZus{}dev3}\PYG{o}{.}\PYG{n}{config}
\end{Verbatim}

Normally when you have a cluster for your application one operating system
instance runs one instance of your application and you have many operating
system instances, but to test the clustering features of riak\_core we will
build 3 releases of our application using offsets for ports and changing the
application name to avoid collitions.
\begin{description}
\item[{config/vm.args}] \leavevmode
A file used to pass options to the erlang VM when starting our application.

\end{description}

The following files are like vars\_dev*.config but for vm.args:

\begin{Verbatim}[commandchars=\\\{\}]
\PYG{n}{config}\PYG{o}{/}\PYG{n}{dev1\PYGZus{}vm}\PYG{o}{.}\PYG{n}{args}
\PYG{n}{config}\PYG{o}{/}\PYG{n}{dev2\PYGZus{}vm}\PYG{o}{.}\PYG{n}{args}
\PYG{n}{config}\PYG{o}{/}\PYG{n}{dev3\PYGZus{}vm}\PYG{o}{.}\PYG{n}{args}
\end{Verbatim}

Those are all the files, follow the links to know more about them.


\section{Playing with Clustering}
\label{starting:playing-with-clustering}
Before starting to add features, let's first play with clustering so we undertand
all those config files above work.

Build 3 releases that can run on the same machine:

\begin{Verbatim}[commandchars=\\\{\}]
make devrel
\end{Verbatim}

This will build 3 releases of the application using different parameters (the
dev1, dev2 and dev3 files we saw earlier) and will place them under:

\begin{Verbatim}[commandchars=\\\{\}]
\PYG{n}{\PYGZus{}build}\PYG{o}{/}\PYG{n}{dev1}
\PYG{n}{\PYGZus{}build}\PYG{o}{/}\PYG{n}{dev2}
\PYG{n}{\PYGZus{}build}\PYG{o}{/}\PYG{n}{dev3}
\end{Verbatim}

This is achived by using the \href{http://www.rebar3.org/docs/profiles}{profiles feature from rebar3}.

Now open 3 consoles and run the following commands one on each console:

\begin{Verbatim}[commandchars=\\\{\}]
make dev1\PYGZhy{}console
make dev2\PYGZhy{}console
make dev3\PYGZhy{}console
\end{Verbatim}

This will start the 3 nodes but the won't know about eachother, for them
to know about eachother we need to ``join'' them, that is to tell one of them
about the other two, this is achieved using the tanodb-admin command, here is
how you should run it manually (don't run them):

\begin{Verbatim}[commandchars=\\\{\}]
\PYGZus{}build/dev2/rel/tanodb/bin/tanodb\PYGZhy{}admin cluster join tanodb1@127.0.0.1
\PYGZus{}build/dev3/rel/tanodb/bin/tanodb\PYGZhy{}admin cluster join tanodb1@127.0.0.1
\end{Verbatim}

We tell dev2 and dev3 to join tanodb1 (dev1), to make this easier and less
error prone run the following command:

\begin{Verbatim}[commandchars=\\\{\}]
make devrel\PYGZhy{}join
\end{Verbatim}

Now let's check the status of the cluster:

\begin{Verbatim}[commandchars=\\\{\}]
make devrel\PYGZhy{}status
\end{Verbatim}

You can read the Makefile to get an idea of what those commands do, in this case
devrel-status does the following:

\begin{Verbatim}[commandchars=\\\{\}]
\PYGZus{}build/dev1/rel/tanodb/bin/tanodb\PYGZhy{}admin member\PYGZhy{}status
\end{Verbatim}

You should see something like this:

\begin{Verbatim}[commandchars=\\\{\}]
================================= Membership ==================================
Status     Ring    Pending    Node
\PYGZhy{}\PYGZhy{}\PYGZhy{}\PYGZhy{}\PYGZhy{}\PYGZhy{}\PYGZhy{}\PYGZhy{}\PYGZhy{}\PYGZhy{}\PYGZhy{}\PYGZhy{}\PYGZhy{}\PYGZhy{}\PYGZhy{}\PYGZhy{}\PYGZhy{}\PYGZhy{}\PYGZhy{}\PYGZhy{}\PYGZhy{}\PYGZhy{}\PYGZhy{}\PYGZhy{}\PYGZhy{}\PYGZhy{}\PYGZhy{}\PYGZhy{}\PYGZhy{}\PYGZhy{}\PYGZhy{}\PYGZhy{}\PYGZhy{}\PYGZhy{}\PYGZhy{}\PYGZhy{}\PYGZhy{}\PYGZhy{}\PYGZhy{}\PYGZhy{}\PYGZhy{}\PYGZhy{}\PYGZhy{}\PYGZhy{}\PYGZhy{}\PYGZhy{}\PYGZhy{}\PYGZhy{}\PYGZhy{}\PYGZhy{}\PYGZhy{}\PYGZhy{}\PYGZhy{}\PYGZhy{}\PYGZhy{}\PYGZhy{}\PYGZhy{}\PYGZhy{}\PYGZhy{}\PYGZhy{}\PYGZhy{}\PYGZhy{}\PYGZhy{}\PYGZhy{}\PYGZhy{}\PYGZhy{}\PYGZhy{}\PYGZhy{}\PYGZhy{}\PYGZhy{}\PYGZhy{}\PYGZhy{}\PYGZhy{}\PYGZhy{}\PYGZhy{}\PYGZhy{}\PYGZhy{}\PYGZhy{}\PYGZhy{}
joining     0.0\PYGZpc{}      \PYGZhy{}\PYGZhy{}      \PYGZsq{}tanodb2@127.0.0.1\PYGZsq{}
joining     0.0\PYGZpc{}      \PYGZhy{}\PYGZhy{}      \PYGZsq{}tanodb3@127.0.0.1\PYGZsq{}
valid     100.0\PYGZpc{}      \PYGZhy{}\PYGZhy{}      \PYGZsq{}tanodb1@127.0.0.1\PYGZsq{}
\PYGZhy{}\PYGZhy{}\PYGZhy{}\PYGZhy{}\PYGZhy{}\PYGZhy{}\PYGZhy{}\PYGZhy{}\PYGZhy{}\PYGZhy{}\PYGZhy{}\PYGZhy{}\PYGZhy{}\PYGZhy{}\PYGZhy{}\PYGZhy{}\PYGZhy{}\PYGZhy{}\PYGZhy{}\PYGZhy{}\PYGZhy{}\PYGZhy{}\PYGZhy{}\PYGZhy{}\PYGZhy{}\PYGZhy{}\PYGZhy{}\PYGZhy{}\PYGZhy{}\PYGZhy{}\PYGZhy{}\PYGZhy{}\PYGZhy{}\PYGZhy{}\PYGZhy{}\PYGZhy{}\PYGZhy{}\PYGZhy{}\PYGZhy{}\PYGZhy{}\PYGZhy{}\PYGZhy{}\PYGZhy{}\PYGZhy{}\PYGZhy{}\PYGZhy{}\PYGZhy{}\PYGZhy{}\PYGZhy{}\PYGZhy{}\PYGZhy{}\PYGZhy{}\PYGZhy{}\PYGZhy{}\PYGZhy{}\PYGZhy{}\PYGZhy{}\PYGZhy{}\PYGZhy{}\PYGZhy{}\PYGZhy{}\PYGZhy{}\PYGZhy{}\PYGZhy{}\PYGZhy{}\PYGZhy{}\PYGZhy{}\PYGZhy{}\PYGZhy{}\PYGZhy{}\PYGZhy{}\PYGZhy{}\PYGZhy{}\PYGZhy{}\PYGZhy{}\PYGZhy{}\PYGZhy{}\PYGZhy{}\PYGZhy{}
Valid:1 / Leaving:0 / Exiting:0 / Joining:2 / Down:0
\end{Verbatim}

It should say that 3 nodes are joining, now check the cluster plan:

\begin{Verbatim}[commandchars=\\\{\}]
make devrel\PYGZhy{}cluster\PYGZhy{}plan
\end{Verbatim}

The output should be something like this:

\begin{Verbatim}[commandchars=\\\{\}]
=============================== Staged Changes ================================
Action         Details(s)
\PYGZhy{}\PYGZhy{}\PYGZhy{}\PYGZhy{}\PYGZhy{}\PYGZhy{}\PYGZhy{}\PYGZhy{}\PYGZhy{}\PYGZhy{}\PYGZhy{}\PYGZhy{}\PYGZhy{}\PYGZhy{}\PYGZhy{}\PYGZhy{}\PYGZhy{}\PYGZhy{}\PYGZhy{}\PYGZhy{}\PYGZhy{}\PYGZhy{}\PYGZhy{}\PYGZhy{}\PYGZhy{}\PYGZhy{}\PYGZhy{}\PYGZhy{}\PYGZhy{}\PYGZhy{}\PYGZhy{}\PYGZhy{}\PYGZhy{}\PYGZhy{}\PYGZhy{}\PYGZhy{}\PYGZhy{}\PYGZhy{}\PYGZhy{}\PYGZhy{}\PYGZhy{}\PYGZhy{}\PYGZhy{}\PYGZhy{}\PYGZhy{}\PYGZhy{}\PYGZhy{}\PYGZhy{}\PYGZhy{}\PYGZhy{}\PYGZhy{}\PYGZhy{}\PYGZhy{}\PYGZhy{}\PYGZhy{}\PYGZhy{}\PYGZhy{}\PYGZhy{}\PYGZhy{}\PYGZhy{}\PYGZhy{}\PYGZhy{}\PYGZhy{}\PYGZhy{}\PYGZhy{}\PYGZhy{}\PYGZhy{}\PYGZhy{}\PYGZhy{}\PYGZhy{}\PYGZhy{}\PYGZhy{}\PYGZhy{}\PYGZhy{}\PYGZhy{}\PYGZhy{}\PYGZhy{}\PYGZhy{}\PYGZhy{}
join           \PYGZsq{}tanodb2@127.0.0.1\PYGZsq{}
join           \PYGZsq{}tanodb3@127.0.0.1\PYGZsq{}
\PYGZhy{}\PYGZhy{}\PYGZhy{}\PYGZhy{}\PYGZhy{}\PYGZhy{}\PYGZhy{}\PYGZhy{}\PYGZhy{}\PYGZhy{}\PYGZhy{}\PYGZhy{}\PYGZhy{}\PYGZhy{}\PYGZhy{}\PYGZhy{}\PYGZhy{}\PYGZhy{}\PYGZhy{}\PYGZhy{}\PYGZhy{}\PYGZhy{}\PYGZhy{}\PYGZhy{}\PYGZhy{}\PYGZhy{}\PYGZhy{}\PYGZhy{}\PYGZhy{}\PYGZhy{}\PYGZhy{}\PYGZhy{}\PYGZhy{}\PYGZhy{}\PYGZhy{}\PYGZhy{}\PYGZhy{}\PYGZhy{}\PYGZhy{}\PYGZhy{}\PYGZhy{}\PYGZhy{}\PYGZhy{}\PYGZhy{}\PYGZhy{}\PYGZhy{}\PYGZhy{}\PYGZhy{}\PYGZhy{}\PYGZhy{}\PYGZhy{}\PYGZhy{}\PYGZhy{}\PYGZhy{}\PYGZhy{}\PYGZhy{}\PYGZhy{}\PYGZhy{}\PYGZhy{}\PYGZhy{}\PYGZhy{}\PYGZhy{}\PYGZhy{}\PYGZhy{}\PYGZhy{}\PYGZhy{}\PYGZhy{}\PYGZhy{}\PYGZhy{}\PYGZhy{}\PYGZhy{}\PYGZhy{}\PYGZhy{}\PYGZhy{}\PYGZhy{}\PYGZhy{}\PYGZhy{}\PYGZhy{}\PYGZhy{}


NOTE: Applying these changes will result in 1 cluster transition

\PYGZsh{}\PYGZsh{}\PYGZsh{}\PYGZsh{}\PYGZsh{}\PYGZsh{}\PYGZsh{}\PYGZsh{}\PYGZsh{}\PYGZsh{}\PYGZsh{}\PYGZsh{}\PYGZsh{}\PYGZsh{}\PYGZsh{}\PYGZsh{}\PYGZsh{}\PYGZsh{}\PYGZsh{}\PYGZsh{}\PYGZsh{}\PYGZsh{}\PYGZsh{}\PYGZsh{}\PYGZsh{}\PYGZsh{}\PYGZsh{}\PYGZsh{}\PYGZsh{}\PYGZsh{}\PYGZsh{}\PYGZsh{}\PYGZsh{}\PYGZsh{}\PYGZsh{}\PYGZsh{}\PYGZsh{}\PYGZsh{}\PYGZsh{}\PYGZsh{}\PYGZsh{}\PYGZsh{}\PYGZsh{}\PYGZsh{}\PYGZsh{}\PYGZsh{}\PYGZsh{}\PYGZsh{}\PYGZsh{}\PYGZsh{}\PYGZsh{}\PYGZsh{}\PYGZsh{}\PYGZsh{}\PYGZsh{}\PYGZsh{}\PYGZsh{}\PYGZsh{}\PYGZsh{}\PYGZsh{}\PYGZsh{}\PYGZsh{}\PYGZsh{}\PYGZsh{}\PYGZsh{}\PYGZsh{}\PYGZsh{}\PYGZsh{}\PYGZsh{}\PYGZsh{}\PYGZsh{}\PYGZsh{}\PYGZsh{}\PYGZsh{}\PYGZsh{}\PYGZsh{}\PYGZsh{}\PYGZsh{}\PYGZsh{}
                         After cluster transition 1/1
\PYGZsh{}\PYGZsh{}\PYGZsh{}\PYGZsh{}\PYGZsh{}\PYGZsh{}\PYGZsh{}\PYGZsh{}\PYGZsh{}\PYGZsh{}\PYGZsh{}\PYGZsh{}\PYGZsh{}\PYGZsh{}\PYGZsh{}\PYGZsh{}\PYGZsh{}\PYGZsh{}\PYGZsh{}\PYGZsh{}\PYGZsh{}\PYGZsh{}\PYGZsh{}\PYGZsh{}\PYGZsh{}\PYGZsh{}\PYGZsh{}\PYGZsh{}\PYGZsh{}\PYGZsh{}\PYGZsh{}\PYGZsh{}\PYGZsh{}\PYGZsh{}\PYGZsh{}\PYGZsh{}\PYGZsh{}\PYGZsh{}\PYGZsh{}\PYGZsh{}\PYGZsh{}\PYGZsh{}\PYGZsh{}\PYGZsh{}\PYGZsh{}\PYGZsh{}\PYGZsh{}\PYGZsh{}\PYGZsh{}\PYGZsh{}\PYGZsh{}\PYGZsh{}\PYGZsh{}\PYGZsh{}\PYGZsh{}\PYGZsh{}\PYGZsh{}\PYGZsh{}\PYGZsh{}\PYGZsh{}\PYGZsh{}\PYGZsh{}\PYGZsh{}\PYGZsh{}\PYGZsh{}\PYGZsh{}\PYGZsh{}\PYGZsh{}\PYGZsh{}\PYGZsh{}\PYGZsh{}\PYGZsh{}\PYGZsh{}\PYGZsh{}\PYGZsh{}\PYGZsh{}\PYGZsh{}\PYGZsh{}\PYGZsh{}

================================= Membership ==================================
Status     Ring    Pending    Node
\PYGZhy{}\PYGZhy{}\PYGZhy{}\PYGZhy{}\PYGZhy{}\PYGZhy{}\PYGZhy{}\PYGZhy{}\PYGZhy{}\PYGZhy{}\PYGZhy{}\PYGZhy{}\PYGZhy{}\PYGZhy{}\PYGZhy{}\PYGZhy{}\PYGZhy{}\PYGZhy{}\PYGZhy{}\PYGZhy{}\PYGZhy{}\PYGZhy{}\PYGZhy{}\PYGZhy{}\PYGZhy{}\PYGZhy{}\PYGZhy{}\PYGZhy{}\PYGZhy{}\PYGZhy{}\PYGZhy{}\PYGZhy{}\PYGZhy{}\PYGZhy{}\PYGZhy{}\PYGZhy{}\PYGZhy{}\PYGZhy{}\PYGZhy{}\PYGZhy{}\PYGZhy{}\PYGZhy{}\PYGZhy{}\PYGZhy{}\PYGZhy{}\PYGZhy{}\PYGZhy{}\PYGZhy{}\PYGZhy{}\PYGZhy{}\PYGZhy{}\PYGZhy{}\PYGZhy{}\PYGZhy{}\PYGZhy{}\PYGZhy{}\PYGZhy{}\PYGZhy{}\PYGZhy{}\PYGZhy{}\PYGZhy{}\PYGZhy{}\PYGZhy{}\PYGZhy{}\PYGZhy{}\PYGZhy{}\PYGZhy{}\PYGZhy{}\PYGZhy{}\PYGZhy{}\PYGZhy{}\PYGZhy{}\PYGZhy{}\PYGZhy{}\PYGZhy{}\PYGZhy{}\PYGZhy{}\PYGZhy{}\PYGZhy{}
valid     100.0\PYGZpc{}     34.4\PYGZpc{}    \PYGZsq{}tanodb1@127.0.0.1\PYGZsq{}
valid       0.0\PYGZpc{}     32.8\PYGZpc{}    \PYGZsq{}tanodb2@127.0.0.1\PYGZsq{}
valid       0.0\PYGZpc{}     32.8\PYGZpc{}    \PYGZsq{}tanodb3@127.0.0.1\PYGZsq{}
\PYGZhy{}\PYGZhy{}\PYGZhy{}\PYGZhy{}\PYGZhy{}\PYGZhy{}\PYGZhy{}\PYGZhy{}\PYGZhy{}\PYGZhy{}\PYGZhy{}\PYGZhy{}\PYGZhy{}\PYGZhy{}\PYGZhy{}\PYGZhy{}\PYGZhy{}\PYGZhy{}\PYGZhy{}\PYGZhy{}\PYGZhy{}\PYGZhy{}\PYGZhy{}\PYGZhy{}\PYGZhy{}\PYGZhy{}\PYGZhy{}\PYGZhy{}\PYGZhy{}\PYGZhy{}\PYGZhy{}\PYGZhy{}\PYGZhy{}\PYGZhy{}\PYGZhy{}\PYGZhy{}\PYGZhy{}\PYGZhy{}\PYGZhy{}\PYGZhy{}\PYGZhy{}\PYGZhy{}\PYGZhy{}\PYGZhy{}\PYGZhy{}\PYGZhy{}\PYGZhy{}\PYGZhy{}\PYGZhy{}\PYGZhy{}\PYGZhy{}\PYGZhy{}\PYGZhy{}\PYGZhy{}\PYGZhy{}\PYGZhy{}\PYGZhy{}\PYGZhy{}\PYGZhy{}\PYGZhy{}\PYGZhy{}\PYGZhy{}\PYGZhy{}\PYGZhy{}\PYGZhy{}\PYGZhy{}\PYGZhy{}\PYGZhy{}\PYGZhy{}\PYGZhy{}\PYGZhy{}\PYGZhy{}\PYGZhy{}\PYGZhy{}\PYGZhy{}\PYGZhy{}\PYGZhy{}\PYGZhy{}\PYGZhy{}
Valid:3 / Leaving:0 / Exiting:0 / Joining:0 / Down:0

WARNING: Not all replicas will be on distinct nodes

Transfers resulting from cluster changes: 42
  21 transfers from \PYGZsq{}tanodb1@127.0.0.1\PYGZsq{} to \PYGZsq{}tanodb3@127.0.0.1\PYGZsq{}
  21 transfers from \PYGZsq{}tanodb1@127.0.0.1\PYGZsq{} to \PYGZsq{}tanodb2@127.0.0.1\PYGZsq{}
\end{Verbatim}

Now we can commit the plan:

\begin{Verbatim}[commandchars=\\\{\}]
make devrel\PYGZhy{}cluster\PYGZhy{}commit
\end{Verbatim}

Which should say something like:

\begin{Verbatim}[commandchars=\\\{\}]
Cluster changes committed
\end{Verbatim}

Now riak\_core started an internal process to join the nodes to the cluster,
this involve some complex processes that we will explore in the following
chapters.

You should see on the consoles where the nodes are running that some logging
is happening describing the process.

Check the status of the cluster again:

\begin{Verbatim}[commandchars=\\\{\}]
make devrel\PYGZhy{}status
\end{Verbatim}

You can see the vnodes transfering, this means the content of some virtual
nodes on one tanodb node are being transfered to another tanodb node:

\begin{Verbatim}[commandchars=\\\{\}]
================================= Membership ==================================
Status     Ring    Pending    Node
\PYGZhy{}\PYGZhy{}\PYGZhy{}\PYGZhy{}\PYGZhy{}\PYGZhy{}\PYGZhy{}\PYGZhy{}\PYGZhy{}\PYGZhy{}\PYGZhy{}\PYGZhy{}\PYGZhy{}\PYGZhy{}\PYGZhy{}\PYGZhy{}\PYGZhy{}\PYGZhy{}\PYGZhy{}\PYGZhy{}\PYGZhy{}\PYGZhy{}\PYGZhy{}\PYGZhy{}\PYGZhy{}\PYGZhy{}\PYGZhy{}\PYGZhy{}\PYGZhy{}\PYGZhy{}\PYGZhy{}\PYGZhy{}\PYGZhy{}\PYGZhy{}\PYGZhy{}\PYGZhy{}\PYGZhy{}\PYGZhy{}\PYGZhy{}\PYGZhy{}\PYGZhy{}\PYGZhy{}\PYGZhy{}\PYGZhy{}\PYGZhy{}\PYGZhy{}\PYGZhy{}\PYGZhy{}\PYGZhy{}\PYGZhy{}\PYGZhy{}\PYGZhy{}\PYGZhy{}\PYGZhy{}\PYGZhy{}\PYGZhy{}\PYGZhy{}\PYGZhy{}\PYGZhy{}\PYGZhy{}\PYGZhy{}\PYGZhy{}\PYGZhy{}\PYGZhy{}\PYGZhy{}\PYGZhy{}\PYGZhy{}\PYGZhy{}\PYGZhy{}\PYGZhy{}\PYGZhy{}\PYGZhy{}\PYGZhy{}\PYGZhy{}\PYGZhy{}\PYGZhy{}\PYGZhy{}\PYGZhy{}\PYGZhy{}
valid      75.0\PYGZpc{}     34.4\PYGZpc{}    \PYGZsq{}tanodb1@127.0.0.1\PYGZsq{}
valid       9.4\PYGZpc{}     32.8\PYGZpc{}    \PYGZsq{}tanodb2@127.0.0.1\PYGZsq{}
valid       7.8\PYGZpc{}     32.8\PYGZpc{}    \PYGZsq{}tanodb3@127.0.0.1\PYGZsq{}
\PYGZhy{}\PYGZhy{}\PYGZhy{}\PYGZhy{}\PYGZhy{}\PYGZhy{}\PYGZhy{}\PYGZhy{}\PYGZhy{}\PYGZhy{}\PYGZhy{}\PYGZhy{}\PYGZhy{}\PYGZhy{}\PYGZhy{}\PYGZhy{}\PYGZhy{}\PYGZhy{}\PYGZhy{}\PYGZhy{}\PYGZhy{}\PYGZhy{}\PYGZhy{}\PYGZhy{}\PYGZhy{}\PYGZhy{}\PYGZhy{}\PYGZhy{}\PYGZhy{}\PYGZhy{}\PYGZhy{}\PYGZhy{}\PYGZhy{}\PYGZhy{}\PYGZhy{}\PYGZhy{}\PYGZhy{}\PYGZhy{}\PYGZhy{}\PYGZhy{}\PYGZhy{}\PYGZhy{}\PYGZhy{}\PYGZhy{}\PYGZhy{}\PYGZhy{}\PYGZhy{}\PYGZhy{}\PYGZhy{}\PYGZhy{}\PYGZhy{}\PYGZhy{}\PYGZhy{}\PYGZhy{}\PYGZhy{}\PYGZhy{}\PYGZhy{}\PYGZhy{}\PYGZhy{}\PYGZhy{}\PYGZhy{}\PYGZhy{}\PYGZhy{}\PYGZhy{}\PYGZhy{}\PYGZhy{}\PYGZhy{}\PYGZhy{}\PYGZhy{}\PYGZhy{}\PYGZhy{}\PYGZhy{}\PYGZhy{}\PYGZhy{}\PYGZhy{}\PYGZhy{}\PYGZhy{}\PYGZhy{}\PYGZhy{}
Valid:3 / Leaving:0 / Exiting:0 / Joining:0 / Down:0
\end{Verbatim}

At some point you should see something like this, which means that the nodes
are joined and balanced:

\begin{Verbatim}[commandchars=\\\{\}]
================================= Membership ==================================
Status     Ring    Pending    Node
\PYGZhy{}\PYGZhy{}\PYGZhy{}\PYGZhy{}\PYGZhy{}\PYGZhy{}\PYGZhy{}\PYGZhy{}\PYGZhy{}\PYGZhy{}\PYGZhy{}\PYGZhy{}\PYGZhy{}\PYGZhy{}\PYGZhy{}\PYGZhy{}\PYGZhy{}\PYGZhy{}\PYGZhy{}\PYGZhy{}\PYGZhy{}\PYGZhy{}\PYGZhy{}\PYGZhy{}\PYGZhy{}\PYGZhy{}\PYGZhy{}\PYGZhy{}\PYGZhy{}\PYGZhy{}\PYGZhy{}\PYGZhy{}\PYGZhy{}\PYGZhy{}\PYGZhy{}\PYGZhy{}\PYGZhy{}\PYGZhy{}\PYGZhy{}\PYGZhy{}\PYGZhy{}\PYGZhy{}\PYGZhy{}\PYGZhy{}\PYGZhy{}\PYGZhy{}\PYGZhy{}\PYGZhy{}\PYGZhy{}\PYGZhy{}\PYGZhy{}\PYGZhy{}\PYGZhy{}\PYGZhy{}\PYGZhy{}\PYGZhy{}\PYGZhy{}\PYGZhy{}\PYGZhy{}\PYGZhy{}\PYGZhy{}\PYGZhy{}\PYGZhy{}\PYGZhy{}\PYGZhy{}\PYGZhy{}\PYGZhy{}\PYGZhy{}\PYGZhy{}\PYGZhy{}\PYGZhy{}\PYGZhy{}\PYGZhy{}\PYGZhy{}\PYGZhy{}\PYGZhy{}\PYGZhy{}\PYGZhy{}\PYGZhy{}
valid      34.4\PYGZpc{}      \PYGZhy{}\PYGZhy{}      \PYGZsq{}tanodb1@127.0.0.1\PYGZsq{}
valid      32.8\PYGZpc{}      \PYGZhy{}\PYGZhy{}      \PYGZsq{}tanodb2@127.0.0.1\PYGZsq{}
valid      32.8\PYGZpc{}      \PYGZhy{}\PYGZhy{}      \PYGZsq{}tanodb3@127.0.0.1\PYGZsq{}
\PYGZhy{}\PYGZhy{}\PYGZhy{}\PYGZhy{}\PYGZhy{}\PYGZhy{}\PYGZhy{}\PYGZhy{}\PYGZhy{}\PYGZhy{}\PYGZhy{}\PYGZhy{}\PYGZhy{}\PYGZhy{}\PYGZhy{}\PYGZhy{}\PYGZhy{}\PYGZhy{}\PYGZhy{}\PYGZhy{}\PYGZhy{}\PYGZhy{}\PYGZhy{}\PYGZhy{}\PYGZhy{}\PYGZhy{}\PYGZhy{}\PYGZhy{}\PYGZhy{}\PYGZhy{}\PYGZhy{}\PYGZhy{}\PYGZhy{}\PYGZhy{}\PYGZhy{}\PYGZhy{}\PYGZhy{}\PYGZhy{}\PYGZhy{}\PYGZhy{}\PYGZhy{}\PYGZhy{}\PYGZhy{}\PYGZhy{}\PYGZhy{}\PYGZhy{}\PYGZhy{}\PYGZhy{}\PYGZhy{}\PYGZhy{}\PYGZhy{}\PYGZhy{}\PYGZhy{}\PYGZhy{}\PYGZhy{}\PYGZhy{}\PYGZhy{}\PYGZhy{}\PYGZhy{}\PYGZhy{}\PYGZhy{}\PYGZhy{}\PYGZhy{}\PYGZhy{}\PYGZhy{}\PYGZhy{}\PYGZhy{}\PYGZhy{}\PYGZhy{}\PYGZhy{}\PYGZhy{}\PYGZhy{}\PYGZhy{}\PYGZhy{}\PYGZhy{}\PYGZhy{}\PYGZhy{}\PYGZhy{}\PYGZhy{}
Valid:3 / Leaving:0 / Exiting:0 / Joining:0 / Down:0
\end{Verbatim}

When you are bored you can stop them:

\begin{Verbatim}[commandchars=\\\{\}]
make devrel\PYGZhy{}stop
\end{Verbatim}


\section{Building a Production Release}
\label{starting:building-a-production-release}
Even when our application doesn't have the features to merit a production
release we are going to learn how to do it here since you can later do it at
any step and get a full release of the app:

\begin{Verbatim}[commandchars=\\\{\}]
rebar3 as prod release
\end{Verbatim}

In that command we as rebar3 to run the release task using the prod profile,
which has some configuration differences with the dev profiles we use so that
it builds something we can unpack and run on another operating system without
installing anything.

In my case I'm developing this on ubuntu, to show you that it works I will
copy the release to a clean ubuntu 15.04 Virtualbox and run it there:

\begin{Verbatim}[commandchars=\\\{\}]
mkdir vm\PYGZhy{}ubuntu\PYGZhy{}1504
cd vm\PYGZhy{}ubuntu\PYGZhy{}1504
\end{Verbatim}

Inside I will create a file called \emph{Vagrantfile} with the following
content:

\begin{Verbatim}[commandchars=\\\{\}]
Vagrant.configure(2) do \textbar{}config\textbar{}
  config.vm.box = \PYGZdq{}ubuntu/vivid64\PYGZdq{}
  config.vm.provider \PYGZdq{}virtualbox\PYGZdq{} do \textbar{}vb\textbar{}
    vb.memory = \PYGZdq{}1024\PYGZdq{}
  end
end
\end{Verbatim}

And then run:

\begin{Verbatim}[commandchars=\\\{\}]
vagrant up
\end{Verbatim}

To start the virutal machine.

Now let's package our release and copy it to a place where the VM can see it:

\begin{Verbatim}[commandchars=\\\{\}]
cd \PYGZus{}build/prod/rel
tar \PYGZhy{}czf tanodb.tgz tanodb
cd \PYGZhy{}
mv \PYGZus{}build/prod/rel/tanodb.tgz vm\PYGZhy{}ubuntu\PYGZhy{}1504
\end{Verbatim}

Let's ssh into the virtual machine:

\begin{Verbatim}[commandchars=\\\{\}]
export TERM=xterm
vagrant ssh
\end{Verbatim}

Inside the virtual machine run:

\begin{Verbatim}[commandchars=\\\{\}]
cp /vagrant/tanodb.tgz .
tar \PYGZhy{}xzf tanodb.tgz
./tanodb/bin/tanodb console
\end{Verbatim}

And it runs!

\begin{notice}{note}{Note:}
You should build the production release on the same operating system
version you are intending to run it to avoid version problems, the
main source of headaches are C extensions disagreeing on libc versions
and similar.

So, even when you could build it on a version that is close and test
it it's better to build releases on the same version to avoid
problems. More so if you are packaging the Erlang runtime with the
release as we are doing here.
\end{notice}


\section{Wrapping Up}
\label{starting:wrapping-up}
Now you know how to create a riak\_core app from a template, how to build a
release and run it, how to build releases for a development cluster, run
the nodes, join them and inspect the cluster status and how to build a
production release and run it on a fresh server.

Quite a lot for the first chapter I would say...


\chapter{Ping as a Service (PaaS)}
\label{ping-as-a-service::doc}\label{ping-as-a-service:ping-as-a-service-paas}
\begin{notice}{note}{Note:}
This chapter and the following ones will reference a real project in a
github repository, follow the links to see the details of what is written
here.
\end{notice}


\section{Setting Up}
\label{ping-as-a-service:setting-up}
After setting up our project we will now expose ping as a REST API, for that
we will use \href{http://ninenines.eu/docs/en/cowboy/1.0/}{The Cowboy Web Server}.

For JSON parsing we will use \href{https://github.com/talentdeficit/jsx}{jsx}.

Here is the \href{https://github.com/marianoguerra/tanodb/commit/b86718c1b8e8689ca8adb15627f59ce44c486bfc}{commit to add the dependencies}.

We also add the \href{http://www.rebar3.org/docs/dependencies\#dependency-lock-management}{rebar.lock}
file to make our builds reproducible.

Just in case we want to use another json library later and to simplify the calls
we \href{https://github.com/marianoguerra/tanodb/commit/fdccd5e2863c8c71599bcd38a26e8b8b5fcd5219}{wrap the json library in our own module}.

Finally we create a \href{http://ninenines.eu/docs/en/cowboy/1.0/manual/cowboy\_rest/}{cowboy rest handler} for our ping resource, \href{https://github.com/marianoguerra/tanodb/blob/220bcade820538aec05993065ac4edf19f3ebcde/apps/tanodb/src/tanodb\_http\_ping.erl}{tanodb\_http\_ping.erl} and \href{https://github.com/marianoguerra/tanodb/commit/220bcade820538aec05993065ac4edf19f3ebcde}{initialize cowboy in tanodb\_app}.


\section{Testing it}
\label{ping-as-a-service:testing-it}
To interact with the REST API we will use httpie since it's simpler to read
(and write) than curl, check how to install it on
\href{http://httpie.org}{the httpie website}

First we build it and run it as usual (this may be the last time I show you explicitly how to do it, so learn it :):

\begin{Verbatim}[commandchars=\\\{\}]
rebar3 release
rebar3 run
\end{Verbatim}

Now on another shell we will make an HTTP request to our ping resource:

\begin{Verbatim}[commandchars=\\\{\}]
http localhost:8080/ping
\end{Verbatim}

And this is what I get:

\begin{Verbatim}[commandchars=\\\{\}]
\PYG{k+kr}{HTTP}\PYG{o}{/}\PYG{l+m}{1.1} \PYG{l+m}{200} \PYG{n+ne}{OK}
\PYG{n+na}{content\PYGZhy{}length}\PYG{o}{:} \PYG{l}{59}
\PYG{n+na}{content\PYGZhy{}type}\PYG{o}{:} \PYG{l}{application/json}
\PYG{n+na}{date}\PYG{o}{:} \PYG{l}{Thu, 29 Oct 2015 19:07:23 GMT}
\PYG{n+na}{server}\PYG{o}{:} \PYG{l}{Cowboy}

\PYG{p}{\PYGZob{}}
\PYG{n+nt}{\PYGZdq{}pong\PYGZdq{}}\PYG{p}{:} \PYG{l+s+s2}{\PYGZdq{}981946412581700398168100746981252653831329677312\PYGZdq{}}
\PYG{p}{\PYGZcb{}}
\end{Verbatim}

If you run it more times the value of the \emph{pong} attribute should change, since
the vnode that handles the request \href{https://github.com/marianoguerra/tanodb/blob/220bcade820538aec05993065ac4edf19f3ebcde/apps/tanodb/src/tanodb.erl\#L16}{is defined by the time that the request is
made}.


\section{Changing Some Configuration}
\label{ping-as-a-service:changing-some-configuration}
Let's say we would like to run the server on another port, for that we need
to change the configuration, we can do this by editing the file:

\begin{Verbatim}[commandchars=\\\{\}]
\PYG{n}{\PYGZus{}build}\PYG{o}{/}\PYG{n}{default}\PYG{o}{/}\PYG{n}{rel}\PYG{o}{/}\PYG{n}{tanodb}\PYG{o}{/}\PYG{n}{etc}\PYG{o}{/}\PYG{n}{tanodb}\PYG{o}{.}\PYG{n}{conf}
\end{Verbatim}

Search for 8080 and change it for 8081, save and close and stop the server if you are running it.

Now we will run it again buit manually to avoid rebar3 from overriding our
change:

\begin{Verbatim}[commandchars=\\\{\}]
./\PYGZus{}build/default/rel/tanodb/bin/tanodb console
\end{Verbatim}

And try a request to see if the port is actually changed:

\begin{Verbatim}[commandchars=\\\{\}]
http localhost:8081/ping
\end{Verbatim}

And this is what I get:

\begin{Verbatim}[commandchars=\\\{\}]
\PYG{k+kr}{HTTP}\PYG{o}{/}\PYG{l+m}{1.1} \PYG{l+m}{200} \PYG{n+ne}{OK}
\PYG{n+na}{content\PYGZhy{}length}\PYG{o}{:} \PYG{l}{60}
\PYG{n+na}{content\PYGZhy{}type}\PYG{o}{:} \PYG{l}{application/json}
\PYG{n+na}{date}\PYG{o}{:} \PYG{l}{Thu, 29 Oct 2015 19:18:03 GMT}
\PYG{n+na}{server}\PYG{o}{:} \PYG{l}{Cowboy}

\PYG{p}{\PYGZob{}}
    \PYG{n+nt}{\PYGZdq{}pong\PYGZdq{}}\PYG{p}{:} \PYG{l+s+s2}{\PYGZdq{}1187470080331358621040493926581979953470445191168\PYGZdq{}}
\PYG{p}{\PYGZcb{}}
\end{Verbatim}

Read tanodb.config to see all the available options, this file is generated
using \href{https://github.com/basho/cuttlefish}{cuttlefish} which takes a
\href{https://github.com/marianoguerra/tanodb/blob/220bcade820538aec05993065ac4edf19f3ebcde/config/config.schema}{schema we define} and uses it to generate
the default config file and later to validate the config file on startup
and generate configuration files that the erlang runtime understands.

If you are curious you can see the generated config files after running the
server at least once under \emph{\_build/default/rel/tanodb/generated.configs/}


\chapter{Metrics}
\label{metrics:metrics}\label{metrics::doc}

\section{API Metrics}
\label{metrics:api-metrics}
Since this is meant to be a production system we can't be far along until we
add metrics, for this we will use \href{https://github.com/Feuerlabs/exometer}{exometer} which is already a dependency of riak\_core so we don't need to add it.

We start by defining a \href{https://github.com/marianoguerra/tanodb/blob/0ea3595aefce0f9098cb651eb33263933ce9d6e7/apps/tanodb/src/tanodb\_metrics.erl}{module named tanodb\_metrics}.

The main functions we care about are :
\begin{description}
\item[{\href{https://github.com/marianoguerra/tanodb/blob/0ea3595aefce0f9098cb651eb33263933ce9d6e7/apps/tanodb/src/tanodb\_metrics.erl\#L16}{init/0}}] \leavevmode
which will initialize all the metrics when the app starts, we will add more metrics here as we add more features.

\item[{\href{https://github.com/marianoguerra/tanodb/blob/0ea3595aefce0f9098cb651eb33263933ce9d6e7/apps/tanodb/src/tanodb\_metrics.erl\#L14}{core\_ping/0}}] \leavevmode
should be called to register metrics about calls to \href{https://github.com/marianoguerra/tanodb/blob/0ea3595aefce0f9098cb651eb33263933ce9d6e7/apps/tanodb/src/tanodb.erl\#L15}{tanodb:ping/0}

\item[{\href{https://github.com/marianoguerra/tanodb/commit/0ea3595aefce0f9098cb651eb33263933ce9d6e7\#diff-afa3f67ec87f742d64ee9ed311455777R8}{all/0}}] \leavevmode
returns the current status of all metrics.

\end{description}

To make the metrics actually work we need to call \href{https://github.com/marianoguerra/tanodb/commit/0ea3595aefce0f9098cb651eb33263933ce9d6e7\#diff-4477d4dd0aa2db0e274a56c9158207bdR13}{tanodb\_metrics:init/0} when we start the application and \href{https://github.com/marianoguerra/tanodb/commit/0ea3595aefce0f9098cb651eb33263933ce9d6e7\#diff-6f7251bf9e224ebabd766f0331b848adR16}{tanodb\_metrics:core\_ping/0} each time tanodb:ping/0 is called.


\subsection{Test It}
\label{metrics:test-it}
Stop, build a release and run the server (I won't tell you how from now on, check previous chapters to see how).

On the server shell run:

\begin{Verbatim}[commandchars=\\\{\}]
\PYG{g+go}{(tanodb@127.0.0.1)1\PYGZgt{} tanodb\PYGZus{}metrics:all().}
\PYG{g+go}{[\PYGZob{}tanodb,[}

\PYG{g+go}{ ...}

\PYG{g+go}{ \PYGZob{}core,[\PYGZob{}ping,[\PYGZob{}count,0\PYGZcb{},\PYGZob{}one,0\PYGZcb{}]\PYGZcb{}]\PYGZcb{}]}

\PYG{g+go}{(tanodb@127.0.0.1)2\PYGZgt{} tanodb:ping().}
\PYG{g+go}{\PYGZob{}pong,593735040165679310520246963290989976735222595584\PYGZcb{}}

\PYG{g+go}{(tanodb@127.0.0.1)3\PYGZgt{} tanodb\PYGZus{}metrics:all().}
\PYG{g+go}{[\PYGZob{}tanodb,[}

\PYG{g+go}{ ...}

\PYG{g+go}{ \PYGZob{}core,[\PYGZob{}ping,[\PYGZob{}count,1\PYGZcb{},\PYGZob{}one,1\PYGZcb{}]\PYGZcb{}]\PYGZcb{}]}

\PYG{g+go}{(tanodb@127.0.0.1)4\PYGZgt{}}
\end{Verbatim}

The \emph{...} are there to skip a lot of metrics about riak\_core itself that
are quite useful but not important at this point.

Let's see the shell session step by step, first we call tanodb\_metrics:all()
and get the core ping metrics, in this case count and one are 0 since we
didn't call ping yet.

\begin{Verbatim}[commandchars=\\\{\}]
\PYG{g+go}{(tanodb@127.0.0.1)1\PYGZgt{} tanodb\PYGZus{}metrics:all().}
\PYG{g+go}{[\PYGZob{}tanodb,[}

\PYG{g+go}{ ...}

\PYG{g+go}{ \PYGZob{}core,[\PYGZob{}ping,[\PYGZob{}count,0\PYGZcb{},\PYGZob{}one,0\PYGZcb{}]\PYGZcb{}]\PYGZcb{}]}
\end{Verbatim}

Then we call ping once.

\begin{Verbatim}[commandchars=\\\{\}]
\PYG{g+go}{(tanodb@127.0.0.1)2\PYGZgt{} tanodb:ping().}
\PYG{g+go}{\PYGZob{}pong,593735040165679310520246963290989976735222595584\PYGZcb{}}
\end{Verbatim}

And ask for the metrics again, we can see that now it registered our call.

\begin{Verbatim}[commandchars=\\\{\}]
\PYG{g+go}{(tanodb@127.0.0.1)3\PYGZgt{} tanodb\PYGZus{}metrics:all().}
\PYG{g+go}{[\PYGZob{}tanodb,[}

\PYG{g+go}{ ...}

\PYG{g+go}{ \PYGZob{}core,[\PYGZob{}ping,[\PYGZob{}count,1\PYGZcb{},\PYGZob{}one,1\PYGZcb{}]\PYGZcb{}]\PYGZcb{}]}
\end{Verbatim}


\section{Erlang Runtime Metrics}
\label{metrics:erlang-runtime-metrics}
Until now we have metrics for riak\_core and for our API, it would be useful to
have some metrics about the Erlang Runtime, like memory, GC, processes,
schedulers etc. For that we will use a really nice library called \href{https://github.com/ferd/recon}{recon} which unified all the information gathering behind
a nice API.

We start by \href{https://github.com/marianoguerra/tanodb/commit/8d6535f360d24a1486bd7b1ed14d7fcde8c465bb\#diff-31d7a50c99c265ca2793c20961b60979R6}{adding recon as a dependency},  then we \href{https://github.com/marianoguerra/tanodb/commit/8d6535f360d24a1486bd7b1ed14d7fcde8c465bb\#diff-afa3f67ec87f742d64ee9ed311455777R24}{create the function tanodb\_metrics:node\_stats/0} and add it to \href{https://github.com/marianoguerra/tanodb/commit/8d6535f360d24a1486bd7b1ed14d7fcde8c465bb\#diff-afa3f67ec87f742d64ee9ed311455777R10}{tanodb\_metrics:all/0}.


\subsection{Test it}
\label{metrics:id1}
Stop, build a release and run. In the shell run:

\begin{Verbatim}[commandchars=\\\{\}]
\PYG{g+go}{(tanodb@127.0.0.1)1\PYGZgt{} tanodb\PYGZus{}metrics:all().}
\PYG{g+go}{[\PYGZob{}tanodb,[}

\PYG{g+go}{    ...}

\PYG{g+go}{ \PYGZob{}node,[\PYGZob{}abs,[\PYGZob{}process\PYGZus{}count,377\PYGZcb{},}
\PYG{g+go}{              \PYGZob{}run\PYGZus{}queue,0\PYGZcb{},}
\PYG{g+go}{              \PYGZob{}error\PYGZus{}logger\PYGZus{}queue\PYGZus{}len,0\PYGZcb{},}
\PYG{g+go}{              \PYGZob{}memory\PYGZus{}total,30418240\PYGZcb{},}
\PYG{g+go}{              \PYGZob{}memory\PYGZus{}procs,11745496\PYGZcb{},}
\PYG{g+go}{              \PYGZob{}memory\PYGZus{}atoms,458994\PYGZcb{},}
\PYG{g+go}{              \PYGZob{}memory\PYGZus{}bin,232112\PYGZcb{},}
\PYG{g+go}{              \PYGZob{}memory\PYGZus{}ets,1470872\PYGZcb{}]\PYGZcb{},}
\PYG{g+go}{        \PYGZob{}inc,[\PYGZob{}bytes\PYGZus{}in,11737\PYGZcb{},}
\PYG{g+go}{              \PYGZob{}bytes\PYGZus{}out,2470\PYGZcb{},}
\PYG{g+go}{              \PYGZob{}gc\PYGZus{}count,7\PYGZcb{},}
\PYG{g+go}{              \PYGZob{}gc\PYGZus{}words\PYGZus{}reclaimed,29948\PYGZcb{},}
\PYG{g+go}{              \PYGZob{}reductions,2601390\PYGZcb{},}
\PYG{g+go}{              \PYGZob{}scheduler\PYGZus{}usage,[\PYGZob{}1,0.9291112866248371\PYGZcb{},}
\PYG{g+go}{                                \PYGZob{}2,0.04754016011809648\PYGZcb{},}
\PYG{g+go}{                                \PYGZob{}3,0.04615958261183974\PYGZcb{},}
\PYG{g+go}{                                \PYGZob{}4,0.03682005933534583\PYGZcb{}]\PYGZcb{}]\PYGZcb{}]\PYGZcb{},}
\PYG{g+go}{ \PYGZob{}core,[\PYGZob{}ping,[\PYGZob{}count,0\PYGZcb{},\PYGZob{}one,0\PYGZcb{}]\PYGZcb{}]\PYGZcb{}]}
\end{Verbatim}

The metrics should be self explanatory, check \href{http://ferd.github.io/recon/}{the recon documentation} for details.


\section{Web Server Metrics (Cowboy)}
\label{metrics:web-server-metrics-cowboy}
We will start with some generic web server metrics, you can add specific ones
with what you have learned in this chapter and by reading \href{https://github.com/Feuerlabs/exometer/tree/master/doc}{the exometer docs}.

For the generic metrics we will use \href{https://github.com/marianoguerra/cowboy\_exometer}{cowboy\_exometer} which is a module I just wrote since it was quite generic :)

We start by adding the \href{https://github.com/marianoguerra/tanodb/commit/8fb792bc01ac58fbdc709a0c9d2f960605255e54\#diff-31d7a50c99c265ca2793c20961b60979R7}{cowboy\_exometer dependency}, this module exposes a middleware and a response hook
to register metrics on all requests, for that we need to \href{https://github.com/marianoguerra/tanodb/commit/8fb792bc01ac58fbdc709a0c9d2f960605255e54\#diff-afa3f67ec87f742d64ee9ed311455777R20}{initialize it providing the endpoints we care about} and when we want to collect the metrics we \href{https://github.com/marianoguerra/tanodb/commit/8fb792bc01ac58fbdc709a0c9d2f960605255e54\#diff-afa3f67ec87f742d64ee9ed311455777R11}{call cowboy\_exometer:stats/1 passing the same endpoints we passed on init}.

Finally we need to tell cowboy that we will \href{https://github.com/marianoguerra/tanodb/commit/8fb792bc01ac58fbdc709a0c9d2f960605255e54\#diff-4477d4dd0aa2db0e274a56c9158207bdR38}{add a middleware and a response hook}.


\subsection{Test it}
\label{metrics:id2}
After all of this, stop, build, run and make some requests:

\begin{Verbatim}[commandchars=\\\{\}]
http localhost:8080/ping
\end{Verbatim}

and then on the node shell ask for the metrics:

\begin{Verbatim}[commandchars=\\\{\}]
\PYG{g+go}{(tanodb@127.0.0.1)1\PYGZgt{} tanodb\PYGZus{}metrics:all().}
\PYG{g+go}{[\PYGZob{}tanodb,[}

\PYG{g+go}{    ...}

\PYG{g+go}{ \PYGZob{}http,[\PYGZob{}resp,[\PYGZob{}by\PYGZus{}code,[\PYGZob{}200,[\PYGZob{}count,1\PYGZcb{},\PYGZob{}one,1\PYGZcb{}]\PYGZcb{},}
\PYG{g+go}{                         \PYGZob{}201,[\PYGZob{}count,0\PYGZcb{},\PYGZob{}one,0\PYGZcb{}]\PYGZcb{},}
\PYG{g+go}{                         \PYGZob{}202,[\PYGZob{}count,0\PYGZcb{},\PYGZob{}one,0\PYGZcb{}]\PYGZcb{},}
\PYG{g+go}{                         \PYGZob{}203,[\PYGZob{}count,0\PYGZcb{},\PYGZob{}one,0\PYGZcb{}]\PYGZcb{},}
\PYG{g+go}{                         \PYGZob{}204,[\PYGZob{}count,0\PYGZcb{},\PYGZob{}one,0\PYGZcb{}]\PYGZcb{},}
\PYG{g+go}{                         \PYGZob{}205,[\PYGZob{}count,0\PYGZcb{},\PYGZob{}one,0\PYGZcb{}]\PYGZcb{},}
\PYG{g+go}{                         \PYGZob{}206,[\PYGZob{}count,0\PYGZcb{},\PYGZob{}one,0\PYGZcb{}]\PYGZcb{},}
\PYG{g+go}{                         \PYGZob{}300,[\PYGZob{}count,0\PYGZcb{},\PYGZob{}one,0\PYGZcb{}]\PYGZcb{},}
\PYG{g+go}{                         \PYGZob{}301,[\PYGZob{}count,0\PYGZcb{},\PYGZob{}one,0\PYGZcb{}]\PYGZcb{},}
\PYG{g+go}{                         \PYGZob{}302,[\PYGZob{}count,0\PYGZcb{},\PYGZob{}one,0\PYGZcb{}]\PYGZcb{},}
\PYG{g+go}{                         \PYGZob{}303,[\PYGZob{}count,0\PYGZcb{},\PYGZob{}one,0\PYGZcb{}]\PYGZcb{},}
\PYG{g+go}{                         \PYGZob{}304,[\PYGZob{}count,0\PYGZcb{},\PYGZob{}one,0\PYGZcb{}]\PYGZcb{},}
\PYG{g+go}{                         \PYGZob{}305,[\PYGZob{}count,0\PYGZcb{},\PYGZob{}one,0\PYGZcb{}]\PYGZcb{},}
\PYG{g+go}{                         \PYGZob{}306,[\PYGZob{}count,0\PYGZcb{},\PYGZob{}one,...\PYGZcb{}]\PYGZcb{},}
\PYG{g+go}{                         \PYGZob{}307,[\PYGZob{}count,...\PYGZcb{},\PYGZob{}...\PYGZcb{}]\PYGZcb{},}
\PYG{g+go}{                         \PYGZob{}308,[\PYGZob{}...\PYGZcb{}\textbar{}...]\PYGZcb{},}
\PYG{g+go}{                         \PYGZob{}400,[...]\PYGZcb{},}
\PYG{g+go}{                         \PYGZob{}401,...\PYGZcb{},}
\PYG{g+go}{                         \PYGZob{}...\PYGZcb{}\textbar{}...]\PYGZcb{}]\PYGZcb{},}
\PYG{g+go}{        \PYGZob{}req,[\PYGZob{}time,[\PYGZob{}\PYGZlt{}\PYGZlt{}\PYGZdq{}ping\PYGZdq{}\PYGZgt{}\PYGZgt{},}
\PYG{g+go}{                      [\PYGZob{}n,3\PYGZcb{},}
\PYG{g+go}{                       \PYGZob{}mean,44126\PYGZcb{},}
\PYG{g+go}{                       \PYGZob{}min,44126\PYGZcb{},}
\PYG{g+go}{                       \PYGZob{}max,44126\PYGZcb{},}
\PYG{g+go}{                       \PYGZob{}median,44126\PYGZcb{},}
\PYG{g+go}{                       \PYGZob{}50,0\PYGZcb{},}
\PYG{g+go}{                       \PYGZob{}75,44126\PYGZcb{},}
\PYG{g+go}{                       \PYGZob{}90,44126\PYGZcb{},}
\PYG{g+go}{                       \PYGZob{}95,44126\PYGZcb{},}
\PYG{g+go}{                       \PYGZob{}99,44126\PYGZcb{},}
\PYG{g+go}{                       \PYGZob{}999,44126\PYGZcb{}]\PYGZcb{}]\PYGZcb{},}
\PYG{g+go}{              \PYGZob{}active,[\PYGZob{}value,0\PYGZcb{},\PYGZob{}ms\PYGZus{}since\PYGZus{}reset,11546\PYGZcb{}]\PYGZcb{},}
\PYG{g+go}{              \PYGZob{}count,[\PYGZob{}\PYGZlt{}\PYGZlt{}\PYGZdq{}ping\PYGZdq{}\PYGZgt{}\PYGZgt{},[\PYGZob{}count,1\PYGZcb{},\PYGZob{}one,1\PYGZcb{}]\PYGZcb{}]\PYGZcb{}]\PYGZcb{}]\PYGZcb{},}
\PYG{g+go}{ \PYGZob{}node,[\PYGZob{}abs,[\PYGZob{}process\PYGZus{}count,428\PYGZcb{},}
\PYG{g+go}{              \PYGZob{}run\PYGZus{}queue,0\PYGZcb{},}
\PYG{g+go}{              \PYGZob{}error\PYGZus{}logger\PYGZus{}queue\PYGZus{}len,0\PYGZcb{},}
\PYG{g+go}{              \PYGZob{}memory\PYGZus{}total,50301760\PYGZcb{},}
\PYG{g+go}{              \PYGZob{}memory\PYGZus{}procs,30854096\PYGZcb{},}
\PYG{g+go}{              \PYGZob{}memory\PYGZus{}atoms,471201\PYGZcb{},}
\PYG{g+go}{              \PYGZob{}memory\PYGZus{}bin,222648\PYGZcb{},}
\PYG{g+go}{              \PYGZob{}memory\PYGZus{}ets,1574728\PYGZcb{}]\PYGZcb{},}
\PYG{g+go}{        \PYGZob{}inc,[\PYGZob{}bytes\PYGZus{}in,11737\PYGZcb{},}
\PYG{g+go}{              \PYGZob{}bytes\PYGZus{}out,2470\PYGZcb{},}
\PYG{g+go}{              \PYGZob{}gc\PYGZus{}count,6\PYGZcb{},}
\PYG{g+go}{              \PYGZob{}gc\PYGZus{}words\PYGZus{}reclaimed,29747\PYGZcb{},}
\PYG{g+go}{              \PYGZob{}reductions,2848780\PYGZcb{},}
\PYG{g+go}{              \PYGZob{}scheduler\PYGZus{}usage,[\PYGZob{}1,0.05329944038387727\PYGZcb{},}
\PYG{g+go}{                                \PYGZob{}2,0.8991375098414373\PYGZcb{},}
\PYG{g+go}{                                \PYGZob{}3,0.03932163131802264\PYGZcb{},}
\PYG{g+go}{                                \PYGZob{}4,0.05719991628720056\PYGZcb{}]\PYGZcb{}]\PYGZcb{}]\PYGZcb{},}
\PYG{g+go}{ \PYGZob{}core,[\PYGZob{}ping,[\PYGZob{}count,1\PYGZcb{},\PYGZob{}one,1\PYGZcb{}]\PYGZcb{}]\PYGZcb{}]}
\end{Verbatim}

You can see on this line that I made one request to ping and it returned 200:

\begin{Verbatim}[commandchars=\\\{\}]
\PYG{g+go}{\PYGZob{}http,[\PYGZob{}resp,[\PYGZob{}by\PYGZus{}code,[\PYGZob{}200,[\PYGZob{}count,1\PYGZcb{},\PYGZob{}one,1\PYGZcb{}]\PYGZcb{},}
\end{Verbatim}

You can also see request time stats per endpoint:

\begin{Verbatim}[commandchars=\\\{\}]
\PYG{g+go}{\PYGZob{}req,[\PYGZob{}time,[\PYGZob{}\PYGZlt{}\PYGZlt{}\PYGZdq{}ping\PYGZdq{}\PYGZgt{}\PYGZgt{},}
\PYG{g+go}{              [\PYGZob{}n,3\PYGZcb{},}
\PYG{g+go}{               \PYGZob{}mean,44126\PYGZcb{},}
\PYG{g+go}{               \PYGZob{}min,44126\PYGZcb{},}
\PYG{g+go}{               \PYGZob{}max,44126\PYGZcb{},}
\PYG{g+go}{               \PYGZob{}median,44126\PYGZcb{},}
\PYG{g+go}{               \PYGZob{}50,0\PYGZcb{},}
\PYG{g+go}{               \PYGZob{}75,44126\PYGZcb{},}
\PYG{g+go}{               \PYGZob{}90,44126\PYGZcb{},}
\PYG{g+go}{               \PYGZob{}95,44126\PYGZcb{},}
\PYG{g+go}{               \PYGZob{}99,44126\PYGZcb{},}
\PYG{g+go}{               \PYGZob{}999,44126\PYGZcb{}]\PYGZcb{}]\PYGZcb{},}
\end{Verbatim}

And request count by endpoint:

\begin{Verbatim}[commandchars=\\\{\}]
\PYG{g+go}{\PYGZob{}count,[\PYGZob{}\PYGZlt{}\PYGZlt{}\PYGZdq{}ping\PYGZdq{}\PYGZgt{}\PYGZgt{},[\PYGZob{}count,1\PYGZcb{},\PYGZob{}one,1\PYGZcb{}]\PYGZcb{}]\PYGZcb{}]\PYGZcb{}]\PYGZcb{},}
\end{Verbatim}


\chapter{Riak Core Metadata}
\label{riak_core_metadata:riak-core-metadata}\label{riak_core_metadata::doc}
\begin{notice}{note}{Note:}
The first 3 sections are taken from here
\href{https://gist.github.com/jrwest/d290c14e1c472e562548}{https://gist.github.com/jrwest/d290c14e1c472e562548}
\end{notice}


\section{1. Overview}
\label{riak_core_metadata:overview}
Cluster Metadata is intended to be used by \emph{riak\_core} applications
wishing to work with information stored cluster-wide. It is useful for
storing application metadata or any information that needs to be
read without blocking on communication over the network.


\subsection{1.1 Data Model}
\label{riak_core_metadata:data-model}
Cluster Metadata is a key-value store. It treats values as opaque
Erlang terms that are fully addressed by their ``Full Prefix'' and
``Key''. A Full Prefix is a \emph{\{atom() \textbar{} binary(), atom() \textbar{} binary()\}},
while a Key is any Erlang term. The first element of the Full Prefix
is referred to as the ``Prefix'' and the second as the ``Sub-Prefix''.


\subsection{1.2 Storage}
\label{riak_core_metadata:storage}
Values are stored on-disk and a full copy is also maintained
in-memory. This allows reads to be performed only from memory, while
writes are affected in both mediums.


\subsection{1.3 Consistency}
\label{riak_core_metadata:consistency}
Updates in Cluster Metadata are eventually consistent. Writing a value
only requires acknowledgment from a single node and as previously
mentioned, reads return values from the local node, only.


\subsection{1.4 Replication}
\label{riak_core_metadata:replication}
Updates are replicated to every node in the cluster, including nodes
that join the cluster after the update has already reached all nodes
in the previous set of members.


\section{2. API}
\label{riak_core_metadata:api}
The interface to Cluster Metadata is provided by the
\href{https://github.com/basho/riak\_core/blob/develop/src/riak\_core\_metadata.erl}{riak\_core\_metadata}
module. The module's documentation is the official source for
information about the API, but some details are re-iterated here.


\subsection{2.1 Reading and Writing}
\label{riak_core_metadata:reading-and-writing}
Reading the local value for a key can be done with the \emph{get/2,3}
functions. Like most \emph{riak\_core\_metadata} functions, the higher arity
version takes a set of possible options, while the lower arity
function uses the defaults.

Updating a key is done using \emph{put/3.4}. Performing a put only blocks
until the write is affected locally. The broadcast of the update is
done asynchronously.


\subsubsection{2.1.1 Deleting Keys}
\label{riak_core_metadata:deleting-keys}
Deletion of keys is logical and tombstones are not
reaped. \emph{delete/2,3} act the same as \emph{put/3,4} with respect to
blocking and broadcast.


\subsection{2.2 Iterators}
\label{riak_core_metadata:iterators}
Along with reading individual keys, the API also allows Full Prefixes
to be iterated over. Iterators can visit both keys and values. They
are not ordered, nor are they read-isolated. However, they do
guarantee that each key is seen \emph{at most once} for the lifetime of an
iterator.

See \emph{iterator/2} and the \emph{itr\_*} functions.


\subsection{2.3 Conflict Resolution}
\label{riak_core_metadata:conflict-resolution}
Conflict resolution can be done on read or write.

On read, if the conflict is resolved, an option, \emph{allow\_put}, passed
to \emph{get/3} or \emph{iterator/2} controls whether or not the resolved value
will be written back to local storage and broadcast asynchronously.

On write, conflicts are resolved by passing a function instead of a
new value to \emph{put/3,4}. The function is passed the list of existing
values and can use this and values captured within the closure to
produce a new value to store.


\subsection{2.4 Detecting Changes in Groups of Keys}
\label{riak_core_metadata:detecting-changes-in-groups-of-keys}
The \emph{prefix\_hash/1} function can be polled to determined when groups
of keys, by Prefix or Full Prefix, have changed.


\section{3. Common Pitfalls \& Other Notes}
\label{riak_core_metadata:common-pitfalls-other-notes}
The following is by no means a complete list of things to keep in mind
when developing on top of Cluster Metadata.


\subsection{3.1 Storage Limitations}
\label{riak_core_metadata:storage-limitations}
Cluster Metadata use \emph{dets} for on-disk storage. There is a \emph{dets}
table per Full Prefix, which limits the amount of data stored under
each Full Prefix to 2GB. This size includes the overhead of
information stored alongside values, such as the logical clock and
key.

Since a full-copy of the data is kept in-memory, its usage must also
be considered.


\subsection{3.2 Replication Limitations}
\label{riak_core_metadata:replication-limitations}
Cluster Metadata uses disterl for message delivery, like most Erlang
applications. Standard caveats and issues with large and/or too
frequent messages still apply.


\subsection{3.3 Last-Write Wins}
\label{riak_core_metadata:last-write-wins}
The default conflict resolution strategy on read is
last-write-wins. The usual caveats about the dangers of this method
apply.


\subsection{3.4 ``Pathological Eventual Consistency''}
\label{riak_core_metadata:pathological-eventual-consistency}
The extremely frequent writing back of resolved values after read in
an eventually consistent store where acknowledgment is only required
from one node for both types of operations can lead to an interesting
pathological case where siblings continue to be produce (although the
set does not grow unbounded). A very rough exploration of this can be
found \href{https://gist.github.com/jrwest/f8c0d49174f4db1c4c88}{here}).

If a \emph{riak\_core} application is likely to have concurrent writes and
wishes to read extremely frequently, e.g. in the Riak request path, it
may be advisable to use \emph{\{allow\_put, false\}} with \emph{get/3}.


\section{4. Playing in the REPL}
\label{riak_core_metadata:playing-in-the-repl}
we start by building and running our app:

\begin{Verbatim}[commandchars=\\\{\}]
rebar3 release
rebar3 run
\end{Verbatim}

First let's setup some variables, FullPrefix is like an identifier for the
place where we are going to store related values, there can be many, some of
them are used by other components of riak\_core as you will see in the next
sections.

\begin{Verbatim}[commandchars=\\\{\}]
\PYG{p}{(}\PYG{n}{tanodb}\PYG{p}{@}\PYG{l+m+mi}{127}\PYG{p}{.}\PYG{l+m+mi}{0}\PYG{p}{.}\PYG{l+m+mi}{0}\PYG{p}{.}\PYG{l+m+mi}{1}\PYG{p}{)}\PYG{l+m+mi}{1}\PYG{o}{\PYGZgt{}} \PYG{n+nv}{FullPrefix} \PYG{o}{=} \PYG{p}{\PYGZob{}}\PYG{o}{\PYGZlt{}}\PYG{o}{\PYGZlt{}}\PYG{l+s}{\PYGZdq{}}\PYG{l+s}{tanodb}\PYG{l+s}{\PYGZdq{}}\PYG{o}{\PYGZgt{}}\PYG{o}{\PYGZgt{}}\PYG{p}{,} \PYG{o}{\PYGZlt{}}\PYG{o}{\PYGZlt{}}\PYG{l+s}{\PYGZdq{}}\PYG{l+s}{config}\PYG{l+s}{\PYGZdq{}}\PYG{o}{\PYGZgt{}}\PYG{o}{\PYGZgt{}}\PYG{p}{\PYGZcb{}}\PYG{p}{.}
\PYG{p}{\PYGZob{}}\PYG{o}{\PYGZlt{}}\PYG{o}{\PYGZlt{}}\PYG{l+s}{\PYGZdq{}}\PYG{l+s}{tanodb}\PYG{l+s}{\PYGZdq{}}\PYG{o}{\PYGZgt{}}\PYG{o}{\PYGZgt{}}\PYG{p}{,}\PYG{o}{\PYGZlt{}}\PYG{o}{\PYGZlt{}}\PYG{l+s}{\PYGZdq{}}\PYG{l+s}{config}\PYG{l+s}{\PYGZdq{}}\PYG{o}{\PYGZgt{}}\PYG{o}{\PYGZgt{}}\PYG{p}{\PYGZcb{}}
\end{Verbatim}

Let's start by trying to get a value that is not set, by default we get undefined.

\begin{Verbatim}[commandchars=\\\{\}]
\PYG{p}{(}\PYG{n}{tanodb}\PYG{p}{@}\PYG{l+m+mi}{127}\PYG{p}{.}\PYG{l+m+mi}{0}\PYG{p}{.}\PYG{l+m+mi}{0}\PYG{p}{.}\PYG{l+m+mi}{1}\PYG{p}{)}\PYG{l+m+mi}{2}\PYG{o}{\PYGZgt{}} \PYG{n+nn}{riak\PYGZus{}core\PYGZus{}metadata}\PYG{p}{:}\PYG{n+nb}{get}\PYG{p}{(}\PYG{n+nv}{FullPrefix}\PYG{p}{,} \PYG{n}{max\PYGZus{}users}\PYG{p}{)}\PYG{p}{.}
\PYG{n}{undefined}
\end{Verbatim}

We can change that by calling the get function that supports options, one of
them is \textbf{default}, so we set it to a value that makes sense for use in case
max\_users is not set.

\begin{Verbatim}[commandchars=\\\{\}]
\PYG{p}{(}\PYG{n}{tanodb}\PYG{p}{@}\PYG{l+m+mi}{127}\PYG{p}{.}\PYG{l+m+mi}{0}\PYG{p}{.}\PYG{l+m+mi}{0}\PYG{p}{.}\PYG{l+m+mi}{1}\PYG{p}{)}\PYG{l+m+mi}{3}\PYG{o}{\PYGZgt{}} \PYG{n+nn}{riak\PYGZus{}core\PYGZus{}metadata}\PYG{p}{:}\PYG{n+nb}{get}\PYG{p}{(}\PYG{n+nv}{FullPrefix}\PYG{p}{,} \PYG{n}{max\PYGZus{}users}\PYG{p}{,} \PYG{p}{[}\PYG{p}{\PYGZob{}}\PYG{n}{default}\PYG{p}{,} \PYG{l+m+mi}{100}\PYG{p}{\PYGZcb{}}\PYG{p}{]}\PYG{p}{)}\PYG{p}{.}
\PYG{l+m+mi}{100}
\end{Verbatim}

Now let's put the value in the store.

\begin{Verbatim}[commandchars=\\\{\}]
\PYG{p}{(}\PYG{n}{tanodb}\PYG{p}{@}\PYG{l+m+mi}{127}\PYG{p}{.}\PYG{l+m+mi}{0}\PYG{p}{.}\PYG{l+m+mi}{0}\PYG{p}{.}\PYG{l+m+mi}{1}\PYG{p}{)}\PYG{l+m+mi}{4}\PYG{o}{\PYGZgt{}} \PYG{n+nn}{riak\PYGZus{}core\PYGZus{}metadata}\PYG{p}{:}\PYG{n+nb}{put}\PYG{p}{(}\PYG{n+nv}{FullPrefix}\PYG{p}{,} \PYG{n}{max\PYGZus{}users}\PYG{p}{,} \PYG{l+m+mi}{150}\PYG{p}{)}\PYG{p}{.}
\PYG{n}{ok}
\end{Verbatim}

And try getting it.

\begin{Verbatim}[commandchars=\\\{\}]
\PYG{p}{(}\PYG{n}{tanodb}\PYG{p}{@}\PYG{l+m+mi}{127}\PYG{p}{.}\PYG{l+m+mi}{0}\PYG{p}{.}\PYG{l+m+mi}{0}\PYG{p}{.}\PYG{l+m+mi}{1}\PYG{p}{)}\PYG{l+m+mi}{5}\PYG{o}{\PYGZgt{}} \PYG{n+nn}{riak\PYGZus{}core\PYGZus{}metadata}\PYG{p}{:}\PYG{n+nb}{get}\PYG{p}{(}\PYG{n+nv}{FullPrefix}\PYG{p}{,} \PYG{n}{max\PYGZus{}users}\PYG{p}{)}\PYG{p}{.}
\PYG{l+m+mi}{150}
\end{Verbatim}

Let's put another value.

\begin{Verbatim}[commandchars=\\\{\}]
\PYG{p}{(}\PYG{n}{tanodb}\PYG{p}{@}\PYG{l+m+mi}{127}\PYG{p}{.}\PYG{l+m+mi}{0}\PYG{p}{.}\PYG{l+m+mi}{0}\PYG{p}{.}\PYG{l+m+mi}{1}\PYG{p}{)}\PYG{l+m+mi}{6}\PYG{o}{\PYGZgt{}} \PYG{n+nn}{riak\PYGZus{}core\PYGZus{}metadata}\PYG{p}{:}\PYG{n+nb}{put}\PYG{p}{(}\PYG{n+nv}{FullPrefix}\PYG{p}{,} \PYG{n}{max\PYGZus{}connections}\PYG{p}{,} \PYG{l+m+mi}{100}\PYG{p}{)}\PYG{p}{.}
\PYG{n}{ok}
\end{Verbatim}

Get all the values in this prefix as a list, the ``d'' there is because {[}100{]} looks
like a string to erlang, don't worry, your value is safe.

\begin{Verbatim}[commandchars=\\\{\}]
\PYG{p}{(}\PYG{n}{tanodb}\PYG{p}{@}\PYG{l+m+mi}{127}\PYG{p}{.}\PYG{l+m+mi}{0}\PYG{p}{.}\PYG{l+m+mi}{0}\PYG{p}{.}\PYG{l+m+mi}{1}\PYG{p}{)}\PYG{l+m+mi}{7}\PYG{o}{\PYGZgt{}} \PYG{n+nn}{riak\PYGZus{}core\PYGZus{}metadata}\PYG{p}{:}\PYG{n+nf}{to\PYGZus{}list}\PYG{p}{(}\PYG{n+nv}{FullPrefix}\PYG{p}{)}\PYG{p}{.}
\PYG{p}{[}\PYG{p}{\PYGZob{}}\PYG{n}{max\PYGZus{}connections}\PYG{p}{,}\PYG{l+s}{\PYGZdq{}}\PYG{l+s}{d}\PYG{l+s}{\PYGZdq{}}\PYG{p}{\PYGZcb{}}\PYG{p}{,}\PYG{p}{\PYGZob{}}\PYG{n}{max\PYGZus{}users}\PYG{p}{,}\PYG{p}{[}\PYG{l+m+mi}{150}\PYG{p}{]}\PYG{p}{\PYGZcb{}}\PYG{p}{]}
\end{Verbatim}

Now let's delete a value.

\begin{Verbatim}[commandchars=\\\{\}]
\PYG{p}{(}\PYG{n}{tanodb}\PYG{p}{@}\PYG{l+m+mi}{127}\PYG{p}{.}\PYG{l+m+mi}{0}\PYG{p}{.}\PYG{l+m+mi}{0}\PYG{p}{.}\PYG{l+m+mi}{1}\PYG{p}{)}\PYG{l+m+mi}{8}\PYG{o}{\PYGZgt{}} \PYG{n+nn}{riak\PYGZus{}core\PYGZus{}metadata}\PYG{p}{:}\PYG{n+nf}{delete}\PYG{p}{(}\PYG{n+nv}{FullPrefix}\PYG{p}{,} \PYG{n}{max\PYGZus{}users}\PYG{p}{)}\PYG{p}{.}
\PYG{n}{ok}
\end{Verbatim}

And put another one.

\begin{Verbatim}[commandchars=\\\{\}]
\PYG{p}{(}\PYG{n}{tanodb}\PYG{p}{@}\PYG{l+m+mi}{127}\PYG{p}{.}\PYG{l+m+mi}{0}\PYG{p}{.}\PYG{l+m+mi}{0}\PYG{p}{.}\PYG{l+m+mi}{1}\PYG{p}{)}\PYG{l+m+mi}{9}\PYG{o}{\PYGZgt{}} \PYG{n+nn}{riak\PYGZus{}core\PYGZus{}metadata}\PYG{p}{:}\PYG{n+nb}{put}\PYG{p}{(}\PYG{n+nv}{FullPrefix}\PYG{p}{,} \PYG{n}{hostname}\PYG{p}{,} \PYG{l+s}{\PYGZdq{}}\PYG{l+s}{tanodb1}\PYG{l+s}{\PYGZdq{}}\PYG{p}{)}\PYG{p}{.}
\PYG{n}{ok}
\end{Verbatim}

Now let's list them again, you will see that deleted values are still there but
\textbf{marked} with a ``thombstone'' value (the atom `\$deleted'), this means we have
to handle them in our functions if we want to avoid trouble.

\begin{Verbatim}[commandchars=\\\{\}]
\PYG{p}{(}\PYG{n}{tanodb}\PYG{p}{@}\PYG{l+m+mi}{127}\PYG{p}{.}\PYG{l+m+mi}{0}\PYG{p}{.}\PYG{l+m+mi}{0}\PYG{p}{.}\PYG{l+m+mi}{1}\PYG{p}{)}\PYG{l+m+mi}{11}\PYG{o}{\PYGZgt{}} \PYG{n+nn}{riak\PYGZus{}core\PYGZus{}metadata}\PYG{p}{:}\PYG{n+nf}{to\PYGZus{}list}\PYG{p}{(}\PYG{n+nv}{FullPrefix}\PYG{p}{)}\PYG{p}{.}
\PYG{p}{[}\PYG{p}{\PYGZob{}}\PYG{n}{max\PYGZus{}connections}\PYG{p}{,}\PYG{l+s}{\PYGZdq{}}\PYG{l+s}{d}\PYG{l+s}{\PYGZdq{}}\PYG{p}{\PYGZcb{}}\PYG{p}{,}
 \PYG{p}{\PYGZob{}}\PYG{n}{max\PYGZus{}users}\PYG{p}{,}\PYG{p}{[}\PYG{n}{\PYGZsq{}\PYGZdl{}deleted\PYGZsq{}}\PYG{p}{]}\PYG{p}{\PYGZcb{}}\PYG{p}{,}
 \PYG{p}{\PYGZob{}}\PYG{n}{hostname}\PYG{p}{,}\PYG{p}{[}\PYG{l+s}{\PYGZdq{}}\PYG{l+s}{tanodb1}\PYG{l+s}{\PYGZdq{}}\PYG{p}{]}\PYG{p}{\PYGZcb{}}\PYG{p}{]}
\end{Verbatim}

Now let's do something more complex, let's iterate over all the values in the
prefix, count the amount of deleted values and accumulate the ``alive'' ones.

Notice I use a function clause to match the thombstone first and then one to
handle ``alive'' values.

\begin{Verbatim}[commandchars=\\\{\}]
\PYG{p}{(}\PYG{n}{tanodb}\PYG{p}{@}\PYG{l+m+mi}{127}\PYG{p}{.}\PYG{l+m+mi}{0}\PYG{p}{.}\PYG{l+m+mi}{0}\PYG{p}{.}\PYG{l+m+mi}{1}\PYG{p}{)}\PYG{l+m+mi}{11}\PYG{o}{\PYGZgt{}} \PYG{n+nn}{riak\PYGZus{}core\PYGZus{}metadata}\PYG{p}{:}\PYG{n+nf}{fold}\PYG{p}{(}\PYG{k}{fun}
\PYG{p}{(}\PYG{n}{tanodb}\PYG{p}{@}\PYG{l+m+mi}{127}\PYG{p}{.}\PYG{l+m+mi}{0}\PYG{p}{.}\PYG{l+m+mi}{0}\PYG{p}{.}\PYG{l+m+mi}{1}\PYG{p}{)}\PYG{l+m+mi}{11}\PYG{o}{\PYGZgt{}}     \PYG{p}{(}\PYG{p}{\PYGZob{}}\PYG{n+nv}{Key}\PYG{p}{,} \PYG{p}{[}\PYG{n}{\PYGZsq{}\PYGZdl{}deleted\PYGZsq{}}\PYG{p}{]}\PYG{p}{\PYGZcb{}}\PYG{p}{,} \PYG{p}{\PYGZob{}}\PYG{n+nv}{Deleted}\PYG{p}{,} \PYG{n+nv}{Alive}\PYG{p}{\PYGZcb{}}\PYG{p}{)} \PYG{o}{\PYGZhy{}}\PYG{o}{\PYGZgt{}}
\PYG{p}{(}\PYG{n}{tanodb}\PYG{p}{@}\PYG{l+m+mi}{127}\PYG{p}{.}\PYG{l+m+mi}{0}\PYG{p}{.}\PYG{l+m+mi}{0}\PYG{p}{.}\PYG{l+m+mi}{1}\PYG{p}{)}\PYG{l+m+mi}{11}\PYG{o}{\PYGZgt{}}         \PYG{p}{\PYGZob{}}\PYG{n+nv}{Deleted} \PYG{o}{+} \PYG{l+m+mi}{1}\PYG{p}{,} \PYG{n+nv}{Alive}\PYG{p}{\PYGZcb{}}\PYG{p}{;}
\PYG{p}{(}\PYG{n}{tanodb}\PYG{p}{@}\PYG{l+m+mi}{127}\PYG{p}{.}\PYG{l+m+mi}{0}\PYG{p}{.}\PYG{l+m+mi}{0}\PYG{p}{.}\PYG{l+m+mi}{1}\PYG{p}{)}\PYG{l+m+mi}{11}\PYG{o}{\PYGZgt{}}     \PYG{p}{(}\PYG{p}{\PYGZob{}}\PYG{n+nv}{Key}\PYG{p}{,} \PYG{p}{[}\PYG{n+nv}{Value}\PYG{p}{]}\PYG{p}{\PYGZcb{}}\PYG{p}{,} \PYG{p}{\PYGZob{}}\PYG{n+nv}{Deleted}\PYG{p}{,} \PYG{n+nv}{Alive}\PYG{p}{\PYGZcb{}}\PYG{p}{)} \PYG{o}{\PYGZhy{}}\PYG{o}{\PYGZgt{}}
\PYG{p}{(}\PYG{n}{tanodb}\PYG{p}{@}\PYG{l+m+mi}{127}\PYG{p}{.}\PYG{l+m+mi}{0}\PYG{p}{.}\PYG{l+m+mi}{0}\PYG{p}{.}\PYG{l+m+mi}{1}\PYG{p}{)}\PYG{l+m+mi}{11}\PYG{o}{\PYGZgt{}}         \PYG{p}{\PYGZob{}}\PYG{n+nv}{Deleted}\PYG{p}{,} \PYG{p}{[}\PYG{p}{\PYGZob{}}\PYG{n+nv}{Key}\PYG{p}{,} \PYG{n+nv}{Value}\PYG{p}{\PYGZcb{}}\PYG{p}{\textbar{}}\PYG{n+nv}{Alive}\PYG{p}{]}\PYG{p}{\PYGZcb{}}
\PYG{p}{(}\PYG{n}{tanodb}\PYG{p}{@}\PYG{l+m+mi}{127}\PYG{p}{.}\PYG{l+m+mi}{0}\PYG{p}{.}\PYG{l+m+mi}{0}\PYG{p}{.}\PYG{l+m+mi}{1}\PYG{p}{)}\PYG{l+m+mi}{11}\PYG{o}{\PYGZgt{}} \PYG{k}{end}\PYG{p}{,} \PYG{p}{\PYGZob{}}\PYG{l+m+mi}{0}\PYG{p}{,} \PYG{p}{[}\PYG{p}{]}\PYG{p}{\PYGZcb{}}\PYG{p}{,} \PYG{n+nv}{FullPrefix}\PYG{p}{)}\PYG{p}{.}

\PYG{p}{\PYGZob{}}\PYG{l+m+mi}{1}\PYG{p}{,}\PYG{p}{[}\PYG{p}{\PYGZob{}}\PYG{n}{max\PYGZus{}connections}\PYG{p}{,}\PYG{l+m+mi}{100}\PYG{p}{\PYGZcb{}}\PYG{p}{,}\PYG{p}{\PYGZob{}}\PYG{n}{hostname}\PYG{p}{,}\PYG{l+s}{\PYGZdq{}}\PYG{l+s}{tanodb1}\PYG{l+s}{\PYGZdq{}}\PYG{p}{\PYGZcb{}}\PYG{p}{]}\PYG{p}{\PYGZcb{}}
\end{Verbatim}

There are more functions I didn't show here since this ones are the main ones
you will uses, you can look at the riak\_core\_metadata module for the other ones,
the module has good documentation for each function.


\chapter{Riak Core Security}
\label{riak_core_security:riak-core-security}\label{riak_core_security::doc}
riak\_core\_security is a module in riak\_core that provides facilities to
implement user/group management, authentication and authorization.

Here we will see an overview of it.


\section{Implementation}
\label{riak_core_security:implementation}
riak\_core\_security is implemented on top of riak\_core\_metadata, it uses the
following keys to store its information:

\begin{Verbatim}[commandchars=\\\{\}]
\PYG{p}{\PYGZob{}}\PYG{o}{\PYGZlt{}}\PYG{o}{\PYGZlt{}}\PYG{l+s}{\PYGZdq{}}\PYG{l+s}{security}\PYG{l+s}{\PYGZdq{}}\PYG{o}{\PYGZgt{}}\PYG{o}{\PYGZgt{}}\PYG{p}{,} \PYG{o}{\PYGZlt{}}\PYG{o}{\PYGZlt{}}\PYG{l+s}{\PYGZdq{}}\PYG{l+s}{users}\PYG{l+s}{\PYGZdq{}}\PYG{o}{\PYGZgt{}}\PYG{o}{\PYGZgt{}}\PYG{p}{\PYGZcb{}}
\PYG{p}{\PYGZob{}}\PYG{o}{\PYGZlt{}}\PYG{o}{\PYGZlt{}}\PYG{l+s}{\PYGZdq{}}\PYG{l+s}{security}\PYG{l+s}{\PYGZdq{}}\PYG{o}{\PYGZgt{}}\PYG{o}{\PYGZgt{}}\PYG{p}{,} \PYG{o}{\PYGZlt{}}\PYG{o}{\PYGZlt{}}\PYG{l+s}{\PYGZdq{}}\PYG{l+s}{groups}\PYG{l+s}{\PYGZdq{}}\PYG{o}{\PYGZgt{}}\PYG{o}{\PYGZgt{}}\PYG{p}{\PYGZcb{}}
\PYG{p}{\PYGZob{}}\PYG{o}{\PYGZlt{}}\PYG{o}{\PYGZlt{}}\PYG{l+s}{\PYGZdq{}}\PYG{l+s}{security}\PYG{l+s}{\PYGZdq{}}\PYG{o}{\PYGZgt{}}\PYG{o}{\PYGZgt{}}\PYG{p}{,} \PYG{o}{\PYGZlt{}}\PYG{o}{\PYGZlt{}}\PYG{l+s}{\PYGZdq{}}\PYG{l+s}{sources}\PYG{l+s}{\PYGZdq{}}\PYG{o}{\PYGZgt{}}\PYG{o}{\PYGZgt{}}\PYG{p}{\PYGZcb{}}
\PYG{p}{\PYGZob{}}\PYG{o}{\PYGZlt{}}\PYG{o}{\PYGZlt{}}\PYG{l+s}{\PYGZdq{}}\PYG{l+s}{security}\PYG{l+s}{\PYGZdq{}}\PYG{o}{\PYGZgt{}}\PYG{o}{\PYGZgt{}}\PYG{p}{,} \PYG{o}{\PYGZlt{}}\PYG{o}{\PYGZlt{}}\PYG{l+s}{\PYGZdq{}}\PYG{l+s}{usergrants}\PYG{l+s}{\PYGZdq{}}\PYG{o}{\PYGZgt{}}\PYG{o}{\PYGZgt{}}\PYG{p}{\PYGZcb{}}
\PYG{p}{\PYGZob{}}\PYG{o}{\PYGZlt{}}\PYG{o}{\PYGZlt{}}\PYG{l+s}{\PYGZdq{}}\PYG{l+s}{security}\PYG{l+s}{\PYGZdq{}}\PYG{o}{\PYGZgt{}}\PYG{o}{\PYGZgt{}}\PYG{p}{,} \PYG{o}{\PYGZlt{}}\PYG{o}{\PYGZlt{}}\PYG{l+s}{\PYGZdq{}}\PYG{l+s}{groupgrants}\PYG{l+s}{\PYGZdq{}}\PYG{o}{\PYGZgt{}}\PYG{o}{\PYGZgt{}}\PYG{p}{\PYGZcb{}}
\PYG{p}{\PYGZob{}}\PYG{o}{\PYGZlt{}}\PYG{o}{\PYGZlt{}}\PYG{l+s}{\PYGZdq{}}\PYG{l+s}{security}\PYG{l+s}{\PYGZdq{}}\PYG{o}{\PYGZgt{}}\PYG{o}{\PYGZgt{}}\PYG{p}{,} \PYG{o}{\PYGZlt{}}\PYG{o}{\PYGZlt{}}\PYG{l+s}{\PYGZdq{}}\PYG{l+s}{status}\PYG{l+s}{\PYGZdq{}}\PYG{o}{\PYGZgt{}}\PYG{o}{\PYGZgt{}}\PYG{p}{\PYGZcb{}} \PYG{o}{\PYGZhy{}}\PYG{o}{\PYGZgt{}} \PYG{n}{enabled}
\PYG{p}{\PYGZob{}}\PYG{o}{\PYGZlt{}}\PYG{o}{\PYGZlt{}}\PYG{l+s}{\PYGZdq{}}\PYG{l+s}{security}\PYG{l+s}{\PYGZdq{}}\PYG{o}{\PYGZgt{}}\PYG{o}{\PYGZgt{}}\PYG{p}{,} \PYG{o}{\PYGZlt{}}\PYG{o}{\PYGZlt{}}\PYG{l+s}{\PYGZdq{}}\PYG{l+s}{config}\PYG{l+s}{\PYGZdq{}}\PYG{o}{\PYGZgt{}}\PYG{o}{\PYGZgt{}}\PYG{p}{\PYGZcb{}} \PYG{o}{\PYGZhy{}}\PYG{o}{\PYGZgt{}} \PYG{n}{ciphers}
\end{Verbatim}

How they are stored should be an implementation detail but sometimes you may
need to fold over values to get information if it's not supported by
riak\_core\_security's API.


\section{Vocabulary}
\label{riak_core_security:vocabulary}

\subsection{Context}
\label{riak_core_security:context}
Opaque information you get back from authentication, you have to pass it back
in to other operations.

At the moment it's a record with three fields:
\begin{itemize}
\item {} 
username

\item {} 
grants

\item {} 
epoch

\end{itemize}

But notice that this is an implementation detail and you should handle it
as an opaque value.

Contexts are only valid until the GRANT epoch changes, and it will change
whenever a GRANT or a REVOKE is performed. This rule may change in the future.


\subsection{Permission}
\label{riak_core_security:permission}
A string that represents some action in a given application, for example
tanodb.get, tanodb.put.

A permission muy be listed as valid in the environment variable \{riak\_core, permissions\}:

\begin{Verbatim}[commandchars=\\\{\}]
(tanodb@127.0.0.1)1\PYGZgt{} application:get\PYGZus{}env(riak\PYGZus{}core, permissions).
\PYGZob{}ok,[\PYGZob{}riak\PYGZus{}core,[get\PYGZus{}bucket,set\PYGZus{}bucket,get\PYGZus{}bucket\PYGZus{}type, set\PYGZus{}bucket\PYGZus{}type]\PYGZcb{}]\PYGZcb{}
\end{Verbatim}

You can list your permissions in config/advanced.config uncommenting the line:

\begin{Verbatim}[commandchars=\\\{\}]
\PYGZpc{} \PYGZob{}permissions, [\PYGZob{} tanodb, [put, get, list, grant, delete]\PYGZcb{}]\PYGZcb{}
\end{Verbatim}

And changing the permissions inside the list.

\begin{notice}{note}{Note:}
tanodb is the name of your app
\end{notice}


\subsection{Role}
\label{riak_core_security:role}
Something you assign permissions to, it can be a user or a group, there are
some reserved roles:
\begin{itemize}
\item {} 
all

\item {} 
on

\item {} 
to

\item {} 
from

\item {} 
any

\end{itemize}


\subsection{Source}
\label{riak_core_security:source}
The source where the user is authenticating, it can be an ip or something else,
you can allow a user to authenticate from a source but not another.


\section{Extra Features}
\label{riak_core_security:extra-features}\begin{itemize}
\item {} 
Certificate Authentication

\item {} 
Plugable Authentication

\end{itemize}


\section{API Overview}
\label{riak_core_security:api-overview}

\subsection{check\_permission}
\label{riak_core_security:check-permission}
\begin{Verbatim}[commandchars=\\\{\}]
\PYG{c}{\PYGZpc{} Check a Global permission, one that is not tied to a bucket}
\PYG{n+nf}{check\PYGZus{}permission}\PYG{p}{(}\PYG{p}{\PYGZob{}}\PYG{n+nv}{Permission}\PYG{p}{\PYGZcb{}}\PYG{p}{,} \PYG{n+nv}{Context}\PYG{p}{)}

\PYG{c}{\PYGZpc{} Check a permission for a specific bucket}
\PYG{n+nf}{check\PYGZus{}permission}\PYG{p}{(}\PYG{p}{\PYGZob{}}\PYG{n+nv}{Permission}\PYG{p}{,} \PYG{n+nv}{Bucket}\PYG{p}{\PYGZcb{}}\PYG{p}{,} \PYG{n+nv}{Context}\PYG{p}{)}
\end{Verbatim}


\subsection{check\_permissions}
\label{riak_core_security:check-permissions}
\begin{Verbatim}[commandchars=\\\{\}]
\PYG{c}{\PYGZpc{} Check that all permissions are valid}
\PYG{n+nf}{check\PYGZus{}permissions}\PYG{p}{(}\PYG{n+nv}{List}\PYG{p}{,} \PYG{n+nv}{Ctx}\PYG{p}{)}
\end{Verbatim}


\subsection{get\_username}
\label{riak_core_security:get-username}
\begin{Verbatim}[commandchars=\\\{\}]
\PYG{c}{\PYGZpc{} return username from context}
\PYG{n+nf}{get\PYGZus{}username}\PYG{p}{(}\PYG{n+nv}{Context}\PYG{p}{)}
\end{Verbatim}


\subsection{authenticate}
\label{riak_core_security:authenticate}
If successful it will return \{ok, Context\}

A username can be tied to specific sources from which he can login, if you
don't need this feature specify a generic source for all your users.

\begin{Verbatim}[commandchars=\\\{\}]
\PYG{n+nf}{authenticate}\PYG{p}{(}\PYG{n+nv}{Username}\PYG{p}{,} \PYG{n+nv}{Password}\PYG{p}{,} \PYG{n+nv}{ConnInfo}\PYG{p}{)}
\end{Verbatim}


\subsection{add\_user}
\label{riak_core_security:add-user}
Valid options:
\begin{itemize}
\item {} 
password

\item {} 
groups: groups must be a string with comma separated groups, like ``g1,g2''

\end{itemize}

\begin{Verbatim}[commandchars=\\\{\}]
\PYG{n+nf}{add\PYGZus{}user}\PYG{p}{(}\PYG{n+nv}{Username}\PYG{p}{,} \PYG{n+nv}{Options}\PYG{p}{)}
\end{Verbatim}


\subsection{add\_group}
\label{riak_core_security:add-group}
Valid options:
\begin{itemize}
\item {} 
password

\end{itemize}

\begin{Verbatim}[commandchars=\\\{\}]
\PYG{n+nf}{add\PYGZus{}group}\PYG{p}{(}\PYG{n+nv}{Groupname}\PYG{p}{,} \PYG{n+nv}{Options}\PYG{p}{)}
\end{Verbatim}


\subsection{alter\_user}
\label{riak_core_security:alter-user}
Options passed will override options already in user's details, this means if
you pass a password it will be changed, if you pass groups the new groups will
be set and the old removed.

\begin{Verbatim}[commandchars=\\\{\}]
\PYG{n+nf}{alter\PYGZus{}user}\PYG{p}{(}\PYG{n+nv}{Username}\PYG{p}{,} \PYG{n+nv}{Options}\PYG{p}{)}
\end{Verbatim}


\subsection{alter\_group}
\label{riak_core_security:alter-group}
Options passed will override options already in groups's details, if you pass
groups the new groups will be set and the old removed.

\begin{Verbatim}[commandchars=\\\{\}]
\PYG{n+nf}{alter\PYGZus{}group}\PYG{p}{(}\PYG{n+nv}{Groupname}\PYG{p}{,} \PYG{n+nv}{Options}\PYG{p}{)}
\end{Verbatim}


\subsection{del\_user}
\label{riak_core_security:del-user}
Deletes user and associated grants

\begin{Verbatim}[commandchars=\\\{\}]
\PYG{n+nf}{del\PYGZus{}user}\PYG{p}{(}\PYG{n+nv}{Username}\PYG{p}{)}
\end{Verbatim}


\subsection{del\_group}
\label{riak_core_security:del-group}
Deletes group and associated grants

\begin{Verbatim}[commandchars=\\\{\}]
\PYG{n+nf}{del\PYGZus{}group}\PYG{p}{(}\PYG{n+nv}{Groupname}\PYG{p}{)}
\end{Verbatim}


\subsection{add\_grant}
\label{riak_core_security:add-grant}
Add Grants to RoleList on Bucket, RoleList can be the atom \textbf{all} to asign
Grants to all roles in that Bucket.

Bucket can be a binary to assign to the whole bucket or \{binary(), binary()\},
to assign to a key in the bucket.

The call will merge previous grants with the new ones.

\begin{Verbatim}[commandchars=\\\{\}]
\PYG{n+nf}{add\PYGZus{}grant}\PYG{p}{(}\PYG{n+nv}{RoleList}\PYG{p}{,} \PYG{n+nv}{Bucket}\PYG{p}{,} \PYG{n+nv}{Grants}\PYG{p}{)}
\end{Verbatim}


\subsection{add\_revoke}
\label{riak_core_security:add-revoke}
Revoke Grants to RoleList on Bucket, RoleList can be the atom \textbf{all} to revoke
Grants to all roles in that Bucket.

\begin{Verbatim}[commandchars=\\\{\}]
\PYG{n+nf}{add\PYGZus{}revoke}\PYG{p}{(}\PYG{n+nv}{RoleList}\PYG{p}{,} \PYG{n+nv}{Bucket}\PYG{p}{,} \PYG{n+nv}{Revokes}\PYG{p}{)}
\end{Verbatim}


\subsection{add\_source}
\label{riak_core_security:add-source}
Users is a list of users or the atom \textbf{all} to apply to all users.
CIDR is a tuple with an IP address and a mask in bits.
Source is an atom:
\begin{itemize}
\item {} 
trust: no password required

\item {} 
password: password authentication

\item {} 
certificate: certificate authentication

\item {} 
Atom: Atom will be used as a custom authentication module, on auth Atom will
be looked up on the env key \{riak\_core, auth\_mods\} if found the returned
value will be used as a module to call
AuthMod:auth(Username, Password, UserData, SourceOptions)

\end{itemize}

Options are options for the source that will be passed during auth

\begin{Verbatim}[commandchars=\\\{\}]
\PYG{n+nf}{add\PYGZus{}source}\PYG{p}{(}\PYG{n+nv}{Users}\PYG{p}{,} \PYG{n+nv}{CIDR}\PYG{p}{,} \PYG{n+nv}{Source}\PYG{p}{,} \PYG{n+nv}{Options}\PYG{p}{)}
\end{Verbatim}

Example calls:

\begin{Verbatim}[commandchars=\\\{\}]
\PYG{n+nn}{riak\PYGZus{}core\PYGZus{}security}\PYG{p}{:}\PYG{n+nf}{add\PYGZus{}source}\PYG{p}{(}\PYG{n}{all}\PYG{p}{,} \PYG{p}{\PYGZob{}}\PYG{p}{\PYGZob{}}\PYG{l+m+mi}{127}\PYG{p}{,} \PYG{l+m+mi}{0}\PYG{p}{,} \PYG{l+m+mi}{0}\PYG{p}{,} \PYG{l+m+mi}{1}\PYG{p}{\PYGZcb{}}\PYG{p}{,} \PYG{l+m+mi}{32}\PYG{p}{\PYGZcb{}}\PYG{p}{,} \PYG{n}{trust}\PYG{p}{,} \PYG{p}{[}\PYG{p}{]}\PYG{p}{)}
\PYG{n+nn}{riak\PYGZus{}core\PYGZus{}security}\PYG{p}{:}\PYG{n+nf}{add\PYGZus{}source}\PYG{p}{(}\PYG{n}{all}\PYG{p}{,} \PYG{p}{\PYGZob{}}\PYG{p}{\PYGZob{}}\PYG{l+m+mi}{127}\PYG{p}{,} \PYG{l+m+mi}{0}\PYG{p}{,} \PYG{l+m+mi}{0}\PYG{p}{,} \PYG{l+m+mi}{1}\PYG{p}{\PYGZcb{}}\PYG{p}{,} \PYG{l+m+mi}{32}\PYG{p}{\PYGZcb{}}\PYG{p}{,} \PYG{n}{password}\PYG{p}{,} \PYG{p}{[}\PYG{p}{]}\PYG{p}{)}
\end{Verbatim}


\subsection{del\_source}
\label{riak_core_security:del-source}
Delete source identified by CIDR for Users, Users can be the atom \textbf{all} to
remove the source from all users. This won't apply to sources added for each
users, only if the source was added explicitly for the \textbf{all} atom.

\begin{Verbatim}[commandchars=\\\{\}]
\PYG{n+nf}{del\PYGZus{}source}\PYG{p}{(}\PYG{n+nv}{Users}\PYG{p}{,} \PYG{n+nv}{CIDR}\PYG{p}{)}
\end{Verbatim}


\subsection{is\_enabled}
\label{riak_core_security:is-enabled}
Returns \textbf{true} if riak\_core\_security is enabled, \textbf{false} otherwise.

\begin{Verbatim}[commandchars=\\\{\}]
\PYG{n+nf}{is\PYGZus{}enabled}\PYG{p}{(}\PYG{p}{)}
\end{Verbatim}


\subsection{enable}
\label{riak_core_security:enable}
Enables riak\_core\_security

\begin{Verbatim}[commandchars=\\\{\}]
\PYG{n+nf}{enable}\PYG{p}{(}\PYG{p}{)}
\end{Verbatim}


\subsection{disable}
\label{riak_core_security:disable}
Disabled riak\_core\_security

\begin{Verbatim}[commandchars=\\\{\}]
\PYG{n+nf}{disable}\PYG{p}{(}\PYG{p}{)}
\end{Verbatim}


\subsection{status}
\label{riak_core_security:status}
Returns an atom representing the  status of riak\_core\_security:
\begin{itemize}
\item {} 
enabled

\item {} 
enabled\_but\_no\_capability

\item {} 
disabled

\end{itemize}

\begin{Verbatim}[commandchars=\\\{\}]
\PYG{n+nf}{status}\PYG{p}{(}\PYG{p}{)}
\end{Verbatim}


\section{Playing in the REPL}
\label{riak_core_security:playing-in-the-repl}
First we will need to uncomment the permissions for our app in config/advanced.config

TODO: commit hash here

Then we build again and run it:

\begin{Verbatim}[commandchars=\\\{\}]
rebar3 release
rebar3 run
\end{Verbatim}

First let's setup some variables

\begin{Verbatim}[commandchars=\\\{\}]
\PYG{p}{(}\PYG{n}{tanodb}\PYG{p}{@}\PYG{l+m+mi}{127}\PYG{p}{.}\PYG{l+m+mi}{0}\PYG{p}{.}\PYG{l+m+mi}{0}\PYG{p}{.}\PYG{l+m+mi}{1}\PYG{p}{)}\PYG{l+m+mi}{1}\PYG{o}{\PYGZgt{}} \PYG{n+nv}{User1} \PYG{o}{=} \PYG{o}{\PYGZlt{}}\PYG{o}{\PYGZlt{}}\PYG{l+s}{\PYGZdq{}}\PYG{l+s}{sandy}\PYG{l+s}{\PYGZdq{}}\PYG{o}{\PYGZgt{}}\PYG{o}{\PYGZgt{}}\PYG{p}{.}
\PYG{o}{\PYGZlt{}}\PYG{o}{\PYGZlt{}}\PYG{l+s}{\PYGZdq{}}\PYG{l+s}{sandy}\PYG{l+s}{\PYGZdq{}}\PYG{o}{\PYGZgt{}}\PYG{o}{\PYGZgt{}}

\PYG{p}{(}\PYG{n}{tanodb}\PYG{p}{@}\PYG{l+m+mi}{127}\PYG{p}{.}\PYG{l+m+mi}{0}\PYG{p}{.}\PYG{l+m+mi}{0}\PYG{p}{.}\PYG{l+m+mi}{1}\PYG{p}{)}\PYG{l+m+mi}{2}\PYG{o}{\PYGZgt{}} \PYG{n+nv}{Pass1} \PYG{o}{=} \PYG{o}{\PYGZlt{}}\PYG{o}{\PYGZlt{}}\PYG{l+s}{\PYGZdq{}}\PYG{l+s}{secret}\PYG{l+s}{\PYGZdq{}}\PYG{o}{\PYGZgt{}}\PYG{o}{\PYGZgt{}}\PYG{p}{.}
\PYG{o}{\PYGZlt{}}\PYG{o}{\PYGZlt{}}\PYG{l+s}{\PYGZdq{}}\PYG{l+s}{secret}\PYG{l+s}{\PYGZdq{}}\PYG{o}{\PYGZgt{}}\PYG{o}{\PYGZgt{}}

\PYG{p}{(}\PYG{n}{tanodb}\PYG{p}{@}\PYG{l+m+mi}{127}\PYG{p}{.}\PYG{l+m+mi}{0}\PYG{p}{.}\PYG{l+m+mi}{0}\PYG{p}{.}\PYG{l+m+mi}{1}\PYG{p}{)}\PYG{l+m+mi}{3}\PYG{o}{\PYGZgt{}} \PYG{n+nv}{ConnInfo} \PYG{o}{=} \PYG{p}{[}\PYG{p}{\PYGZob{}}\PYG{n}{ip}\PYG{p}{,} \PYG{p}{\PYGZob{}}\PYG{l+m+mi}{127}\PYG{p}{,} \PYG{l+m+mi}{0}\PYG{p}{,} \PYG{l+m+mi}{0}\PYG{p}{,} \PYG{l+m+mi}{1}\PYG{p}{\PYGZcb{}}\PYG{p}{\PYGZcb{}}\PYG{p}{]}\PYG{p}{.}
\PYG{p}{[}\PYG{p}{\PYGZob{}}\PYG{n}{ip}\PYG{p}{,}\PYG{p}{\PYGZob{}}\PYG{l+m+mi}{127}\PYG{p}{,}\PYG{l+m+mi}{0}\PYG{p}{,}\PYG{l+m+mi}{0}\PYG{p}{,}\PYG{l+m+mi}{1}\PYG{p}{\PYGZcb{}}\PYG{p}{\PYGZcb{}}\PYG{p}{]}

\PYG{p}{(}\PYG{n}{tanodb}\PYG{p}{@}\PYG{l+m+mi}{127}\PYG{p}{.}\PYG{l+m+mi}{0}\PYG{p}{.}\PYG{l+m+mi}{0}\PYG{p}{.}\PYG{l+m+mi}{1}\PYG{p}{)}\PYG{l+m+mi}{4}\PYG{o}{\PYGZgt{}} \PYG{n+nv}{Source1} \PYG{o}{=} \PYG{p}{\PYGZob{}}\PYG{p}{\PYGZob{}}\PYG{l+m+mi}{127}\PYG{p}{,} \PYG{l+m+mi}{0}\PYG{p}{,} \PYG{l+m+mi}{0}\PYG{p}{,} \PYG{l+m+mi}{1}\PYG{p}{\PYGZcb{}}\PYG{p}{,} \PYG{l+m+mi}{32}\PYG{p}{\PYGZcb{}}\PYG{p}{.}
\PYG{p}{\PYGZob{}}\PYG{p}{\PYGZob{}}\PYG{l+m+mi}{127}\PYG{p}{,}\PYG{l+m+mi}{0}\PYG{p}{,}\PYG{l+m+mi}{0}\PYG{p}{,}\PYG{l+m+mi}{1}\PYG{p}{\PYGZcb{}}\PYG{p}{,}\PYG{l+m+mi}{32}\PYG{p}{\PYGZcb{}}

\PYG{p}{(}\PYG{n}{tanodb}\PYG{p}{@}\PYG{l+m+mi}{127}\PYG{p}{.}\PYG{l+m+mi}{0}\PYG{p}{.}\PYG{l+m+mi}{0}\PYG{p}{.}\PYG{l+m+mi}{1}\PYG{p}{)}\PYG{l+m+mi}{5}\PYG{o}{\PYGZgt{}} \PYG{n+nv}{Bucket1} \PYG{o}{=} \PYG{o}{\PYGZlt{}}\PYG{o}{\PYGZlt{}}\PYG{l+s}{\PYGZdq{}}\PYG{l+s}{bucket\PYGZus{}sandy}\PYG{l+s}{\PYGZdq{}}\PYG{o}{\PYGZgt{}}\PYG{o}{\PYGZgt{}}\PYG{p}{.}
\PYG{o}{\PYGZlt{}}\PYG{o}{\PYGZlt{}}\PYG{l+s}{\PYGZdq{}}\PYG{l+s}{bucket\PYGZus{}sandy}\PYG{l+s}{\PYGZdq{}}\PYG{o}{\PYGZgt{}}\PYG{o}{\PYGZgt{}}

\PYG{p}{(}\PYG{n}{tanodb}\PYG{p}{@}\PYG{l+m+mi}{127}\PYG{p}{.}\PYG{l+m+mi}{0}\PYG{p}{.}\PYG{l+m+mi}{0}\PYG{p}{.}\PYG{l+m+mi}{1}\PYG{p}{)}\PYG{l+m+mi}{6}\PYG{o}{\PYGZgt{}} \PYG{n+nv}{PermGet} \PYG{o}{=} \PYG{l+s}{\PYGZdq{}}\PYG{l+s}{tanodb.get}\PYG{l+s}{\PYGZdq{}}\PYG{p}{.}
\PYG{l+s}{\PYGZdq{}}\PYG{l+s}{tanodb.get}\PYG{l+s}{\PYGZdq{}}

\PYG{p}{(}\PYG{n}{tanodb}\PYG{p}{@}\PYG{l+m+mi}{127}\PYG{p}{.}\PYG{l+m+mi}{0}\PYG{p}{.}\PYG{l+m+mi}{0}\PYG{p}{.}\PYG{l+m+mi}{1}\PYG{p}{)}\PYG{l+m+mi}{7}\PYG{o}{\PYGZgt{}} \PYG{n+nv}{PermPut} \PYG{o}{=} \PYG{l+s}{\PYGZdq{}}\PYG{l+s}{tanodb.put}\PYG{l+s}{\PYGZdq{}}\PYG{p}{.}
\PYG{l+s}{\PYGZdq{}}\PYG{l+s}{tanodb.put}\PYG{l+s}{\PYGZdq{}}

\PYG{p}{(}\PYG{n}{tanodb}\PYG{p}{@}\PYG{l+m+mi}{127}\PYG{p}{.}\PYG{l+m+mi}{0}\PYG{p}{.}\PYG{l+m+mi}{0}\PYG{p}{.}\PYG{l+m+mi}{1}\PYG{p}{)}\PYG{l+m+mi}{8}\PYG{o}{\PYGZgt{}} \PYG{n+nv}{PermList} \PYG{o}{=} \PYG{l+s}{\PYGZdq{}}\PYG{l+s}{tanodb.list}\PYG{l+s}{\PYGZdq{}}\PYG{p}{.}
\PYG{l+s}{\PYGZdq{}}\PYG{l+s}{tanodb.list}\PYG{l+s}{\PYGZdq{}}

\PYG{p}{(}\PYG{n}{tanodb}\PYG{p}{@}\PYG{l+m+mi}{127}\PYG{p}{.}\PYG{l+m+mi}{0}\PYG{p}{.}\PYG{l+m+mi}{0}\PYG{p}{.}\PYG{l+m+mi}{1}\PYG{p}{)}\PYG{l+m+mi}{9}\PYG{o}{\PYGZgt{}} \PYG{n+nv}{GroupWriter} \PYG{o}{=} \PYG{o}{\PYGZlt{}}\PYG{o}{\PYGZlt{}}\PYG{l+s}{\PYGZdq{}}\PYG{l+s}{writers}\PYG{l+s}{\PYGZdq{}}\PYG{o}{\PYGZgt{}}\PYG{o}{\PYGZgt{}}\PYG{p}{.}
\PYG{o}{\PYGZlt{}}\PYG{o}{\PYGZlt{}}\PYG{l+s}{\PYGZdq{}}\PYG{l+s}{writers}\PYG{l+s}{\PYGZdq{}}\PYG{o}{\PYGZgt{}}\PYG{o}{\PYGZgt{}}

\PYG{p}{(}\PYG{n}{tanodb}\PYG{p}{@}\PYG{l+m+mi}{127}\PYG{p}{.}\PYG{l+m+mi}{0}\PYG{p}{.}\PYG{l+m+mi}{0}\PYG{p}{.}\PYG{l+m+mi}{1}\PYG{p}{)}\PYG{l+m+mi}{10}\PYG{o}{\PYGZgt{}} \PYG{n+nv}{GroupReader} \PYG{o}{=} \PYG{o}{\PYGZlt{}}\PYG{o}{\PYGZlt{}}\PYG{l+s}{\PYGZdq{}}\PYG{l+s}{readers}\PYG{l+s}{\PYGZdq{}}\PYG{o}{\PYGZgt{}}\PYG{o}{\PYGZgt{}}\PYG{p}{.}
\PYG{o}{\PYGZlt{}}\PYG{o}{\PYGZlt{}}\PYG{l+s}{\PYGZdq{}}\PYG{l+s}{readers}\PYG{l+s}{\PYGZdq{}}\PYG{o}{\PYGZgt{}}\PYG{o}{\PYGZgt{}}
\end{Verbatim}

We didn't add the user yet, so the following should fail

\begin{Verbatim}[commandchars=\\\{\}]
\PYG{p}{(}\PYG{n}{tanodb}\PYG{p}{@}\PYG{l+m+mi}{127}\PYG{p}{.}\PYG{l+m+mi}{0}\PYG{p}{.}\PYG{l+m+mi}{0}\PYG{p}{.}\PYG{l+m+mi}{1}\PYG{p}{)}\PYG{l+m+mi}{11}\PYG{o}{\PYGZgt{}} \PYG{n+nn}{riak\PYGZus{}core\PYGZus{}security}\PYG{p}{:}\PYG{n+nf}{authenticate}\PYG{p}{(}\PYG{n+nv}{User1}\PYG{p}{,} \PYG{n+nv}{Pass1}\PYG{p}{,} \PYG{n+nv}{ConnInfo}\PYG{p}{)}\PYG{p}{.}
\PYG{p}{\PYGZob{}}\PYG{n}{error}\PYG{p}{,}\PYG{n}{unknown\PYGZus{}user}\PYG{p}{\PYGZcb{}}
\end{Verbatim}

Let's add the user

\begin{Verbatim}[commandchars=\\\{\}]
\PYG{p}{(}\PYG{n}{tanodb}\PYG{p}{@}\PYG{l+m+mi}{127}\PYG{p}{.}\PYG{l+m+mi}{0}\PYG{p}{.}\PYG{l+m+mi}{0}\PYG{p}{.}\PYG{l+m+mi}{1}\PYG{p}{)}\PYG{l+m+mi}{12}\PYG{o}{\PYGZgt{}} \PYG{n+nn}{riak\PYGZus{}core\PYGZus{}security}\PYG{p}{:}\PYG{n+nf}{add\PYGZus{}user}\PYG{p}{(}\PYG{n+nv}{User1}\PYG{p}{,} \PYG{p}{[}\PYG{p}{\PYGZob{}}\PYG{l+s}{\PYGZdq{}}\PYG{l+s}{password}\PYG{l+s}{\PYGZdq{}}\PYG{p}{,} \PYG{n+nb}{binary\PYGZus{}to\PYGZus{}list}\PYG{p}{(}\PYG{n+nv}{Pass1}\PYG{p}{)}\PYG{p}{\PYGZcb{}}\PYG{p}{]}\PYG{p}{)}\PYG{p}{.}
\PYG{n}{ok}
\end{Verbatim}

Adding it twice should fail

\begin{Verbatim}[commandchars=\\\{\}]
\PYG{p}{(}\PYG{n}{tanodb}\PYG{p}{@}\PYG{l+m+mi}{127}\PYG{p}{.}\PYG{l+m+mi}{0}\PYG{p}{.}\PYG{l+m+mi}{0}\PYG{p}{.}\PYG{l+m+mi}{1}\PYG{p}{)}\PYG{l+m+mi}{13}\PYG{o}{\PYGZgt{}} \PYG{n+nn}{riak\PYGZus{}core\PYGZus{}security}\PYG{p}{:}\PYG{n+nf}{add\PYGZus{}user}\PYG{p}{(}\PYG{n+nv}{User1}\PYG{p}{,} \PYG{p}{[}\PYG{p}{\PYGZob{}}\PYG{l+s}{\PYGZdq{}}\PYG{l+s}{password}\PYG{l+s}{\PYGZdq{}}\PYG{p}{,} \PYG{n+nb}{binary\PYGZus{}to\PYGZus{}list}\PYG{p}{(}\PYG{n+nv}{Pass1}\PYG{p}{)}\PYG{p}{\PYGZcb{}}\PYG{p}{]}\PYG{p}{)}\PYG{p}{.}
\PYG{p}{\PYGZob{}}\PYG{n}{error}\PYG{p}{,}\PYG{n}{role\PYGZus{}exists}\PYG{p}{\PYGZcb{}}
\end{Verbatim}

We didn't add the source for the user so the following should fail

\begin{Verbatim}[commandchars=\\\{\}]
\PYG{p}{(}\PYG{n}{tanodb}\PYG{p}{@}\PYG{l+m+mi}{127}\PYG{p}{.}\PYG{l+m+mi}{0}\PYG{p}{.}\PYG{l+m+mi}{0}\PYG{p}{.}\PYG{l+m+mi}{1}\PYG{p}{)}\PYG{l+m+mi}{14}\PYG{o}{\PYGZgt{}} \PYG{n+nn}{riak\PYGZus{}core\PYGZus{}security}\PYG{p}{:}\PYG{n+nf}{authenticate}\PYG{p}{(}\PYG{n+nv}{User1}\PYG{p}{,} \PYG{n+nv}{Pass1}\PYG{p}{,} \PYG{n+nv}{ConnInfo}\PYG{p}{)}\PYG{p}{.}
\PYG{p}{\PYGZob{}}\PYG{n}{error}\PYG{p}{,}\PYG{n}{no\PYGZus{}matching\PYGZus{}sources}\PYG{p}{\PYGZcb{}}
\end{Verbatim}

Add a local source that requires password for all users

\begin{Verbatim}[commandchars=\\\{\}]
\PYG{p}{(}\PYG{n}{tanodb}\PYG{p}{@}\PYG{l+m+mi}{127}\PYG{p}{.}\PYG{l+m+mi}{0}\PYG{p}{.}\PYG{l+m+mi}{0}\PYG{p}{.}\PYG{l+m+mi}{1}\PYG{p}{)}\PYG{l+m+mi}{15}\PYG{o}{\PYGZgt{}} \PYG{n+nn}{riak\PYGZus{}core\PYGZus{}security}\PYG{p}{:}\PYG{n+nf}{add\PYGZus{}source}\PYG{p}{(}\PYG{n}{all}\PYG{p}{,} \PYG{n+nv}{Source1}\PYG{p}{,} \PYG{n}{password}\PYG{p}{,} \PYG{p}{[}\PYG{p}{]}\PYG{p}{)}\PYG{p}{.}
\PYG{n}{ok}
\end{Verbatim}

Now it should work

\begin{Verbatim}[commandchars=\\\{\}]
\PYG{p}{(}\PYG{n}{tanodb}\PYG{p}{@}\PYG{l+m+mi}{127}\PYG{p}{.}\PYG{l+m+mi}{0}\PYG{p}{.}\PYG{l+m+mi}{0}\PYG{p}{.}\PYG{l+m+mi}{1}\PYG{p}{)}\PYG{l+m+mi}{16}\PYG{o}{\PYGZgt{}} \PYG{p}{\PYGZob{}}\PYG{n}{ok}\PYG{p}{,} \PYG{n+nv}{Ctx1}\PYG{p}{\PYGZcb{}} \PYG{o}{=} \PYG{n+nn}{riak\PYGZus{}core\PYGZus{}security}\PYG{p}{:}\PYG{n+nf}{authenticate}\PYG{p}{(}\PYG{n+nv}{User1}\PYG{p}{,} \PYG{n+nv}{Pass1}\PYG{p}{,} \PYG{n+nv}{ConnInfo}\PYG{p}{)}\PYG{p}{.}
\PYG{p}{\PYGZob{}}\PYG{n}{ok}\PYG{p}{,}\PYG{p}{\PYGZob{}}\PYG{n}{context}\PYG{p}{,}\PYG{o}{\PYGZlt{}}\PYG{o}{\PYGZlt{}}\PYG{l+s}{\PYGZdq{}}\PYG{l+s}{sandy}\PYG{l+s}{\PYGZdq{}}\PYG{o}{\PYGZgt{}}\PYG{o}{\PYGZgt{}}\PYG{p}{,}\PYG{p}{[}\PYG{p}{]}\PYG{p}{,}\PYG{p}{\PYGZob{}}\PYG{l+m+mi}{1444}\PYG{p}{,}\PYG{l+m+mi}{659568}\PYG{p}{,}\PYG{l+m+mi}{765253}\PYG{p}{\PYGZcb{}}\PYG{p}{\PYGZcb{}}\PYG{p}{\PYGZcb{}}
\end{Verbatim}

Checking permissions should fail, since we didn't granted any permissions yet

\begin{Verbatim}[commandchars=\\\{\}]
\PYG{p}{(}\PYG{n}{tanodb}\PYG{p}{@}\PYG{l+m+mi}{127}\PYG{p}{.}\PYG{l+m+mi}{0}\PYG{p}{.}\PYG{l+m+mi}{0}\PYG{p}{.}\PYG{l+m+mi}{1}\PYG{p}{)}\PYG{l+m+mi}{17}\PYG{o}{\PYGZgt{}} \PYG{n+nn}{riak\PYGZus{}core\PYGZus{}security}\PYG{p}{:}\PYG{n+nf}{check\PYGZus{}permission}\PYG{p}{(}\PYG{p}{\PYGZob{}}\PYG{n+nv}{PermGet}\PYG{p}{,} \PYG{n+nv}{Bucket1}\PYG{p}{\PYGZcb{}}\PYG{p}{,} \PYG{n+nv}{Ctx1}\PYG{p}{)}\PYG{p}{.}
\PYG{p}{\PYGZob{}}\PYG{n}{false}\PYG{p}{,}\PYG{o}{\PYGZlt{}}\PYG{o}{\PYGZlt{}}\PYG{l+s}{\PYGZdq{}}\PYG{l+s}{Permission denied: User \PYGZsq{}sandy\PYGZsq{} does not have \PYGZsq{}tanodb.get\PYGZsq{} on bucket\PYGZus{}sandy}\PYG{l+s}{\PYGZdq{}}\PYG{o}{\PYGZgt{}}\PYG{o}{\PYGZgt{}}\PYG{p}{,}
   \PYG{p}{\PYGZob{}}\PYG{n}{context}\PYG{p}{,}\PYG{o}{\PYGZlt{}}\PYG{o}{\PYGZlt{}}\PYG{l+s}{\PYGZdq{}}\PYG{l+s}{sandy}\PYG{l+s}{\PYGZdq{}}\PYG{o}{\PYGZgt{}}\PYG{o}{\PYGZgt{}}\PYG{p}{,}\PYG{p}{[}\PYG{p}{]}\PYG{p}{,}\PYG{p}{\PYGZob{}}\PYG{l+m+mi}{1444}\PYG{p}{,}\PYG{l+m+mi}{659568}\PYG{p}{,}\PYG{l+m+mi}{765253}\PYG{p}{\PYGZcb{}}\PYG{p}{\PYGZcb{}}\PYG{p}{\PYGZcb{}}
\end{Verbatim}

Let's grant PermGet to User1

\begin{Verbatim}[commandchars=\\\{\}]
\PYG{p}{(}\PYG{n}{tanodb}\PYG{p}{@}\PYG{l+m+mi}{127}\PYG{p}{.}\PYG{l+m+mi}{0}\PYG{p}{.}\PYG{l+m+mi}{0}\PYG{p}{.}\PYG{l+m+mi}{1}\PYG{p}{)}\PYG{l+m+mi}{18}\PYG{o}{\PYGZgt{}} \PYG{n+nn}{riak\PYGZus{}core\PYGZus{}security}\PYG{p}{:}\PYG{n+nf}{add\PYGZus{}grant}\PYG{p}{(}\PYG{p}{[}\PYG{n+nv}{User1}\PYG{p}{]}\PYG{p}{,} \PYG{n+nv}{Bucket1}\PYG{p}{,} \PYG{p}{[}\PYG{n+nv}{PermGet}\PYG{p}{]}\PYG{p}{)}\PYG{p}{.}
\PYG{n}{ok}
\end{Verbatim}

And try again

\begin{Verbatim}[commandchars=\\\{\}]
\PYG{p}{(}\PYG{n}{tanodb}\PYG{p}{@}\PYG{l+m+mi}{127}\PYG{p}{.}\PYG{l+m+mi}{0}\PYG{p}{.}\PYG{l+m+mi}{0}\PYG{p}{.}\PYG{l+m+mi}{1}\PYG{p}{)}\PYG{l+m+mi}{19}\PYG{o}{\PYGZgt{}} \PYG{n+nn}{riak\PYGZus{}core\PYGZus{}security}\PYG{p}{:}\PYG{n+nf}{check\PYGZus{}permission}\PYG{p}{(}\PYG{p}{\PYGZob{}}\PYG{n+nv}{PermGet}\PYG{p}{,} \PYG{n+nv}{Bucket1}\PYG{p}{\PYGZcb{}}\PYG{p}{,} \PYG{n+nv}{Ctx1}\PYG{p}{)}\PYG{p}{.}
\PYG{p}{\PYGZob{}}\PYG{n}{true}\PYG{p}{,}\PYG{p}{\PYGZob{}}\PYG{n}{context}\PYG{p}{,}\PYG{o}{\PYGZlt{}}\PYG{o}{\PYGZlt{}}\PYG{l+s}{\PYGZdq{}}\PYG{l+s}{sandy}\PYG{l+s}{\PYGZdq{}}\PYG{o}{\PYGZgt{}}\PYG{o}{\PYGZgt{}}\PYG{p}{,}
           \PYG{p}{[}\PYG{p}{\PYGZob{}}\PYG{o}{\PYGZlt{}}\PYG{o}{\PYGZlt{}}\PYG{l+s}{\PYGZdq{}}\PYG{l+s}{bucket\PYGZus{}sandy}\PYG{l+s}{\PYGZdq{}}\PYG{o}{\PYGZgt{}}\PYG{o}{\PYGZgt{}}\PYG{p}{,}\PYG{p}{[}\PYG{l+s}{\PYGZdq{}}\PYG{l+s}{tanodb.get}\PYG{l+s}{\PYGZdq{}}\PYG{p}{]}\PYG{p}{\PYGZcb{}}\PYG{p}{]}\PYG{p}{,}
           \PYG{p}{\PYGZob{}}\PYG{l+m+mi}{1444}\PYG{p}{,}\PYG{l+m+mi}{659568}\PYG{p}{,}\PYG{l+m+mi}{779759}\PYG{p}{\PYGZcb{}}\PYG{p}{\PYGZcb{}}\PYG{p}{\PYGZcb{}}
\end{Verbatim}

Create some groups, each group belongs to the previous one

\begin{Verbatim}[commandchars=\\\{\}]
\PYG{p}{(}\PYG{n}{tanodb}\PYG{p}{@}\PYG{l+m+mi}{127}\PYG{p}{.}\PYG{l+m+mi}{0}\PYG{p}{.}\PYG{l+m+mi}{0}\PYG{p}{.}\PYG{l+m+mi}{1}\PYG{p}{)}\PYG{l+m+mi}{20}\PYG{o}{\PYGZgt{}} \PYG{n+nn}{riak\PYGZus{}core\PYGZus{}security}\PYG{p}{:}\PYG{n+nf}{add\PYGZus{}group}\PYG{p}{(}\PYG{n+nv}{GroupReader}\PYG{p}{,} \PYG{p}{[}\PYG{p}{]}\PYG{p}{)}\PYG{p}{.}
\PYG{n+nf}{ok}

\PYG{p}{(}\PYG{n}{tanodb}\PYG{p}{@}\PYG{l+m+mi}{127}\PYG{p}{.}\PYG{l+m+mi}{0}\PYG{p}{.}\PYG{l+m+mi}{0}\PYG{p}{.}\PYG{l+m+mi}{1}\PYG{p}{)}\PYG{l+m+mi}{21}\PYG{o}{\PYGZgt{}} \PYG{n+nn}{riak\PYGZus{}core\PYGZus{}security}\PYG{p}{:}\PYG{n+nf}{add\PYGZus{}group}\PYG{p}{(}\PYG{n+nv}{GroupWriter}\PYG{p}{,} \PYG{p}{[}\PYG{p}{\PYGZob{}}\PYG{l+s}{\PYGZdq{}}\PYG{l+s}{groups}\PYG{l+s}{\PYGZdq{}}\PYG{p}{,} \PYG{p}{[}\PYG{n+nv}{GroupReader}\PYG{p}{]}\PYG{p}{\PYGZcb{}}\PYG{p}{]}\PYG{p}{)}\PYG{p}{.}
\PYG{n}{ok}
\end{Verbatim}

Let's grant permissions to each group

\begin{Verbatim}[commandchars=\\\{\}]
\PYG{p}{(}\PYG{n}{tanodb}\PYG{p}{@}\PYG{l+m+mi}{127}\PYG{p}{.}\PYG{l+m+mi}{0}\PYG{p}{.}\PYG{l+m+mi}{0}\PYG{p}{.}\PYG{l+m+mi}{1}\PYG{p}{)}\PYG{l+m+mi}{22}\PYG{o}{\PYGZgt{}} \PYG{n+nn}{riak\PYGZus{}core\PYGZus{}security}\PYG{p}{:}\PYG{n+nf}{add\PYGZus{}grant}\PYG{p}{(}\PYG{p}{[}\PYG{n+nv}{GroupReader}\PYG{p}{]}\PYG{p}{,} \PYG{n+nv}{Bucket1}\PYG{p}{,} \PYG{p}{[}\PYG{n+nv}{PermGet}\PYG{p}{]}\PYG{p}{)}\PYG{p}{.}
\PYG{n+nf}{ok}

\PYG{p}{(}\PYG{n}{tanodb}\PYG{p}{@}\PYG{l+m+mi}{127}\PYG{p}{.}\PYG{l+m+mi}{0}\PYG{p}{.}\PYG{l+m+mi}{0}\PYG{p}{.}\PYG{l+m+mi}{1}\PYG{p}{)}\PYG{l+m+mi}{23}\PYG{o}{\PYGZgt{}} \PYG{n+nn}{riak\PYGZus{}core\PYGZus{}security}\PYG{p}{:}\PYG{n+nf}{add\PYGZus{}grant}\PYG{p}{(}\PYG{p}{[}\PYG{n+nv}{GroupWriter}\PYG{p}{]}\PYG{p}{,} \PYG{n+nv}{Bucket1}\PYG{p}{,} \PYG{p}{[}\PYG{n+nv}{PermPut}\PYG{p}{]}\PYG{p}{)}\PYG{p}{.}
\PYG{n}{ok}
\end{Verbatim}

Now let's join User1 to some groups and try permissions

\begin{Verbatim}[commandchars=\\\{\}]
\PYG{p}{(}\PYG{n}{tanodb}\PYG{p}{@}\PYG{l+m+mi}{127}\PYG{p}{.}\PYG{l+m+mi}{0}\PYG{p}{.}\PYG{l+m+mi}{0}\PYG{p}{.}\PYG{l+m+mi}{1}\PYG{p}{)}\PYG{l+m+mi}{24}\PYG{o}{\PYGZgt{}} \PYG{n+nn}{riak\PYGZus{}core\PYGZus{}security}\PYG{p}{:}\PYG{n+nf}{alter\PYGZus{}user}\PYG{p}{(}\PYG{n+nv}{User1}\PYG{p}{,} \PYG{p}{[}\PYG{p}{\PYGZob{}}\PYG{l+s}{\PYGZdq{}}\PYG{l+s}{groups}\PYG{l+s}{\PYGZdq{}}\PYG{p}{,} \PYG{p}{[}\PYG{n+nv}{GroupReader}\PYG{p}{]}\PYG{p}{\PYGZcb{}}\PYG{p}{]}\PYG{p}{)}\PYG{p}{.}
\PYG{n}{ok}
\end{Verbatim}

We can see User1 is a member of the group

\begin{Verbatim}[commandchars=\\\{\}]
\PYG{p}{(}\PYG{n}{tanodb}\PYG{p}{@}\PYG{l+m+mi}{127}\PYG{p}{.}\PYG{l+m+mi}{0}\PYG{p}{.}\PYG{l+m+mi}{0}\PYG{p}{.}\PYG{l+m+mi}{1}\PYG{p}{)}\PYG{l+m+mi}{25}\PYG{o}{\PYGZgt{}} \PYG{n+nn}{riak\PYGZus{}core\PYGZus{}security}\PYG{p}{:}\PYG{n+nf}{print\PYGZus{}user}\PYG{p}{(}\PYG{n+nv}{User1}\PYG{p}{)}\PYG{p}{.}
\PYG{n}{ok}
\end{Verbatim}

\begin{Verbatim}[commandchars=\\\{\}]
+\PYGZhy{}\PYGZhy{}\PYGZhy{}\PYGZhy{}\PYGZhy{}\PYGZhy{}\PYGZhy{}\PYGZhy{}\PYGZhy{}\PYGZhy{}+\PYGZhy{}\PYGZhy{}\PYGZhy{}\PYGZhy{}\PYGZhy{}\PYGZhy{}\PYGZhy{}\PYGZhy{}\PYGZhy{}\PYGZhy{}\PYGZhy{}\PYGZhy{}\PYGZhy{}\PYGZhy{}\PYGZhy{}+\PYGZhy{}\PYGZhy{}\PYGZhy{}\PYGZhy{}\PYGZhy{}\PYGZhy{}\PYGZhy{}\PYGZhy{}\PYGZhy{}\PYGZhy{}\PYGZhy{}\PYGZhy{}\PYGZhy{}\PYGZhy{}\PYGZhy{}\PYGZhy{}\PYGZhy{}\PYGZhy{}\PYGZhy{}\PYGZhy{}\PYGZhy{}\PYGZhy{}\PYGZhy{}\PYGZhy{}\PYGZhy{}\PYGZhy{}\PYGZhy{}\PYGZhy{}\PYGZhy{}\PYGZhy{}\PYGZhy{}\PYGZhy{}\PYGZhy{}\PYGZhy{}\PYGZhy{}\PYGZhy{}\PYGZhy{}\PYGZhy{}\PYGZhy{}\PYGZhy{}+\PYGZhy{}\PYGZhy{}\PYGZhy{}\PYGZhy{}\PYGZhy{}\PYGZhy{}\PYGZhy{}\PYGZhy{}\PYGZhy{}\PYGZhy{}\PYGZhy{}\PYGZhy{}\PYGZhy{}\PYGZhy{}\PYGZhy{}\PYGZhy{}\PYGZhy{}\PYGZhy{}\PYGZhy{}\PYGZhy{}\PYGZhy{}\PYGZhy{}\PYGZhy{}\PYGZhy{}\PYGZhy{}\PYGZhy{}\PYGZhy{}\PYGZhy{}\PYGZhy{}\PYGZhy{}+

\textbar{} username \textbar{}   member of   \textbar{}                password                \textbar{}           options            \textbar{}
+\PYGZhy{}\PYGZhy{}\PYGZhy{}\PYGZhy{}\PYGZhy{}\PYGZhy{}\PYGZhy{}\PYGZhy{}\PYGZhy{}\PYGZhy{}+\PYGZhy{}\PYGZhy{}\PYGZhy{}\PYGZhy{}\PYGZhy{}\PYGZhy{}\PYGZhy{}\PYGZhy{}\PYGZhy{}\PYGZhy{}\PYGZhy{}\PYGZhy{}\PYGZhy{}\PYGZhy{}\PYGZhy{}+\PYGZhy{}\PYGZhy{}\PYGZhy{}\PYGZhy{}\PYGZhy{}\PYGZhy{}\PYGZhy{}\PYGZhy{}\PYGZhy{}\PYGZhy{}\PYGZhy{}\PYGZhy{}\PYGZhy{}\PYGZhy{}\PYGZhy{}\PYGZhy{}\PYGZhy{}\PYGZhy{}\PYGZhy{}\PYGZhy{}\PYGZhy{}\PYGZhy{}\PYGZhy{}\PYGZhy{}\PYGZhy{}\PYGZhy{}\PYGZhy{}\PYGZhy{}\PYGZhy{}\PYGZhy{}\PYGZhy{}\PYGZhy{}\PYGZhy{}\PYGZhy{}\PYGZhy{}\PYGZhy{}\PYGZhy{}\PYGZhy{}\PYGZhy{}\PYGZhy{}+\PYGZhy{}\PYGZhy{}\PYGZhy{}\PYGZhy{}\PYGZhy{}\PYGZhy{}\PYGZhy{}\PYGZhy{}\PYGZhy{}\PYGZhy{}\PYGZhy{}\PYGZhy{}\PYGZhy{}\PYGZhy{}\PYGZhy{}\PYGZhy{}\PYGZhy{}\PYGZhy{}\PYGZhy{}\PYGZhy{}\PYGZhy{}\PYGZhy{}\PYGZhy{}\PYGZhy{}\PYGZhy{}\PYGZhy{}\PYGZhy{}\PYGZhy{}\PYGZhy{}\PYGZhy{}+
\textbar{}  sandy   \textbar{}    readers    \textbar{}9c8984b176e07eb7ba9ff1e3ada5a43ecb8a812e\textbar{}              []              \textbar{}
+\PYGZhy{}\PYGZhy{}\PYGZhy{}\PYGZhy{}\PYGZhy{}\PYGZhy{}\PYGZhy{}\PYGZhy{}\PYGZhy{}\PYGZhy{}+\PYGZhy{}\PYGZhy{}\PYGZhy{}\PYGZhy{}\PYGZhy{}\PYGZhy{}\PYGZhy{}\PYGZhy{}\PYGZhy{}\PYGZhy{}\PYGZhy{}\PYGZhy{}\PYGZhy{}\PYGZhy{}\PYGZhy{}+\PYGZhy{}\PYGZhy{}\PYGZhy{}\PYGZhy{}\PYGZhy{}\PYGZhy{}\PYGZhy{}\PYGZhy{}\PYGZhy{}\PYGZhy{}\PYGZhy{}\PYGZhy{}\PYGZhy{}\PYGZhy{}\PYGZhy{}\PYGZhy{}\PYGZhy{}\PYGZhy{}\PYGZhy{}\PYGZhy{}\PYGZhy{}\PYGZhy{}\PYGZhy{}\PYGZhy{}\PYGZhy{}\PYGZhy{}\PYGZhy{}\PYGZhy{}\PYGZhy{}\PYGZhy{}\PYGZhy{}\PYGZhy{}\PYGZhy{}\PYGZhy{}\PYGZhy{}\PYGZhy{}\PYGZhy{}\PYGZhy{}\PYGZhy{}\PYGZhy{}+\PYGZhy{}\PYGZhy{}\PYGZhy{}\PYGZhy{}\PYGZhy{}\PYGZhy{}\PYGZhy{}\PYGZhy{}\PYGZhy{}\PYGZhy{}\PYGZhy{}\PYGZhy{}\PYGZhy{}\PYGZhy{}\PYGZhy{}\PYGZhy{}\PYGZhy{}\PYGZhy{}\PYGZhy{}\PYGZhy{}\PYGZhy{}\PYGZhy{}\PYGZhy{}\PYGZhy{}\PYGZhy{}\PYGZhy{}\PYGZhy{}\PYGZhy{}\PYGZhy{}\PYGZhy{}+
\end{Verbatim}

She can do PermGet on Bucket1, but she could before since she has the
permission explicitly set

\begin{Verbatim}[commandchars=\\\{\}]
\PYG{p}{(}\PYG{n}{tanodb}\PYG{p}{@}\PYG{l+m+mi}{127}\PYG{p}{.}\PYG{l+m+mi}{0}\PYG{p}{.}\PYG{l+m+mi}{0}\PYG{p}{.}\PYG{l+m+mi}{1}\PYG{p}{)}\PYG{l+m+mi}{26}\PYG{o}{\PYGZgt{}} \PYG{n+nn}{riak\PYGZus{}core\PYGZus{}security}\PYG{p}{:}\PYG{n+nf}{check\PYGZus{}permission}\PYG{p}{(}\PYG{p}{\PYGZob{}}\PYG{n+nv}{PermGet}\PYG{p}{,} \PYG{n+nv}{Bucket1}\PYG{p}{\PYGZcb{}}\PYG{p}{,} \PYG{n+nv}{Ctx1}\PYG{p}{)}\PYG{p}{.}
\PYG{p}{\PYGZob{}}\PYG{n}{true}\PYG{p}{,}\PYG{p}{\PYGZob{}}\PYG{n}{context}\PYG{p}{,}\PYG{o}{\PYGZlt{}}\PYG{o}{\PYGZlt{}}\PYG{l+s}{\PYGZdq{}}\PYG{l+s}{sandy}\PYG{l+s}{\PYGZdq{}}\PYG{o}{\PYGZgt{}}\PYG{o}{\PYGZgt{}}\PYG{p}{,}
           \PYG{p}{[}\PYG{p}{\PYGZob{}}\PYG{o}{\PYGZlt{}}\PYG{o}{\PYGZlt{}}\PYG{l+s}{\PYGZdq{}}\PYG{l+s}{bucket\PYGZus{}sandy}\PYG{l+s}{\PYGZdq{}}\PYG{o}{\PYGZgt{}}\PYG{o}{\PYGZgt{}}\PYG{p}{,}\PYG{p}{[}\PYG{l+s}{\PYGZdq{}}\PYG{l+s}{tanodb.get}\PYG{l+s}{\PYGZdq{}}\PYG{p}{]}\PYG{p}{\PYGZcb{}}\PYG{p}{]}\PYG{p}{,}
           \PYG{p}{\PYGZob{}}\PYG{l+m+mi}{1444}\PYG{p}{,}\PYG{l+m+mi}{659568}\PYG{p}{,}\PYG{l+m+mi}{837358}\PYG{p}{\PYGZcb{}}\PYG{p}{\PYGZcb{}}\PYG{p}{\PYGZcb{}}
\end{Verbatim}

Let's revoke it

\begin{Verbatim}[commandchars=\\\{\}]
\PYG{p}{(}\PYG{n}{tanodb}\PYG{p}{@}\PYG{l+m+mi}{127}\PYG{p}{.}\PYG{l+m+mi}{0}\PYG{p}{.}\PYG{l+m+mi}{0}\PYG{p}{.}\PYG{l+m+mi}{1}\PYG{p}{)}\PYG{l+m+mi}{27}\PYG{o}{\PYGZgt{}} \PYG{n+nn}{riak\PYGZus{}core\PYGZus{}security}\PYG{p}{:}\PYG{n+nf}{add\PYGZus{}revoke}\PYG{p}{(}\PYG{p}{[}\PYG{n+nv}{User1}\PYG{p}{]}\PYG{p}{,} \PYG{n+nv}{Bucket1}\PYG{p}{,} \PYG{p}{[}\PYG{n+nv}{PermGet}\PYG{p}{]}\PYG{p}{)}\PYG{p}{.}
\PYG{n}{ok}
\end{Verbatim}

Still can

\begin{Verbatim}[commandchars=\\\{\}]
\PYG{p}{(}\PYG{n}{tanodb}\PYG{p}{@}\PYG{l+m+mi}{127}\PYG{p}{.}\PYG{l+m+mi}{0}\PYG{p}{.}\PYG{l+m+mi}{0}\PYG{p}{.}\PYG{l+m+mi}{1}\PYG{p}{)}\PYG{l+m+mi}{28}\PYG{o}{\PYGZgt{}} \PYG{n+nn}{riak\PYGZus{}core\PYGZus{}security}\PYG{p}{:}\PYG{n+nf}{check\PYGZus{}permission}\PYG{p}{(}\PYG{p}{\PYGZob{}}\PYG{n+nv}{PermGet}\PYG{p}{,} \PYG{n+nv}{Bucket1}\PYG{p}{\PYGZcb{}}\PYG{p}{,} \PYG{n+nv}{Ctx1}\PYG{p}{)}\PYG{p}{.}
\PYG{p}{\PYGZob{}}\PYG{n}{true}\PYG{p}{,}\PYG{p}{\PYGZob{}}\PYG{n}{context}\PYG{p}{,}\PYG{o}{\PYGZlt{}}\PYG{o}{\PYGZlt{}}\PYG{l+s}{\PYGZdq{}}\PYG{l+s}{sandy}\PYG{l+s}{\PYGZdq{}}\PYG{o}{\PYGZgt{}}\PYG{o}{\PYGZgt{}}\PYG{p}{,}
           \PYG{p}{[}\PYG{p}{\PYGZob{}}\PYG{o}{\PYGZlt{}}\PYG{o}{\PYGZlt{}}\PYG{l+s}{\PYGZdq{}}\PYG{l+s}{bucket\PYGZus{}sandy}\PYG{l+s}{\PYGZdq{}}\PYG{o}{\PYGZgt{}}\PYG{o}{\PYGZgt{}}\PYG{p}{,}\PYG{p}{[}\PYG{l+s}{\PYGZdq{}}\PYG{l+s}{tanodb.get}\PYG{l+s}{\PYGZdq{}}\PYG{p}{]}\PYG{p}{\PYGZcb{}}\PYG{p}{]}\PYG{p}{,}
           \PYG{p}{\PYGZob{}}\PYG{l+m+mi}{1444}\PYG{p}{,}\PYG{l+m+mi}{659568}\PYG{p}{,}\PYG{l+m+mi}{847161}\PYG{p}{\PYGZcb{}}\PYG{p}{\PYGZcb{}}\PYG{p}{\PYGZcb{}}
\end{Verbatim}

But can't put on that bucket

\begin{Verbatim}[commandchars=\\\{\}]
\PYG{p}{(}\PYG{n}{tanodb}\PYG{p}{@}\PYG{l+m+mi}{127}\PYG{p}{.}\PYG{l+m+mi}{0}\PYG{p}{.}\PYG{l+m+mi}{0}\PYG{p}{.}\PYG{l+m+mi}{1}\PYG{p}{)}\PYG{l+m+mi}{29}\PYG{o}{\PYGZgt{}} \PYG{n+nn}{riak\PYGZus{}core\PYGZus{}security}\PYG{p}{:}\PYG{n+nf}{check\PYGZus{}permission}\PYG{p}{(}\PYG{p}{\PYGZob{}}\PYG{n+nv}{PermPut}\PYG{p}{,} \PYG{n+nv}{Bucket1}\PYG{p}{\PYGZcb{}}\PYG{p}{,} \PYG{n+nv}{Ctx1}\PYG{p}{)}\PYG{p}{.}
\PYG{p}{\PYGZob{}}\PYG{n}{false}\PYG{p}{,}\PYG{o}{\PYGZlt{}}\PYG{o}{\PYGZlt{}}\PYG{l+s}{\PYGZdq{}}\PYG{l+s}{Permission denied: User \PYGZsq{}sandy\PYGZsq{} does not have \PYGZsq{}tanodb.put\PYGZsq{} on bucket\PYGZus{}sandy}\PYG{l+s}{\PYGZdq{}}\PYG{o}{\PYGZgt{}}\PYG{o}{\PYGZgt{}}\PYG{p}{,}
   \PYG{p}{\PYGZob{}}\PYG{n}{context}\PYG{p}{,}\PYG{o}{\PYGZlt{}}\PYG{o}{\PYGZlt{}}\PYG{l+s}{\PYGZdq{}}\PYG{l+s}{sandy}\PYG{l+s}{\PYGZdq{}}\PYG{o}{\PYGZgt{}}\PYG{o}{\PYGZgt{}}\PYG{p}{,}
            \PYG{p}{[}\PYG{p}{\PYGZob{}}\PYG{o}{\PYGZlt{}}\PYG{o}{\PYGZlt{}}\PYG{l+s}{\PYGZdq{}}\PYG{l+s}{bucket\PYGZus{}sandy}\PYG{l+s}{\PYGZdq{}}\PYG{o}{\PYGZgt{}}\PYG{o}{\PYGZgt{}}\PYG{p}{,}\PYG{p}{[}\PYG{l+s}{\PYGZdq{}}\PYG{l+s}{tanodb.get}\PYG{l+s}{\PYGZdq{}}\PYG{p}{]}\PYG{p}{\PYGZcb{}}\PYG{p}{]}\PYG{p}{,}
            \PYG{p}{\PYGZob{}}\PYG{l+m+mi}{1444}\PYG{p}{,}\PYG{l+m+mi}{659568}\PYG{p}{,}\PYG{l+m+mi}{848204}\PYG{p}{\PYGZcb{}}\PYG{p}{\PYGZcb{}}\PYG{p}{\PYGZcb{}}
\end{Verbatim}

Now let's join User1 to some groups and try permissions

\begin{Verbatim}[commandchars=\\\{\}]
\PYG{p}{(}\PYG{n}{tanodb}\PYG{p}{@}\PYG{l+m+mi}{127}\PYG{p}{.}\PYG{l+m+mi}{0}\PYG{p}{.}\PYG{l+m+mi}{0}\PYG{p}{.}\PYG{l+m+mi}{1}\PYG{p}{)}\PYG{l+m+mi}{30}\PYG{o}{\PYGZgt{}} \PYG{n+nn}{riak\PYGZus{}core\PYGZus{}security}\PYG{p}{:}\PYG{n+nf}{alter\PYGZus{}user}\PYG{p}{(}\PYG{n+nv}{User1}\PYG{p}{,} \PYG{p}{[}\PYG{p}{\PYGZob{}}\PYG{l+s}{\PYGZdq{}}\PYG{l+s}{groups}\PYG{l+s}{\PYGZdq{}}\PYG{p}{,} \PYG{p}{[}\PYG{n+nv}{GroupWriter}\PYG{p}{]}\PYG{p}{\PYGZcb{}}\PYG{p}{]}\PYG{p}{)}\PYG{p}{.}
\PYG{n}{ok}
\end{Verbatim}

We can see User1 is a member of the group, but no more of GroupReader

\begin{Verbatim}[commandchars=\\\{\}]
\PYG{p}{(}\PYG{n}{tanodb}\PYG{p}{@}\PYG{l+m+mi}{127}\PYG{p}{.}\PYG{l+m+mi}{0}\PYG{p}{.}\PYG{l+m+mi}{0}\PYG{p}{.}\PYG{l+m+mi}{1}\PYG{p}{)}\PYG{l+m+mi}{31}\PYG{o}{\PYGZgt{}} \PYG{n+nn}{riak\PYGZus{}core\PYGZus{}security}\PYG{p}{:}\PYG{n+nf}{print\PYGZus{}user}\PYG{p}{(}\PYG{n+nv}{User1}\PYG{p}{)}\PYG{p}{.}
\PYG{n}{ok}
\end{Verbatim}

\begin{Verbatim}[commandchars=\\\{\}]
+\PYGZhy{}\PYGZhy{}\PYGZhy{}\PYGZhy{}\PYGZhy{}\PYGZhy{}\PYGZhy{}\PYGZhy{}\PYGZhy{}\PYGZhy{}+\PYGZhy{}\PYGZhy{}\PYGZhy{}\PYGZhy{}\PYGZhy{}\PYGZhy{}\PYGZhy{}\PYGZhy{}\PYGZhy{}\PYGZhy{}\PYGZhy{}\PYGZhy{}\PYGZhy{}\PYGZhy{}\PYGZhy{}+\PYGZhy{}\PYGZhy{}\PYGZhy{}\PYGZhy{}\PYGZhy{}\PYGZhy{}\PYGZhy{}\PYGZhy{}\PYGZhy{}\PYGZhy{}\PYGZhy{}\PYGZhy{}\PYGZhy{}\PYGZhy{}\PYGZhy{}\PYGZhy{}\PYGZhy{}\PYGZhy{}\PYGZhy{}\PYGZhy{}\PYGZhy{}\PYGZhy{}\PYGZhy{}\PYGZhy{}\PYGZhy{}\PYGZhy{}\PYGZhy{}\PYGZhy{}\PYGZhy{}\PYGZhy{}\PYGZhy{}\PYGZhy{}\PYGZhy{}\PYGZhy{}\PYGZhy{}\PYGZhy{}\PYGZhy{}\PYGZhy{}\PYGZhy{}\PYGZhy{}+\PYGZhy{}\PYGZhy{}\PYGZhy{}\PYGZhy{}\PYGZhy{}\PYGZhy{}\PYGZhy{}\PYGZhy{}\PYGZhy{}\PYGZhy{}\PYGZhy{}\PYGZhy{}\PYGZhy{}\PYGZhy{}\PYGZhy{}\PYGZhy{}\PYGZhy{}\PYGZhy{}\PYGZhy{}\PYGZhy{}\PYGZhy{}\PYGZhy{}\PYGZhy{}\PYGZhy{}\PYGZhy{}\PYGZhy{}\PYGZhy{}\PYGZhy{}\PYGZhy{}\PYGZhy{}+
\textbar{} username \textbar{}   member of   \textbar{}                password                \textbar{}           options            \textbar{}
+\PYGZhy{}\PYGZhy{}\PYGZhy{}\PYGZhy{}\PYGZhy{}\PYGZhy{}\PYGZhy{}\PYGZhy{}\PYGZhy{}\PYGZhy{}+\PYGZhy{}\PYGZhy{}\PYGZhy{}\PYGZhy{}\PYGZhy{}\PYGZhy{}\PYGZhy{}\PYGZhy{}\PYGZhy{}\PYGZhy{}\PYGZhy{}\PYGZhy{}\PYGZhy{}\PYGZhy{}\PYGZhy{}+\PYGZhy{}\PYGZhy{}\PYGZhy{}\PYGZhy{}\PYGZhy{}\PYGZhy{}\PYGZhy{}\PYGZhy{}\PYGZhy{}\PYGZhy{}\PYGZhy{}\PYGZhy{}\PYGZhy{}\PYGZhy{}\PYGZhy{}\PYGZhy{}\PYGZhy{}\PYGZhy{}\PYGZhy{}\PYGZhy{}\PYGZhy{}\PYGZhy{}\PYGZhy{}\PYGZhy{}\PYGZhy{}\PYGZhy{}\PYGZhy{}\PYGZhy{}\PYGZhy{}\PYGZhy{}\PYGZhy{}\PYGZhy{}\PYGZhy{}\PYGZhy{}\PYGZhy{}\PYGZhy{}\PYGZhy{}\PYGZhy{}\PYGZhy{}\PYGZhy{}+\PYGZhy{}\PYGZhy{}\PYGZhy{}\PYGZhy{}\PYGZhy{}\PYGZhy{}\PYGZhy{}\PYGZhy{}\PYGZhy{}\PYGZhy{}\PYGZhy{}\PYGZhy{}\PYGZhy{}\PYGZhy{}\PYGZhy{}\PYGZhy{}\PYGZhy{}\PYGZhy{}\PYGZhy{}\PYGZhy{}\PYGZhy{}\PYGZhy{}\PYGZhy{}\PYGZhy{}\PYGZhy{}\PYGZhy{}\PYGZhy{}\PYGZhy{}\PYGZhy{}\PYGZhy{}+
\textbar{}  sandy   \textbar{}    writers    \textbar{}9c8984b176e07eb7ba9ff1e3ada5a43ecb8a812e\textbar{}              []              \textbar{}
+\PYGZhy{}\PYGZhy{}\PYGZhy{}\PYGZhy{}\PYGZhy{}\PYGZhy{}\PYGZhy{}\PYGZhy{}\PYGZhy{}\PYGZhy{}+\PYGZhy{}\PYGZhy{}\PYGZhy{}\PYGZhy{}\PYGZhy{}\PYGZhy{}\PYGZhy{}\PYGZhy{}\PYGZhy{}\PYGZhy{}\PYGZhy{}\PYGZhy{}\PYGZhy{}\PYGZhy{}\PYGZhy{}+\PYGZhy{}\PYGZhy{}\PYGZhy{}\PYGZhy{}\PYGZhy{}\PYGZhy{}\PYGZhy{}\PYGZhy{}\PYGZhy{}\PYGZhy{}\PYGZhy{}\PYGZhy{}\PYGZhy{}\PYGZhy{}\PYGZhy{}\PYGZhy{}\PYGZhy{}\PYGZhy{}\PYGZhy{}\PYGZhy{}\PYGZhy{}\PYGZhy{}\PYGZhy{}\PYGZhy{}\PYGZhy{}\PYGZhy{}\PYGZhy{}\PYGZhy{}\PYGZhy{}\PYGZhy{}\PYGZhy{}\PYGZhy{}\PYGZhy{}\PYGZhy{}\PYGZhy{}\PYGZhy{}\PYGZhy{}\PYGZhy{}\PYGZhy{}\PYGZhy{}+\PYGZhy{}\PYGZhy{}\PYGZhy{}\PYGZhy{}\PYGZhy{}\PYGZhy{}\PYGZhy{}\PYGZhy{}\PYGZhy{}\PYGZhy{}\PYGZhy{}\PYGZhy{}\PYGZhy{}\PYGZhy{}\PYGZhy{}\PYGZhy{}\PYGZhy{}\PYGZhy{}\PYGZhy{}\PYGZhy{}\PYGZhy{}\PYGZhy{}\PYGZhy{}\PYGZhy{}\PYGZhy{}\PYGZhy{}\PYGZhy{}\PYGZhy{}\PYGZhy{}\PYGZhy{}+
\end{Verbatim}

User1 can now put on that bucket

\begin{Verbatim}[commandchars=\\\{\}]
\PYG{p}{(}\PYG{n}{tanodb}\PYG{p}{@}\PYG{l+m+mi}{127}\PYG{p}{.}\PYG{l+m+mi}{0}\PYG{p}{.}\PYG{l+m+mi}{0}\PYG{p}{.}\PYG{l+m+mi}{1}\PYG{p}{)}\PYG{l+m+mi}{32}\PYG{o}{\PYGZgt{}} \PYG{n+nn}{riak\PYGZus{}core\PYGZus{}security}\PYG{p}{:}\PYG{n+nf}{check\PYGZus{}permission}\PYG{p}{(}\PYG{p}{\PYGZob{}}\PYG{n+nv}{PermPut}\PYG{p}{,} \PYG{n+nv}{Bucket1}\PYG{p}{\PYGZcb{}}\PYG{p}{,} \PYG{n+nv}{Ctx1}\PYG{p}{)}\PYG{p}{.}
\PYG{p}{\PYGZob{}}\PYG{n}{true}\PYG{p}{,}\PYG{p}{\PYGZob{}}\PYG{n}{context}\PYG{p}{,}\PYG{o}{\PYGZlt{}}\PYG{o}{\PYGZlt{}}\PYG{l+s}{\PYGZdq{}}\PYG{l+s}{sandy}\PYG{l+s}{\PYGZdq{}}\PYG{o}{\PYGZgt{}}\PYG{o}{\PYGZgt{}}\PYG{p}{,}
           \PYG{p}{[}\PYG{p}{\PYGZob{}}\PYG{o}{\PYGZlt{}}\PYG{o}{\PYGZlt{}}\PYG{l+s}{\PYGZdq{}}\PYG{l+s}{bucket\PYGZus{}sandy}\PYG{l+s}{\PYGZdq{}}\PYG{o}{\PYGZgt{}}\PYG{o}{\PYGZgt{}}\PYG{p}{,}\PYG{p}{[}\PYG{l+s}{\PYGZdq{}}\PYG{l+s}{tanodb.get}\PYG{l+s}{\PYGZdq{}}\PYG{p}{,}\PYG{l+s}{\PYGZdq{}}\PYG{l+s}{tanodb.put}\PYG{l+s}{\PYGZdq{}}\PYG{p}{]}\PYG{p}{\PYGZcb{}}\PYG{p}{]}\PYG{p}{,}
           \PYG{p}{\PYGZob{}}\PYG{l+m+mi}{1444}\PYG{p}{,}\PYG{l+m+mi}{659568}\PYG{p}{,}\PYG{l+m+mi}{859448}\PYG{p}{\PYGZcb{}}\PYG{p}{\PYGZcb{}}\PYG{p}{\PYGZcb{}}
\end{Verbatim}

Still can get since GroupWriter is member of the group GroupReader

\begin{Verbatim}[commandchars=\\\{\}]
\PYG{p}{(}\PYG{n}{tanodb}\PYG{p}{@}\PYG{l+m+mi}{127}\PYG{p}{.}\PYG{l+m+mi}{0}\PYG{p}{.}\PYG{l+m+mi}{0}\PYG{p}{.}\PYG{l+m+mi}{1}\PYG{p}{)}\PYG{l+m+mi}{33}\PYG{o}{\PYGZgt{}} \PYG{n+nn}{riak\PYGZus{}core\PYGZus{}security}\PYG{p}{:}\PYG{n+nf}{check\PYGZus{}permission}\PYG{p}{(}\PYG{p}{\PYGZob{}}\PYG{n+nv}{PermGet}\PYG{p}{,} \PYG{n+nv}{Bucket1}\PYG{p}{\PYGZcb{}}\PYG{p}{,} \PYG{n+nv}{Ctx1}\PYG{p}{)}\PYG{p}{.}
\PYG{p}{\PYGZob{}}\PYG{n}{true}\PYG{p}{,}\PYG{p}{\PYGZob{}}\PYG{n}{context}\PYG{p}{,}\PYG{o}{\PYGZlt{}}\PYG{o}{\PYGZlt{}}\PYG{l+s}{\PYGZdq{}}\PYG{l+s}{sandy}\PYG{l+s}{\PYGZdq{}}\PYG{o}{\PYGZgt{}}\PYG{o}{\PYGZgt{}}\PYG{p}{,}
           \PYG{p}{[}\PYG{p}{\PYGZob{}}\PYG{o}{\PYGZlt{}}\PYG{o}{\PYGZlt{}}\PYG{l+s}{\PYGZdq{}}\PYG{l+s}{bucket\PYGZus{}sandy}\PYG{l+s}{\PYGZdq{}}\PYG{o}{\PYGZgt{}}\PYG{o}{\PYGZgt{}}\PYG{p}{,}\PYG{p}{[}\PYG{l+s}{\PYGZdq{}}\PYG{l+s}{tanodb.get}\PYG{l+s}{\PYGZdq{}}\PYG{p}{,}\PYG{l+s}{\PYGZdq{}}\PYG{l+s}{tanodb.put}\PYG{l+s}{\PYGZdq{}}\PYG{p}{]}\PYG{p}{\PYGZcb{}}\PYG{p}{]}\PYG{p}{,}
           \PYG{p}{\PYGZob{}}\PYG{l+m+mi}{1444}\PYG{p}{,}\PYG{l+m+mi}{659568}\PYG{p}{,}\PYG{l+m+mi}{860961}\PYG{p}{\PYGZcb{}}\PYG{p}{\PYGZcb{}}\PYG{p}{\PYGZcb{}}
\end{Verbatim}

Now let's add a new grant to GroupReader so they can list the bucket

\begin{Verbatim}[commandchars=\\\{\}]
\PYG{p}{(}\PYG{n}{tanodb}\PYG{p}{@}\PYG{l+m+mi}{127}\PYG{p}{.}\PYG{l+m+mi}{0}\PYG{p}{.}\PYG{l+m+mi}{0}\PYG{p}{.}\PYG{l+m+mi}{1}\PYG{p}{)}\PYG{l+m+mi}{34}\PYG{o}{\PYGZgt{}} \PYG{n+nn}{riak\PYGZus{}core\PYGZus{}security}\PYG{p}{:}\PYG{n+nf}{add\PYGZus{}grant}\PYG{p}{(}\PYG{p}{[}\PYG{n+nv}{GroupReader}\PYG{p}{]}\PYG{p}{,} \PYG{n+nv}{Bucket1}\PYG{p}{,} \PYG{p}{[}\PYG{n+nv}{PermList}\PYG{p}{]}\PYG{p}{)}\PYG{p}{.}
\PYG{n}{ok}
\end{Verbatim}

Now User1 has the list permission since she is a member of GroupWriter
which is a member of GroupReader who has permissions to list Bucket1

\begin{Verbatim}[commandchars=\\\{\}]
\PYG{p}{(}\PYG{n}{tanodb}\PYG{p}{@}\PYG{l+m+mi}{127}\PYG{p}{.}\PYG{l+m+mi}{0}\PYG{p}{.}\PYG{l+m+mi}{0}\PYG{p}{.}\PYG{l+m+mi}{1}\PYG{p}{)}\PYG{l+m+mi}{35}\PYG{o}{\PYGZgt{}} \PYG{n+nn}{riak\PYGZus{}core\PYGZus{}security}\PYG{p}{:}\PYG{n+nf}{check\PYGZus{}permission}\PYG{p}{(}\PYG{p}{\PYGZob{}}\PYG{n+nv}{PermList}\PYG{p}{,} \PYG{n+nv}{Bucket1}\PYG{p}{\PYGZcb{}}\PYG{p}{,} \PYG{n+nv}{Ctx1}\PYG{p}{)}\PYG{p}{.}
\PYG{p}{\PYGZob{}}\PYG{n}{true}\PYG{p}{,}\PYG{p}{\PYGZob{}}\PYG{n}{context}\PYG{p}{,}\PYG{o}{\PYGZlt{}}\PYG{o}{\PYGZlt{}}\PYG{l+s}{\PYGZdq{}}\PYG{l+s}{sandy}\PYG{l+s}{\PYGZdq{}}\PYG{o}{\PYGZgt{}}\PYG{o}{\PYGZgt{}}\PYG{p}{,}
           \PYG{p}{[}\PYG{p}{\PYGZob{}}\PYG{o}{\PYGZlt{}}\PYG{o}{\PYGZlt{}}\PYG{l+s}{\PYGZdq{}}\PYG{l+s}{bucket\PYGZus{}sandy}\PYG{l+s}{\PYGZdq{}}\PYG{o}{\PYGZgt{}}\PYG{o}{\PYGZgt{}}\PYG{p}{,}
             \PYG{p}{[}\PYG{l+s}{\PYGZdq{}}\PYG{l+s}{tanodb.get}\PYG{l+s}{\PYGZdq{}}\PYG{p}{,}\PYG{l+s}{\PYGZdq{}}\PYG{l+s}{tanodb.list}\PYG{l+s}{\PYGZdq{}}\PYG{p}{,}\PYG{l+s}{\PYGZdq{}}\PYG{l+s}{tanodb.put}\PYG{l+s}{\PYGZdq{}}\PYG{p}{]}\PYG{p}{\PYGZcb{}}\PYG{p}{]}\PYG{p}{,}
           \PYG{p}{\PYGZob{}}\PYG{l+m+mi}{1444}\PYG{p}{,}\PYG{l+m+mi}{659568}\PYG{p}{,}\PYG{l+m+mi}{872565}\PYG{p}{\PYGZcb{}}\PYG{p}{\PYGZcb{}}\PYG{p}{\PYGZcb{}}
\end{Verbatim}

Let's remove GroupReader membership from GroupWriter

\begin{Verbatim}[commandchars=\\\{\}]
\PYG{p}{(}\PYG{n}{tanodb}\PYG{p}{@}\PYG{l+m+mi}{127}\PYG{p}{.}\PYG{l+m+mi}{0}\PYG{p}{.}\PYG{l+m+mi}{0}\PYG{p}{.}\PYG{l+m+mi}{1}\PYG{p}{)}\PYG{l+m+mi}{36}\PYG{o}{\PYGZgt{}} \PYG{n+nn}{riak\PYGZus{}core\PYGZus{}security}\PYG{p}{:}\PYG{n+nf}{alter\PYGZus{}group}\PYG{p}{(}\PYG{n+nv}{GroupWriter}\PYG{p}{,} \PYG{p}{[}\PYG{p}{\PYGZob{}}\PYG{l+s}{\PYGZdq{}}\PYG{l+s}{groups}\PYG{l+s}{\PYGZdq{}}\PYG{p}{,} \PYG{p}{[}\PYG{p}{]}\PYG{p}{\PYGZcb{}}\PYG{p}{]}\PYG{p}{)}\PYG{p}{.}
\PYG{n}{ok}
\end{Verbatim}

Now User1 can't list on Bucket1 anymore

\begin{Verbatim}[commandchars=\\\{\}]
\PYG{p}{(}\PYG{n}{tanodb}\PYG{p}{@}\PYG{l+m+mi}{127}\PYG{p}{.}\PYG{l+m+mi}{0}\PYG{p}{.}\PYG{l+m+mi}{0}\PYG{p}{.}\PYG{l+m+mi}{1}\PYG{p}{)}\PYG{l+m+mi}{37}\PYG{o}{\PYGZgt{}} \PYG{n+nn}{riak\PYGZus{}core\PYGZus{}security}\PYG{p}{:}\PYG{n+nf}{check\PYGZus{}permission}\PYG{p}{(}\PYG{p}{\PYGZob{}}\PYG{n+nv}{PermList}\PYG{p}{,} \PYG{n+nv}{Bucket1}\PYG{p}{\PYGZcb{}}\PYG{p}{,} \PYG{n+nv}{Ctx1}\PYG{p}{)}\PYG{p}{.}
\PYG{p}{\PYGZob{}}\PYG{n}{false}\PYG{p}{,}\PYG{o}{\PYGZlt{}}\PYG{o}{\PYGZlt{}}\PYG{l+s}{\PYGZdq{}}\PYG{l+s}{Permission denied: User \PYGZsq{}sandy\PYGZsq{} does not have \PYGZsq{}tanodb.list\PYGZsq{} on bucket\PYGZus{}sandy}\PYG{l+s}{\PYGZdq{}}\PYG{o}{\PYGZgt{}}\PYG{o}{\PYGZgt{}}\PYG{p}{,}
   \PYG{p}{\PYGZob{}}\PYG{n}{context}\PYG{p}{,}\PYG{o}{\PYGZlt{}}\PYG{o}{\PYGZlt{}}\PYG{l+s}{\PYGZdq{}}\PYG{l+s}{sandy}\PYG{l+s}{\PYGZdq{}}\PYG{o}{\PYGZgt{}}\PYG{o}{\PYGZgt{}}\PYG{p}{,}
            \PYG{p}{[}\PYG{p}{\PYGZob{}}\PYG{o}{\PYGZlt{}}\PYG{o}{\PYGZlt{}}\PYG{l+s}{\PYGZdq{}}\PYG{l+s}{bucket\PYGZus{}sandy}\PYG{l+s}{\PYGZdq{}}\PYG{o}{\PYGZgt{}}\PYG{o}{\PYGZgt{}}\PYG{p}{,}\PYG{p}{[}\PYG{l+s}{\PYGZdq{}}\PYG{l+s}{tanodb.put}\PYG{l+s}{\PYGZdq{}}\PYG{p}{]}\PYG{p}{\PYGZcb{}}\PYG{p}{]}\PYG{p}{,}
            \PYG{p}{\PYGZob{}}\PYG{l+m+mi}{1444}\PYG{p}{,}\PYG{l+m+mi}{659568}\PYG{p}{,}\PYG{l+m+mi}{881585}\PYG{p}{\PYGZcb{}}\PYG{p}{\PYGZcb{}}\PYG{p}{\PYGZcb{}}
\end{Verbatim}

Let's try one more thing, add GroupWriter to GroupReader

\begin{Verbatim}[commandchars=\\\{\}]
\PYG{p}{(}\PYG{n}{tanodb}\PYG{p}{@}\PYG{l+m+mi}{127}\PYG{p}{.}\PYG{l+m+mi}{0}\PYG{p}{.}\PYG{l+m+mi}{0}\PYG{p}{.}\PYG{l+m+mi}{1}\PYG{p}{)}\PYG{l+m+mi}{38}\PYG{o}{\PYGZgt{}} \PYG{n+nn}{riak\PYGZus{}core\PYGZus{}security}\PYG{p}{:}\PYG{n+nf}{alter\PYGZus{}group}\PYG{p}{(}\PYG{n+nv}{GroupWriter}\PYG{p}{,} \PYG{p}{[}\PYG{p}{\PYGZob{}}\PYG{l+s}{\PYGZdq{}}\PYG{l+s}{groups}\PYG{l+s}{\PYGZdq{}}\PYG{p}{,} \PYG{p}{[}\PYG{n+nv}{GroupReader}\PYG{p}{]}\PYG{p}{\PYGZcb{}}\PYG{p}{]}\PYG{p}{)}\PYG{p}{.}
\PYG{n}{ok}
\end{Verbatim}

This works again

\begin{Verbatim}[commandchars=\\\{\}]
\PYG{p}{(}\PYG{n}{tanodb}\PYG{p}{@}\PYG{l+m+mi}{127}\PYG{p}{.}\PYG{l+m+mi}{0}\PYG{p}{.}\PYG{l+m+mi}{0}\PYG{p}{.}\PYG{l+m+mi}{1}\PYG{p}{)}\PYG{l+m+mi}{39}\PYG{o}{\PYGZgt{}} \PYG{n+nn}{riak\PYGZus{}core\PYGZus{}security}\PYG{p}{:}\PYG{n+nf}{check\PYGZus{}permission}\PYG{p}{(}\PYG{p}{\PYGZob{}}\PYG{n+nv}{PermList}\PYG{p}{,} \PYG{n+nv}{Bucket1}\PYG{p}{\PYGZcb{}}\PYG{p}{,} \PYG{n+nv}{Ctx1}\PYG{p}{)}\PYG{p}{.}
\PYG{p}{\PYGZob{}}\PYG{n}{true}\PYG{p}{,}\PYG{p}{\PYGZob{}}\PYG{n}{context}\PYG{p}{,}\PYG{o}{\PYGZlt{}}\PYG{o}{\PYGZlt{}}\PYG{l+s}{\PYGZdq{}}\PYG{l+s}{sandy}\PYG{l+s}{\PYGZdq{}}\PYG{o}{\PYGZgt{}}\PYG{o}{\PYGZgt{}}\PYG{p}{,}
           \PYG{p}{[}\PYG{p}{\PYGZob{}}\PYG{o}{\PYGZlt{}}\PYG{o}{\PYGZlt{}}\PYG{l+s}{\PYGZdq{}}\PYG{l+s}{bucket\PYGZus{}sandy}\PYG{l+s}{\PYGZdq{}}\PYG{o}{\PYGZgt{}}\PYG{o}{\PYGZgt{}}\PYG{p}{,}
             \PYG{p}{[}\PYG{l+s}{\PYGZdq{}}\PYG{l+s}{tanodb.get}\PYG{l+s}{\PYGZdq{}}\PYG{p}{,}\PYG{l+s}{\PYGZdq{}}\PYG{l+s}{tanodb.list}\PYG{l+s}{\PYGZdq{}}\PYG{p}{,}\PYG{l+s}{\PYGZdq{}}\PYG{l+s}{tanodb.put}\PYG{l+s}{\PYGZdq{}}\PYG{p}{]}\PYG{p}{\PYGZcb{}}\PYG{p}{]}\PYG{p}{,}
           \PYG{p}{\PYGZob{}}\PYG{l+m+mi}{1444}\PYG{p}{,}\PYG{l+m+mi}{659568}\PYG{p}{,}\PYG{l+m+mi}{890698}\PYG{p}{\PYGZcb{}}\PYG{p}{\PYGZcb{}}\PYG{p}{\PYGZcb{}}
\end{Verbatim}

Let's now remove GroupReader completely

\begin{Verbatim}[commandchars=\\\{\}]
\PYG{p}{(}\PYG{n}{tanodb}\PYG{p}{@}\PYG{l+m+mi}{127}\PYG{p}{.}\PYG{l+m+mi}{0}\PYG{p}{.}\PYG{l+m+mi}{0}\PYG{p}{.}\PYG{l+m+mi}{1}\PYG{p}{)}\PYG{l+m+mi}{40}\PYG{o}{\PYGZgt{}} \PYG{n+nn}{riak\PYGZus{}core\PYGZus{}security}\PYG{p}{:}\PYG{n+nf}{del\PYGZus{}group}\PYG{p}{(}\PYG{n+nv}{GroupReader}\PYG{p}{)}\PYG{p}{.}
\PYG{n}{ok}
\end{Verbatim}

This should fail again

\begin{Verbatim}[commandchars=\\\{\}]
\PYG{p}{(}\PYG{n}{tanodb}\PYG{p}{@}\PYG{l+m+mi}{127}\PYG{p}{.}\PYG{l+m+mi}{0}\PYG{p}{.}\PYG{l+m+mi}{0}\PYG{p}{.}\PYG{l+m+mi}{1}\PYG{p}{)}\PYG{l+m+mi}{41}\PYG{o}{\PYGZgt{}} \PYG{n+nn}{riak\PYGZus{}core\PYGZus{}security}\PYG{p}{:}\PYG{n+nf}{check\PYGZus{}permission}\PYG{p}{(}\PYG{p}{\PYGZob{}}\PYG{n+nv}{PermList}\PYG{p}{,} \PYG{n+nv}{Bucket1}\PYG{p}{\PYGZcb{}}\PYG{p}{,} \PYG{n+nv}{Ctx1}\PYG{p}{)}\PYG{p}{.}
\PYG{p}{\PYGZob{}}\PYG{n}{false}\PYG{p}{,}\PYG{o}{\PYGZlt{}}\PYG{o}{\PYGZlt{}}\PYG{l+s}{\PYGZdq{}}\PYG{l+s}{Permission denied: User \PYGZsq{}sandy\PYGZsq{} does not have \PYGZsq{}tanodb.list\PYGZsq{} on bucket\PYGZus{}sandy}\PYG{l+s}{\PYGZdq{}}\PYG{o}{\PYGZgt{}}\PYG{o}{\PYGZgt{}}\PYG{p}{,}
   \PYG{p}{\PYGZob{}}\PYG{n}{context}\PYG{p}{,}\PYG{o}{\PYGZlt{}}\PYG{o}{\PYGZlt{}}\PYG{l+s}{\PYGZdq{}}\PYG{l+s}{sandy}\PYG{l+s}{\PYGZdq{}}\PYG{o}{\PYGZgt{}}\PYG{o}{\PYGZgt{}}\PYG{p}{,}
            \PYG{p}{[}\PYG{p}{\PYGZob{}}\PYG{o}{\PYGZlt{}}\PYG{o}{\PYGZlt{}}\PYG{l+s}{\PYGZdq{}}\PYG{l+s}{bucket\PYGZus{}sandy}\PYG{l+s}{\PYGZdq{}}\PYG{o}{\PYGZgt{}}\PYG{o}{\PYGZgt{}}\PYG{p}{,}\PYG{p}{[}\PYG{l+s}{\PYGZdq{}}\PYG{l+s}{tanodb.put}\PYG{l+s}{\PYGZdq{}}\PYG{p}{]}\PYG{p}{\PYGZcb{}}\PYG{p}{]}\PYG{p}{,}
            \PYG{p}{\PYGZob{}}\PYG{l+m+mi}{1444}\PYG{p}{,}\PYG{l+m+mi}{659568}\PYG{p}{,}\PYG{l+m+mi}{914573}\PYG{p}{\PYGZcb{}}\PYG{p}{\PYGZcb{}}\PYG{p}{\PYGZcb{}}
\end{Verbatim}

Let's clean everything up

\begin{Verbatim}[commandchars=\\\{\}]
\PYG{p}{(}\PYG{n}{tanodb}\PYG{p}{@}\PYG{l+m+mi}{127}\PYG{p}{.}\PYG{l+m+mi}{0}\PYG{p}{.}\PYG{l+m+mi}{0}\PYG{p}{.}\PYG{l+m+mi}{1}\PYG{p}{)}\PYG{l+m+mi}{42}\PYG{o}{\PYGZgt{}} \PYG{n+nn}{riak\PYGZus{}core\PYGZus{}security}\PYG{p}{:}\PYG{n+nf}{del\PYGZus{}group}\PYG{p}{(}\PYG{n+nv}{GroupWriter}\PYG{p}{)}\PYG{p}{.}
\PYG{n+nf}{ok}

\PYG{p}{(}\PYG{n}{tanodb}\PYG{p}{@}\PYG{l+m+mi}{127}\PYG{p}{.}\PYG{l+m+mi}{0}\PYG{p}{.}\PYG{l+m+mi}{0}\PYG{p}{.}\PYG{l+m+mi}{1}\PYG{p}{)}\PYG{l+m+mi}{43}\PYG{o}{\PYGZgt{}} \PYG{n+nn}{riak\PYGZus{}core\PYGZus{}security}\PYG{p}{:}\PYG{n+nf}{del\PYGZus{}user}\PYG{p}{(}\PYG{n+nv}{User1}\PYG{p}{)}\PYG{p}{.}
\PYG{n+nf}{ok}

\PYG{p}{(}\PYG{n}{tanodb}\PYG{p}{@}\PYG{l+m+mi}{127}\PYG{p}{.}\PYG{l+m+mi}{0}\PYG{p}{.}\PYG{l+m+mi}{0}\PYG{p}{.}\PYG{l+m+mi}{1}\PYG{p}{)}\PYG{l+m+mi}{44}\PYG{o}{\PYGZgt{}} \PYG{n+nn}{riak\PYGZus{}core\PYGZus{}security}\PYG{p}{:}\PYG{n+nf}{del\PYGZus{}source}\PYG{p}{(}\PYG{n}{all}\PYG{p}{,} \PYG{n+nv}{Source1}\PYG{p}{)}\PYG{p}{.}
\PYG{n}{ok}
\end{Verbatim}

If you want to create a user that is member a more than one group at the same time in the same add\_user call you have to pass a string with comma separated names of the groups the user is going to be member of, like this:

\begin{Verbatim}[commandchars=\\\{\}]
\PYG{n+nn}{riak\PYGZus{}core\PYGZus{}security}\PYG{p}{:}\PYG{n+nf}{add\PYGZus{}user}\PYG{p}{(}\PYG{n+nv}{User1}\PYG{p}{,} \PYG{p}{[}\PYG{p}{\PYGZob{}}\PYG{l+s}{\PYGZdq{}}\PYG{l+s}{password}\PYG{l+s}{\PYGZdq{}}\PYG{p}{,} \PYG{n+nb}{binary\PYGZus{}to\PYGZus{}list}\PYG{p}{(}\PYG{n+nv}{Pass1}\PYG{p}{)}\PYG{p}{\PYGZcb{}}\PYG{p}{,} \PYG{p}{\PYGZob{}}\PYG{l+s}{\PYGZdq{}}\PYG{l+s}{groups}\PYG{l+s}{\PYGZdq{}}\PYG{p}{,} \PYG{l+s}{\PYGZdq{}}\PYG{l+s}{readers,writers}\PYG{l+s}{\PYGZdq{}}\PYG{p}{\PYGZcb{}}\PYG{p}{]}\PYG{p}{)}\PYG{p}{.}
\end{Verbatim}

If you want to retry from scratch removing all state you can do the following:

\begin{Verbatim}[commandchars=\\\{\}]
rm \PYGZhy{}rf \PYGZus{}build/default/rel
rebar3 release
rebar3 run
\end{Verbatim}



\renewcommand{\indexname}{Index}
\printindex
\end{document}
